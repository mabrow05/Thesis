\chapter{Introduction to the theory of neutron $\beta$-decay}
\label{ch:Introduction}

The purpose of this dissertation is to describe in detail
the methods used in extracting the $\beta$-decay asymmetry parameter
for the UCNA Experiment. This chapter hopes to motivate the inception
of the parameter of interest and its role in the theory of $\beta$-decay.
Also introduced are characteristics of the neutron itself with an
emphasis on ultracold neutrons (UCN), the namesake of the UCNA experiment.

%%%%%%%%%%%%%%%%%%%%%%%%%%%%%%%%%%%%%%%%%%%%%%%%%%%%%%%%%%%%%%%%%%%%%%%%%%%%%%%
%%%%%%%%%%%%%%%%%%%%%%%%%%%%%%%%%%%%%%%%%%%%%%%%%%%%%%%%%%%%%%%%%%%%%%%%%%%%%%%

\section{The Standard Model}
The Standard Model of particle physics encompasses everything we know about
interactions between particles and describes nature using the most fundamental
building blocks yet discovered. This section is used as an introduction
to these building blocks, the quarks and leptons, and to the mediators of the
interactions between them, to preface the upcoming descriptions of the neutron
and $\beta$-decay.

\subsection{The Building Blocks}

\subsection{The Forces}

\subsection{Symmetries}



%%%%%%%%%%%%%%%%%%%%%%%%%%%%%%%%%%%%%%%%%%%%%%%%%%%%%%%%%%%%%%%%%%%%%%%%%%%%%%%
%%%%%%%%%%%%%%%%%%%%%%%%%%%%%%%%%%%%%%%%%%%%%%%%%%%%%%%%%%%%%%%%%%%%%%%%%%%%%%%

\section{$\beta$-decay of the neutron}

\subsection{Properties of the neutron}
\label{ssec:neutronProperties}
Let us take a moment to give a very brief introduction to the neutron to motivate the
upcoming sections. This will also serve as a useful tool for those who are
unfamiliar with nuclear and field theories, as the following sections become
quite technical. The majority of this dissertation can be read and mostly understood
without much knowledge of the theoretical description behind neutron $\beta$-decay,
so the properties of the neutron are a natural starting point.

The neutron is a net neutrally charged composite particle. The terms net and
composite hint at the inner structure of the neutron, made up of fundamental
particles called quarks, in this case two down ($d$) quarks and one up ($u$) quark.
The quarks do carry charge, with the $u$ charge $+2/3$ and the $d$ charge
$-1/3$, so that the net charge of the neutron is zero. This can be compared
to the proton, another composite particle made up of three quarks (two $u$ and
one $d$ quark), whose net charge is $+1$ and the other nucleon found
within a nucleaus along with the neutron. Four other quarks exist,
the charm, strange, bottom, and top in order of increasing mass. The neutron
is the second lightest three-quark composite particle (baryon) 
behind only the proton.

The free neutron undergoes $\beta$-decay, defined as a transition from the neutron
to a proton, electron, and electron anti-neutrino,
\begin{equation*}
  n\rightarrow p + e^- + \bar{\nu}_{e}.
\end{equation*}
\noindent The three-body decay gives rise to a continuous energy spectrum
for the proton, electron, and anti-neutrino by conservation of energy and momentum
as seen in figure \ref{fig:spectralShapes}. The lifetime of the free neutron is
approximately fifteen minutes.

The neutron is spin $1/2$ particle with magnetic moment...

\subsection{Fermi's Theory of $\beta$-decay}
After Pauli postulated the existence of the neutrino to explain the continuous energy
distribution of the $\beta$-decay electron, Fermi attempted to theoretically describe
the process in a similar manner to

\subsection{Parity Violation in Weak Decays}

\subsection{Correlation Coefficients}
Taking the interaction Hamiltonian from Lee and Yang with all generalized terms included,
Jackson, Trieman, and Wyld \cite{jackson1957a,jackson1957b} first derived an expression
for the differential decay rate for polarized nuclei as a function of the emitted electron momentum
and spin, the neutrino momentum, and the nuclear spin of the decaying nucleus.
Ebel and Feldman \cite{ebel1957} added terms to the expression of Jackson, Trieman, and Wyld,
and under the assumption that the spin of the mother nucleus and the spin of the outgoing electron
are observable this gives
%
\begin{multline}
  \frac{d\Gamma}{dE_e dE_\nu d\Omega_e d\Omega_\nu} = \frac{1}{2} \frac{F(\pm Z, E_e)}{\big( 2\pi \big)^5}
  p_e E_e \big( E^0 - E_e \big)^2 \\ \times \xi 
  \Bigg\{ 1 + a\frac{\boldsymbol{p_e \cdot p_\nu}}{E_e E_\nu} + b\frac{m_e}{E_e} 
  + \frac{\boldsymbol{\langle J \rangle}}{J} \Bigg[ A\frac{\boldsymbol{p_e}}{E_e}
    + B\frac{\boldsymbol{p_\nu}}{E_\nu} + D\frac{\boldsymbol{p_e \times p_\nu}}{E_e E_\nu}\Bigg] \\
  + \Bigg[ \frac{J(J+1)-3\langle (\boldsymbol{J \cdot \hat{\jmath}})^2 \rangle}{J(2J-1)} \Bigg]
  \Bigg( c\Bigg[ \frac{\boldsymbol{p_e} \times \boldsymbol{p_\nu}}{3E_eE_\nu} -
    \frac{(\boldsymbol{p_e\cdot \hat{\jmath}})(\boldsymbol{p_\nu\cdot \hat{\jmath}}) }{E_eE_\nu} \Bigg]
  + I \Bigg[ \frac{1}{3}\frac{\boldsymbol{\sigma \cdot p_\nu}}{E_\nu}
    - \frac{(\boldsymbol{\sigma \cdot \hat{\jmath}})(\boldsymbol{p_\nu \cdot \hat{\jmath}})}{E_\nu} \Bigg] \\
  + K'\frac{\boldsymbol{\sigma \cdot p_e}}{E_e+m_e} \Bigg[ \frac{1}{3}\frac{\boldsymbol{p_e \cdot p_\nu}}{E_e E_\nu}
    - \frac{(\boldsymbol{p_e \cdot \hat{\jmath}})(\boldsymbol{p_\nu \cdot \hat{\jmath}})}{E_e E_\nu} \Bigg] 
  + M \Bigg[ \frac{1}{3}\frac{\boldsymbol{\sigma \cdot p_e \times p_\nu}}{E_e E_\nu}
    - \frac{(\boldsymbol{\sigma \cdot \hat{\jmath}})(\boldsymbol{\hat{\jmath} \cdot p_e \times p_\nu })}{E_e E_\nu} \Bigg] \Bigg) \\
  + \boldsymbol{\sigma} \cdot \Bigg[ N\frac{\langle \boldsymbol{J} \rangle}{J}
    + Q\frac{\boldsymbol{p_e}}{E_e+m_e}\Bigg(\frac{\langle \boldsymbol{J} \rangle}{J}\boldsymbol{\cdot} \frac{\boldsymbol{p_e}}{E_e}\Bigg)
    + R\frac{\langle \boldsymbol{J} \rangle}{J}\boldsymbol{\times} \frac{\boldsymbol{p_e}}{E_e}
    + S\frac{\langle \boldsymbol{J} \rangle}{J} \frac{\boldsymbol{p_e\cdot p_\nu}}{E_e E_\nu} \\
    + T\frac{\boldsymbol{p_e}}{E_e}\frac{\langle \boldsymbol{J} \rangle}{J} \boldsymbol{\cdot} \frac{\boldsymbol{p_\nu}}{E_\nu}
    + U\frac{\boldsymbol{p_\nu}}{E_\nu}\frac{\langle \boldsymbol{J} \rangle}{J} \boldsymbol{\cdot} \frac{\boldsymbol{p_e}}{E_e}
    + W\frac{\boldsymbol{p_e}}{E_e+m_e}\frac{\langle \boldsymbol{J} \rangle}{J} \boldsymbol{\cdot} \frac{\boldsymbol{p_e \times p_\nu}}{E_e E_\nu}
    \Bigg]
  + V\frac{\langle \boldsymbol{J} \rangle}{J} \boldsymbol{\cdot} \frac{\boldsymbol{\sigma \times p_\nu}}{E_\nu}
  \Bigg\}
  \label{eq:jackson}
\end{multline}
%
where $F(\pm Z, E_e)$ is the Fermi function, $E$ is the energy of a given particle, $E^0$ is the endpoint
energy of the electron, $\boldsymbol{p}$ is the particle momentum, $\boldsymbol{J}$ is the spin of the
decaying nucleus (or nucleon), and $\sigma$ is the spin of the electron. All of the correlation coefficients
are functions of the coupling constants in the weak Lagrangian (see equation \ref{eq:leeyang}),
as is $\xi$. The correlation coefficients and $\xi$ are also functions of Fermi and Gamow-Teller
transition amplitudes. For the complete defitions see
\cite{jackson1957a,jackson1957b,ebel1957}.

In the case of $\beta$-decay of polarized free neutrons and assuming the electron spin is undetectable,
equation \ref{eq:jackson} simplifies drastically,
%
\begin{multline}
  \frac{d\Gamma}{dE_e dE_\nu d\Omega_e d\Omega_\nu} = \frac{1}{2} \frac{F(\pm Z, E_e)}{\big( 2\pi \big)^5}
  p_e E_e \big( E^0 - E_e \big)^2 \\ \times \xi 
  \Bigg\{ 1 + a\frac{\boldsymbol{p_e \cdot p_\nu}}{E_e E_\nu} + b\frac{m_e}{E_e} 
  + \frac{\boldsymbol{\langle J \rangle}}{J} \Bigg[ A\frac{\boldsymbol{p_e}}{E_e}
    + B\frac{\boldsymbol{p_\nu}}{E_\nu} + D\frac{\boldsymbol{p_e \times p_\nu}}{E_e E_\nu}\Bigg]
  \Bigg\}.
  \label{eq:jacksonSimple}
\end{multline}
%

From \cite{jackson1957a}, we can write down the full expression (ignoring coulomb corrections)
for $A\xi$ 



\subsection{Weak Interaction in the Standard Model}
Talk about the existence of the weak interaction within the standard model and
the vertices and maybe write down the leptonic and hadronic current.


\subsection{Summary}


\section{Ultracold Neutrons}



