\chapter{Introduction to the theory of neutron $\beta$-decay}
\label{ch:Introduction}

%%%%%%%%%%%%%%%%%%%%%%%%%%%%%%%%%%%%%%%%%%%%%%%%%%%%%%%%%%%%%%%%%%%%%%%%%%%%%%%
%%%%%%%%%%%%%%%%%%%%%%%%%%%%%%%%%%%%%%%%%%%%%%%%%%%%%%%%%%%%%%%%%%%%%%%%%%%%%%%

\section{The Standard Model}
The Standard Model of particle physics encompasses everything we know about
interactions between particles and describes nature using the most fundamental
building blocks yet discovered. This section is used as an introduction
to these building blocks, the quarks and leptons, and to the mediators of the
interactions between them, to preface the upcoming descriptions of the neutron
and $\beta$-decay.

\subsection{The Building Blocks}

\subsection{The Forces}

\subsection{Symmetries}

\section{Properties of the neutron}
\label{sec:neutronProperties}
Let us take a moment to give a very brief introduction to the neutron to motivate the
upcoming sections. This will also serve as a useful tool for those who are
unfamiliar with nuclear and field theories, as the following sections become
quite technical. The majority of this dissertation can be read and mostly understood
without much knowledge of the theoretical description behind neutron $\beta$-decay,
so the properties of the neutron are a natural starting point.

The neutron is a net neutrally charged composite particle. The terms net and
composite hint at the inner structure of the neutron, made up of fundamental
particles called quarks, in this case two down ($d$) quarks and one up ($u$) quark.
The quarks do carry charge, with the up charge $+1/2$ and the down charge
$-2/3$, so that the net charge of the neutron is zero. Four other quarks exist,
the charm, strange, bottom, and top in order of increasing mass. The neutron
is the second lightest three-quark composite particle (baryon) one can imagine
behind only the proton.

The neutron undergoes $\beta$-decay, defined as a transition from the neutron
to a proton, electron, and electron anti-neutrino,
\begin{equation*}
  n\rightarrow p + e^- + \bar{\nu}_{e}.
\end{equation*}
\noindent The three-body decay gives rise to a continuous energy spectrum
for the proton, electron, and anti-neutrino by conservation of energy and momentum
as seen in figure \ref{fig:spectralShapes}. The lifetime of the free neutron is
approximately fifteen minutes. 

%%%%%%%%%%%%%%%%%%%%%%%%%%%%%%%%%%%%%%%%%%%%%%%%%%%%%%%%%%%%%%%%%%%%%%%%%%%%%%%
%%%%%%%%%%%%%%%%%%%%%%%%%%%%%%%%%%%%%%%%%%%%%%%%%%%%%%%%%%%%%%%%%%%%%%%%%%%%%%%

\section{$\beta$-decay of the neutron}

\subsection{The weak interaction}
Talk about the existence of the weak interaction within the standard model and
the vertices and maybe write down the leptonic and hadronic current.

\subsection{Correlation Coefficients}
\subsection{Parity violation}
Maximally violates parity, talk about Wu's experiment. Emphasize that this is
precisely what we are measuring. Connect back to the relationship in the hadronic
current

\section{Ultracold Neutrons}



