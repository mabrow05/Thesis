\chapter{Introduction}
\label{ch:Introduction}

The purpose of this dissertation is to describe in detail
the methods used in extracting the $\beta$-decay asymmetry parameter
$A$
for the UCNA Experiment. This chapter hopes to motivate the inception
of the parameter of interest and its role in the theory of $\beta$-decay.
Also introduced are characteristics of the neutron itself with an
emphasis on ultracold neutrons (UCN), the namesake of the UCNA experiment.
The chapter will conclude with a brief summary of the
theory and the motivation for carrying out precision measurements of the
$\beta$-decay asymmetry parameter in free neutron decay.

%%%%%%%%%%%%%%%%%%%%%%%%%%%%%%%%%%%%%%%%%%%%%%%%%%%%%%%%%%%%%%%%%%%%%%%%%%%%%%%
%%%%%%%%%%%%%%%%%%%%%%%%%%%%%%%%%%%%%%%%%%%%%%%%%%%%%%%%%%%%%%%%%%%%%%%%%%%%%%%

\iffalse
\section{The Standard Model}
The Standard Model of particle physics encompasses everything we know about
interactions between particles and describes nature using the most fundamental
building blocks yet discovered. This section is used as an introduction
to these building blocks, the quarks and leptons, and to the mediators of the
interactions between them, to preface the upcoming descriptions of the neutron
and $\beta$-decay. 

\subsection{The Building Blocks}

\subsection{The Forces}

\subsection{Symmetries}

\fi

%%%%%%%%%%%%%%%%%%%%%%%%%%%%%%%%%%%%%%%%%%%%%%%%%%%%%%%%%%%%%%%%%%%%%%%%%%%%%%%
%%%%%%%%%%%%%%%%%%%%%%%%%%%%%%%%%%%%%%%%%%%%%%%%%%%%%%%%%%%%%%%%%%%%%%%%%%%%%%%

\section{Properties of the Neutron}
\label{sec:neutronProperties}
Let us take a moment to give a very brief introduction to the neutron to motivate the
upcoming sections. The majority of this dissertation can be read and mostly understood
without much knowledge of the theoretical description behind neutron $\beta$-decay,
so the properties of the neutron are a natural starting point.

The neutron is a neutrally charged composite particle. The terms net and
composite hint at the inner structure of the neutron, made up of fundamental
particles called quarks, in this case two down ($d$) quarks and one up ($u$) quark.
The mass of the $u$ quark is $\sim2.2~\mathrm{MeV/c}^2$ and the mass of the
$d$ quark is $\sim4.7~\mathrm{MeV/c}^2$.
The quarks carry charge, with the $u$ charge $+2/3e$ and the $d$ charge
$-1/3e$ (where $e$ indicates the magnitude of the electron charge),
so that the net charge of the neutron is zero. This can be compared
to the proton, another composite particle made up of three quarks (two $u$ quarks and
one $d$ quark), whose net charge is $+e$. Protons and neutrons are described
by the common term nucleons, as they are the particles within the nucleus of all
atoms. Four other quarks exist, 
the charm ($+2/3e$, $\sim1.28~\mathrm{GeV/c}^2$), strange ($-1/3e$, $\sim0.096~\mathrm{GeV/c}^2$),
top ($+2/3e$, $\sim173.1~\mathrm{GeV/c}^2$), and  bottom ($-1/3e$, $\sim4.18~\mathrm{GeV/c}^2$). Notice
the units on the mass of these heavier quarks is an order of magnitude larger then the $u$ quark and $d$ quark.
Particles that are composed of three quarks are called baryons, and the neutron
is the second lightest baryon
behind only the proton.

\begin{figure}
  \centering
  \includegraphics[page=1,scale=0.4]{1-Introduction/betaSpectrum.pdf}
    \caption{The electron kinetic energy spectrum in free neutron $\beta$-decay. The horizontal
      axis has units of energy in MeV in this case, but the shape of the spectrum
      is of general interest. Note that the decay probability does go to zero at zero
      MeV.}
  \label{fig:betaSpectrum}
\end{figure}

The free neutron undergoes $\beta$-decay, defined as a transition from the neutron
to a proton, electron, and electron anti-neutrino,
\begin{equation*}
  n\rightarrow p + e^- + \bar{\nu}_{e},
\end{equation*}
\noindent as is also seen pictorially in Figure \ref{fig:pointInteraction} panel (a).
The three-body decay gives rise to a continuous energy spectrum
for the proton, electron, and anti-neutrino by conservation of energy and momentum.
The electron kinetic energy ($T_e$) spectrum, the measurement of which is a primary focus of this
dissertation, is seen in Figure \ref{fig:betaSpectrum}. The endpoint kinetic energy of the electron
is $\sim 782$~keV, as shown by the maximum energy in the spectrum. The lifetime of the free neutron is
approximately fifteen minutes.


\section{$\beta$-Decay and Weak Interactions: A Brief History}

Prior to 1930, $\beta$-decay of nuclei was a polarizing topic of debate. Originally
only the decay electron was detected, and given such an observed two-body decay (the recoil
nucleus and the emitted electron being the two bodies) one would expect a discrete
electron energy that is completely defined in terms of the masses of the initial and
final particles (as calculated by conidering conservation of four-momentum).
Instead, a continuous energy spectrum was observed for the electron, which
initially led some to believe that energy conservation was moot. Others, like
Wolfgang Pauli, were not ready
to abandon energy conservation. He postulated that a neutral particle
could also be emitted in the decay. This third particle would share the energy available
in the reaction, explain the continuous energy distribution of the electron, and go
on undetected as it would not interact electromagnetically.
Pauli initially called the missing particle the neutron because of its neutral charge, but this name
was soon after given to the neutral particle discovered within the nucleus by Chadwick. Chadwick's
neutron was far too heavy
to be the missing particle in $\beta$-decay, as its mass was near that of the proton.
Pauli however did not abandon the idea of a light neutral particle, and Fermi took a keen
interest in the problem himself \cite{pauli1994writings}. The theoretically conceived particle was then termed
the neutrino by Fermi, and it would be a crux of Fermi's theory of $\beta$-decay.
A quarter century later, the existence of the neutrino would be confirmed
experimentally \cite{cowan1956detection}.

\subsection{Fermi's Theory of $\beta$-Decay}
After Pauli postulated the existence of the neutrino to explain the continuous energy
distribution of the $\beta$-decay electron, Fermi attempted to theoretically describe
the process in a similar manner to the theory of emission of
gamma radiation from an excited nucleus \cite{fermi1934,wilson1968fermi}.
His theory relied on two postulates: the existence of the neutrino and that the nucleus consisted
of heavy particles only, the neutron and the proton, both of which would turn out to be true.

\begin{table}[h]
  \caption{Transformation behavior of all possible bilinear covariants. The third column shows the
  transformation under parity.} 
  \centering
  \begin{tabular}{l c}
    \hline \hline \\ [-1.75ex]
    $\bar{\psi}\psi$ & scalar \\ [0.50ex]
    $\bar{\psi}\gamma^5\psi$ & pseudoscalar \\ [0.50ex]
    $\bar{\psi}\gamma^{\mu}\psi$ & vector\\ [0.50ex]
    $\bar{\psi}\gamma^{\mu}\gamma^5\psi$ & axial vector \\ [0.50ex]
    $\bar{\psi}\sigma^{\mu\nu}\psi$ & antisymmetric tensor \\ [0.50ex]   
    \hline
  \end{tabular}
  \label{tab:bilinearCov}
\end{table}

\iffalse
\begin{table}[h]
  \caption{Transformation behavior of all possible bilinear covariants. The third column shows the
  transformation under parity.} 
  \centering
  \begin{tabular}{l c}
    \hline \\-1.75
    Current & Name & Parity \\
    \hline \hline \\ [-1.75ex]
    $\bar{\psi}\psi$ & scalar  & $1$\\ [0.50ex]
    $\bar{\psi}\gamma^5\psi$ & pseudoscalar & $-1$ \\ [0.50ex]
    $\bar{\psi}\gamma^{\mu}\psi$ & vector & $-1$ \\ [0.50ex]
    $\bar{\psi}\gamma^{\mu}\gamma^5\psi$ & axial vector $1$ \\ [0.50ex]
    $\bar{\psi}\sigma^{\mu\nu}\psi$ & antisymmetric tensor \\ [0.50ex]   
    \hline
  \end{tabular}
  \label{tab:bilinearCov}
\end{table}
\fi

Fermi's interaction Hamiltonian took the form (written in a different manner from Fermi's original
paper on the subject for the sake of clarity) 
%
\begin{equation}
  H = C_V\big( \bar{\psi}_p \gamma_\mu \psi_n \big) \big( \bar{\psi}_e \gamma^\mu \psi_\nu \big), 
\end{equation}
%
where $\bar{\psi} \gamma_\mu \psi$ is a vector current (see Table \ref{tab:bilinearCov}).
The $\psi$s in the above equation and the rest of the this thesis represent solutions to
the Dirac equation for spin-1/2 fermions, and they take into account the mass and momentum
of the particle involved in the interaction, as well as whether it is a particle or antiparticle
that participates.
The assumption
that the current-current interaction would be of the vector-vector variety was a natural
choice as this is the case for electromagnetism.

\begin{figure}
  \centering
  \centering
  %\begin{tabular} {cc}
  \subfloat[Contact Interaction at the particle level]{\includegraphics[page=1,scale=0.4]{1-Introduction/pointInteraction.pdf}}\\
  \subfloat[Quark level decay]{\includegraphics[page=1,scale=0.4]{1-Introduction/quarkInteraction.pdf}}
  %\end{tabular}
  \caption{The contact interaction of Fermi's theory of $\beta$-decay is shown in a.). The theory is
    capable of describing other processes which move one of the outgoing particle lines
    to the left side, like electron capture by the proton producing a neutron and an
    electron neutrino. b.) shows the decay at the quark level, where the initial state is the
    quark makeup of the neutron, and the subsequent decay of one of the down quarks into an up
    quark creates the W boson which decays into the electron and electron anti-neutrino.}
  \label{fig:pointInteraction}
\end{figure}

Fermi's theory of $\beta$-decay treats the decay as a four-point contact interaction
(see Figure \ref{fig:pointInteraction} panel (a)), where
the currents are evaluated at the same point in space and time \cite{renton1990}. For those
familiar with quantum field theory within the Standard Model, this differs from the typical
interaction Hamiltonian as there is no propagator, or force carrier, to mediate the
interaction between the two currents. At the most fundamental level, the decay proceeds as is shown
in Figure \ref{fig:pointInteraction} panel (b), where a $d$ quark decays into a $u$ quark
via the emission of a massive $W^-$ boson. The mass of the $W$ ($\sim 80$~GeV) is much larger than
the energy available in the decay, and thus the $W$ almost instantaneously decays into
the $e^-$ and $\bar{\nu}_e$. The fact that the subsequent $W$ decay is ``almost'' instantaneous
makes Fermi's contact interaction a good approximation. More formally, the small momentum transfer in nuclear
$\beta$-decay compared to the mass of the $W$ boson makes the propagator approach a constant form
(in the limit $q^2\rightarrow 0$), thus absorbing
the propagator into the coupling constant.

The theory
worked well at predicting the energy spectrum of the electron from which
Fermi deduced that the neutrino must be nearly massless, but suffered one flaw. The vector
nature of the theory did not permit the observed allowed nuclear $\beta$-decay transitions
which can transform the spin of the decaying nucleus. This was pointed out by Gamow and Teller in
1936 \cite{gamow1936}, where they show that a current-current interaction that transforms like
a pseudovector properly assigns the spins of the products in thorium decays. As such,
one can generalize Fermi's theory by including all possible bilinear covariant terms that
satisfy Lorentz invariance:
%
\begin{multline}
  H = C_S\big( \bar{\psi}_p \psi_n \big) \big( \bar{\psi}_e  \psi_\nu \big) 
  +C_V\big( \bar{\psi}_p \gamma_\mu \psi_n \big) \big( \bar{\psi}_e \gamma^\mu \psi_\nu \big) \\
  +C_T\big( \bar{\psi}_p \sigma^{\mu\nu} \psi_n \big) \big( \bar{\psi}_e \sigma^{\mu\nu} \psi_\nu \big) 
  +C_A\big( \bar{\psi}_p \gamma_\mu \gamma^5 \psi_n \big) \big( \bar{\psi}_e \gamma^\mu \gamma^5 \psi_\nu \big) 
  +C_P\big( \bar{\psi}_p  \gamma^5 \psi_n \big) \big( \bar{\psi}_e  \gamma^5 \psi_\nu \big),
  \label{eq:FermiFull}
\end{multline}
%
where the coefficients quantify the coupling to each respective current.

\subsubsection{Fermi and Gamow-Teller Transitions}

This shortcoming of Fermi's original vector-vector interaction
lends itself to the definition of
two types of allowed $\beta$-decay transitions. The Fermi transition proceeds through the
scalar and vector currents with $\Delta J=0$ and no parity change. Gamow-Teller transitions
correspond to the axial vector and tensor currents and have $\Delta J=0,\pm1$ with no parity change,
but excluding the $0^+\rightarrow 0^+$ nuclear transitions. The pseudoscalar term does not
contribute to the amplitude in the low energy limit \cite{renton1990}.
 

\subsection{Parity Violation in Weak Decays}

\subsubsection{Lee and Yang}

Prior to the 1950's, all discrete symmetries (parity, charge conjugation, and time reversal)
were thought to be conserved separately for all interactions in nature. A dilemma arose when
two particles, called the $\tau$ and $\theta$ mesons at the time, possessed the same mass
and charge as the $K^+$ meson, but decayed to different final states of parity. The $\tau$ decayed into three pions
and the $\theta$ into two pions. The initial classification of the two otherwise identical particles into
separate particles was logical given that parity conservation was then sacrosanct,
but in 1956 Lee and Yang proposed a very different solution to the problem. They realized that
there was no evidence that parity must be conserved in the weak interaction.
In the full expression of Fermi's theory (Equation \ref{eq:FermiFull}),
while including all possible combinations of bilinear covariants, there was no mixing between the
individual currents and each individual current-current term transforms like a scalar. For example,
two individual axial vector currents would each transform as
$P(\bar{\psi}_p \gamma_\mu \gamma^5 \psi_n) \rightarrow -\bar{\psi}_p \gamma_\mu \gamma^5 \psi_n$ under parity, but the
muliplication of two such terms transforms like a scalar. Thus, none of the terms in \ref{eq:FermiFull})
are capable of violating parity, as the Hamiltonian would remain invariant under a parity transformation. 

In their 1956 paper \cite{leeyang1956}, Lee and Yang modified the weak itneraction Hamiltonian
by including the possibility of a pseudoscalar current-current interaction, namely
%
\begin{multline}
  H = \bar{\psi}_p \psi_n \big( C_S\bar{\psi}_e  \psi_\nu + C'_S\bar{\psi}_e \gamma^5 \psi_\nu \big) \\
  + \bar{\psi}_p \gamma_\mu \psi_n  \big( C_V\bar{\psi}_e \gamma^\mu \psi_\nu + C'_V\bar{\psi}_e \gamma^\mu \gamma^5 \psi_\nu \big) 
  + \bar{\psi}_p \sigma^{\mu\nu} \psi_n \big( C_T\bar{\psi}_e \sigma^{\mu\nu} \psi_\nu + C'_T\bar{\psi}_e \sigma^{\mu\nu} \gamma^5 \psi_\nu \big)\\
  + \bar{\psi}_p \gamma_\mu \gamma^5 \psi_n \big( C_A\bar{\psi}_e \gamma^\mu \gamma^5 \psi_\nu + C'_A\bar{\psi}_e \gamma^\mu \psi_\nu \big) 
  + \bar{\psi}_p  \gamma^5 \psi_n \big( C_P\bar{\psi}_e  \gamma^5 \psi_\nu + C'_P\bar{\psi}_e \psi_\nu \big).
  \label{eq:leeyang}
\end{multline}
%
If any of the $C'_i$ coefficents are nonzero, parity would not be conserved due to
the scalar-pseudoscalar form of the current-current interactions that are labeled by
the primed coefficients.

Along with the inclusion of potential parity violating terms in the Hamiltonian, Lee and Yang
presented several potential tests of parity violation in the weak sector. One such proposition
was the measurement of the correlation between the spin of a polarized nucleus and
the momentum of the $\beta$-decay electron.

\subsubsection{Discovery of Parity Violation}

Following the publication of Lee and Yang's newly modified theory of $\beta$-decay, C. S. Wu
and collaborators designed an experiment to test the potential violation
of parity in the $\beta$-decay of $^{60}\mathrm{Co}$
($^{60}\mathrm{Co} \rightarrow {^{60}\mathrm{Ni}} + e^- + \bar{\nu}_e$) \cite{wu1957}. The premise of the experiment
was simple: place an electron detector along the $+z$ axis, polarize the $^{60}\mathrm{Co}$ nuclei
using a magnetic field in the $\pm z$ direction, and measure the electron rate in each polarization
configuration. If parity is violated, a clear asymmetry would be present between the two polarizations,
which is precisely what was discovered and can be seen in Figure \ref{fig:wuData}.

\begin{figure}[h]
  \centering
  \includegraphics[scale=0.50]{1-Introduction/wuData.png}  
  \caption{Data from C. S. Wu's experiment \cite{wu1957} measuring the correlation between the emitted
    direction of the electron from the decay of polarized $^{60}\mathrm{Co}$. The two
    curves represent the counting rates when the nuclei were oriented in opposite
    directions with respect to the electron detector. The existence of a splitting
    indicates a violation of parity as the electrons are preferentially emitted in the
    direction opposite the spin. The asymmetry disappearance coincices with the spin relaxation time
    of the $^{60}\mathrm{Co}$ nuclei.}
  \label{fig:wuData}
\end{figure}

The measurement showed that the electrons were preferentially
emitted in the direction opposite the spin of the nuclei. It is straightforward to illustrate
how such a preference in emission direction indicates parity violation. Start by considering that
a correlation between the spin of the nucleus and the momentum of the electron
must take the form $\vec{\sigma} \cdot \vec{p}$.
The spin is a type of angular momentum and transforms as an axial vector under spatial inversion
($P\vec{\sigma}=\vec{\sigma}$), while the momentum is a polar vector and
simply changes sign under parity ($P\vec{p}=-\vec{p}$). Thus the $\vec{\sigma} \cdot \vec{p}$
combination is itself odd under parity ($P\vec{\sigma} \cdot \vec{p} = \vec{\sigma} \cdot (-\vec{p}) = -\vec{\sigma} \cdot \vec{p}$).
The presence of
a $\vec{\sigma} \cdot \vec{p}$ term in the decay rate, the existence of which the Wu result confirms,
%as will be shown in Section \ref{ssec:correlations})
makes the
decay rate noninvariant under a parity transformation.
The parity inverted scenario would show a preference for the electrons to be emitted aligned
with the spin, which is not the observed case. Thus parity is not conserved in the
weak interaction.
From the theory presented thus far,
the interaction involved was either axial vector or tensor, because decay of
$^{60}\mathrm{Co} (J^p=5^+) \rightarrow {^{60}\mathrm{Ni}}(J^p=4^+)$ proceeds stricly through
the Gamow-Teller transition. Whether the tensor or axial vector current (or both) was responsible
was yet to be determined.

It should be noted that, following the news of Wu's result, Garwin, Lederman, and
Weinrich \cite{garwin1957} from Columbia confirmed
that parity is violated using the subsequent decays of $\pi^+ \rightarrow \mu^++\nu_\mu$
followed by $\mu^+ \rightarrow e^+ + \bar{\nu}_e + \nu_\mu$. This was another process
recommended by Lee and Yang. The premise is that the chiral odd neutrino (the spin of the neutrino
is anti-aligned with its momentum, more on this in Section \ref{ssec:VminusA}) in the first decay
forces the muon to be polarized in the direction of its momentum to conserve angular
momentum. The polarized muon then decays and can thus be analyzed much like one would analyze the
$\beta$-decay of a polarized nucleus, looking for correlations between the muon and
electron polarizations. The results were again conclusive that parity is not conserved in the weak
interaction.

\subsection{Correlation Coefficients} \label{ssec:correlations}
Taking the interaction Hamiltonian from Lee and Yang with all generalized terms included,
Jackson, Treiman, and Wyld \cite{jackson1957a,jackson1957b} first derived an expression
for the differential decay rate for polarized nuclei as a function of the emitted electron momentum
and spin, the neutrino momentum, and the nuclear spin of the decaying nucleus.
Ebel and Feldman \cite{ebel1957} added terms to the expression of Jackson, Treiman, and Wyld,
and, under the assumption that the spin of the mother nucleus and the spin of the outgoing electron
are observable, this gives
%
\begin{multline}
  \frac{d\Gamma}{dE_e dE_\nu d\Omega_e d\Omega_\nu} = \frac{1}{2} \frac{F(\pm Z, E_e)}{\big( 2\pi \big)^5}
  p_e E_e \big( E_0 - E_e \big)^2 \\ \times \xi 
  \Bigg\{ 1 + a\frac{\vec{p}_e \cdot \vec{p}_\nu}{E_e E_\nu} + b\frac{m_e}{E_e} 
  + \frac{\langle\vec{ J }\rangle}{J} \cdot \Bigg[ A\frac{\vec{p}_e}{E_e}
    + B\frac{\vec{p}_\nu}{E_\nu} + D\frac{\vec{p}_e \times \vec{p}_\nu}{E_e E_\nu}\Bigg] \\
  + \Bigg[ \frac{J(J+1)-3\langle (\vec{J} \cdot \hat{\jmath})^2 \rangle}{J(2J-1)} \Bigg]
  \Bigg( c\Bigg[ \frac{\vec{p}_e \times \vec{p}_\nu}{3E_eE_\nu} -
    \frac{(\vec{p}_e \cdot \hat{\jmath})(\vec{p}_\nu \cdot \hat{\jmath}) }{E_eE_\nu} \Bigg]
  + I \Bigg[ \frac{1}{3}\frac{\vec{\sigma} \cdot \vec{p}_\nu}{E_\nu}
    - \frac{(\vec{\sigma} \cdot \hat{\jmath})(\vec{p}_\nu \cdot \hat{\jmath})}{E_\nu} \Bigg] \\
  + K'\frac{\vec{\sigma} \cdot \vec{p}_e}{E_e+m_e} \Bigg[ \frac{1}{3}\frac{\vec{p}_e \cdot \vec{p}_\nu}{E_e E_\nu}
    - \frac{(\vec{p}_e \cdot \hat{\jmath})(\vec{p}_\nu \cdot \hat{\jmath})}{E_e E_\nu} \Bigg] 
  + M \Bigg[ \frac{1}{3}\frac{\vec{\sigma} \cdot \vec{p}_e \times \vec{p}_\nu}{E_e E_\nu}
    - \frac{(\vec{\sigma} \cdot \hat{\jmath})(\hat{\jmath} \cdot \vec{p}_e \times \vec{p}_\nu )}{E_e E_\nu} \Bigg] \Bigg) \\
  + \vec{\sigma} \cdot \Bigg[ N\frac{\langle \vec{J} \rangle}{J}
    + Q\frac{\vec{p}_e}{E_e+m_e}\Bigg(\frac{\langle \vec{J} \rangle}{J} \cdot \frac{\vec{p}_e}{E_e}\Bigg)
    + R\frac{\langle \vec{J} \rangle}{J} \times \frac{\vec{p}_e}{E_e}
    + S\frac{\langle \vec{J} \rangle}{J} \frac{\vec{p}_e\cdot \vec{p}_\nu}{E_e E_\nu} \\
    + T\frac{\vec{p}_e}{E_e}\frac{\langle \vec{J} \rangle}{J} \cdot \frac{\vec{p}_\nu}{E_\nu}
    + U\frac{\vec{p}_\nu}{E_\nu}\frac{\langle \vec{J} \rangle}{J} \cdot \frac{\vec{p}_e}{E_e}
    + W\frac{\vec{p}_e}{E_e+m_e}\frac{\langle \vec{J} \rangle}{J} \cdot \frac{\vec{p}_e \times \vec{p}_\nu}{E_e E_\nu}
    \Bigg]
  + V\frac{\langle \vec{J} \rangle}{J} \cdot \frac{\vec{\sigma} \times \vec{p}_\nu}{E_\nu}
  \Bigg\}
  \label{eq:jackson}
\end{multline}
%
where $F(\pm Z, E_e)$ is the Fermi function (a shape correction to the spectrum
from Coulomb interactions), $E$ is the energy of a given particle, $E_0$ is the endpoint
energy of the electron, $\vec{p}$ is the particle momentum, $\vec{J}$ is the spin of the
decaying nucleus (or nucleon), $\hat{\jmath}$ is a unit vector in the direction of $\vec{J}$,
and $\vec{\sigma}$ is the spin of the electron. All of the correlation coefficients
are functions of the coupling constants in the weak Hamiltonian (see Equation \ref{eq:leeyang}),
as is $\xi$. The correlation coefficients and $\xi$ are also functions of Fermi and Gamow-Teller
transition amplitudes \footnote{For the complete definitions see \cite{jackson1957a,jackson1957b,ebel1957}}.

Foreshadowing upcoming sections, we write down the definitions of $\xi$ and $A\xi$ (ignoring Coulomb
corrections):
%
\begin{equation}
  \xi = |M_F|^2\big(|C_S|^2+|C_V|^2+|C'_S|^2+|C'_V|^2\big)+|M_{GT}|^2\big(|C_T|^2+|C_A|^2+|C'_T|^2+|C'_A|^2\big)
  \label{eq:xi}
\end{equation}
%
and
\begin{multline}
  A\xi = 2\mathrm{Re}\bigg[\pm |M_{GT}|^2 \lambda_{J'J}\big(C_TC'^*_T-C_AC'^*_A \big) \\
    + \delta_{J'J}M_FM_{GT}\bigg( \frac{J}{J+1} \bigg)^{\frac{1}{2}}\big(C_SC'^*_T+C'_SC^*_T -C_VC'^*_A-C'_VC^*_A \big) \bigg]
  \label{eq:XiA}
\end{multline}
where
\begin{equation}
\lambda_{J'J} =
\begin{dcases}
  1, & J\rightarrow J'=J-1 \\
  \frac{1}{J+1}, &  J\rightarrow J'=J \\
  \frac{-J}{J+1}, &  J\rightarrow J'=J+1 \\
\end{dcases}
\end{equation}
and the $\pm$ sign indicates $\beta^-(\beta^+)$ decay respectively (the plus sign goes with
$\beta^-$, the $-$ with $\beta^+$ to be clear).
The rest of the coefficients have varying combinations of the the complex coupling
constants $C$, $C'$, and the Fermi and
Gamow-Teller amplitudes.

Jackson, Treiman, and Wyld concisely state the impact these terms have on tests of C, P, and
T invariance:
\begin{displayquote}
Invariance with respect to space inversion implies that the coupling constants $C'$
vanish (or alternatively the $C$ vanish). Invariance with respect to charge
conjugation implies that the constants $C$ are real and the constants $C'$
pure imaginary, up to an overall phase. Invariance under time reversal
would imply that all coupling constants $C$, $C'$ are real, again up to an
overall phase.
\end{displayquote}

Measurements of the correlation parameters in nuclear systems shed light on the
relationships between the different coupling constants, but it would take observations
of the behavior of the observed neutrinos and electrons to determine the true structure
of the interaction Hamiltonian.

\subsection{$V-A$ Structure} \label{ssec:VminusA}

By 1956, the theory of the weak interaction had developed from a purely vector current-current
process into
a combination of all possible Lorentz invariant current-current interactions and finally to the potential mixture
of current-current terms that transform as scalars and pseudoscalars in order to accommodate
parity violation. At this point, the theory needed to rely on experiment to determine
which couplings were nonzero.

In early 1957, upon Wu reporting preliminary results regarding the large asymmetry seen in the $\beta$-decay
of $^{60}\mathrm{Co}$ to Lee and Yang, the duo developed a two component theory
of the neutrino where a massless neutrino exists in a predefined polarization, or that
its spin is always oriented in the same direction with respect to its momentum \cite{lee1957}
\footnote{Technically this only applies to neutrinos that interact via
  the weak interaction, but seeing as neutrinos are only
  observed to interact weakly, this can be generalized to all observable neutrinos. Also, Lee and
  Yang's theory had the wrong sign for the $\vec{\sigma} \cdot \vec{p}$ of the neutrino.}.
This was the first theory to indicate that the weak interaction coupled only to particles
of a certain handedness, where right (left) handedness refers to eigenstates with eigenvalues of the
chirality projection operator ($P_{\pm} = (1\pm \gamma^5)/2$) equal to $+1\mathrm{ }(-1)$. This operator
should look familiar, as it is the term Lee and Yang inserted into Fermi's theory of $\beta$-decay
when $C_i=C'_i$. 

In 1958, Feynman and Gell-Mann \cite{feynman1958} (and separately Sudarshan and Marshak
\cite{sudarshan1958}) would report the currently accepted V-A (vector minus axial vector)
theory of the weak interaction. The name derives from the form of the two currents in the
current-current Hamiltonian.
The new theory involved the coupling of the weak current to strictly left-handed particles and right-handed
antiparticles. More simply, a massless neutrino will always have helicity
$\vec{\sigma} \cdot \vec{p}/p = -1$ while an anti-neutrino will have $\vec{\sigma} \cdot \vec{p}/p = +1$,
where the fact that helicity and chirality are one in the same for massless particles is utilized.
The form of the V-A interaction Hamiltonian
%
\begin{equation}
  H = G_F \Big(\bar{\psi}_n \gamma_\mu(1-\gamma^5)\psi_p \Big)\Big(\bar{\psi}_{\bar{\nu}} \gamma^\mu(1-\gamma^5)\psi_e \Big),
  \label{eq:VminusA}
\end{equation}
%
where $G_F$ is the Fermi coupling constant of the weak interaction,
assumes that the V-A structure is universal within the weak interaction\footnote{This is
  only strictly true for the charged weak interaction (so a change in charge occurs in each current),
  but at this time only the charged weak interaction
  was known.}.

The theories of Feynman and Gell-Mann (and Sudarshan and Marshak)
explained the observations at that time
regarding the weak interaction. Then, also in 1958, confirmation of the V-A theory arrived when
Goldhaber, Grodzins, and Sunyar indirectly measured
the helicity of the neutrino in electron capture on $^{152}\mathrm{Eu}$ and determined the
emitted neutrino was indeed left-handed
\cite{goldhaber1958,greiner1996}. It should be noted that these two theories differed from that of
Lee and Yang in the definition of the chirality of the particles ($-$) and anti-particles ($+$), where
Lee and Yang had the opposite definition.

Now one may be tempted to ask why only the vector and axial vector components of the original
parity violating Hamiltonian remain, and this is easily answered by imagining that the
left-handed coupling was discovered prior to modification of the Hamiltonian in equation
\ref{eq:FermiFull} by Lee and Yang to account for parity violation. We can define the left-handed
particle states as 
%
\begin{equation}
  \psi_L = \frac{1}{2}\big(1-\gamma^5\big)\psi.
\end{equation}
%
Then using this in the current yields
%
\begin{equation}
  \bar{\psi}_{1,L} O_i \psi_{2,L} = \frac{1}{4}\overline{\big(1-\gamma^5\big)\psi_{1}} O_i \big(1-\gamma^5\big)\psi_{2}
\end{equation}
%
where $O_i$ refers to all of the operators which produce S, V, T, A, and P bilinear covariants
from Table \ref{tab:bilinearCov} and subscript 1,2 refers to two different particle states.
The $O_i$ are those that are present in \ref{eq:FermiFull}.

\begin{table}[h]
  \caption{The chirality projected operators.} 
  \centering
  \begin{tabular}{c c c }
    \hline \hline \\ [-1.75ex]
    Type & $O_i$ & $O'_i$ \\ [0.50ex]
    \hline \\ 
    S & $1$ & 0 \\ [0.50ex]
    P & $\gamma^5$ & 0 \\ [0.50ex]
    V & $\gamma^{\mu}$ & $\frac{1}{2}\gamma^\mu(1-\gamma^5)$ \\ [0.50ex]
    A & $\gamma^{\mu}\gamma^5$ & $-\frac{1}{2}\gamma^\mu(1-\gamma^5)$ \\ [0.50ex]
    T & $\sigma^{\mu\nu}$ & 0 \\ [0.50ex]   
    \hline
  \end{tabular}
  \label{tab:operatorPrime}
\end{table}

Now it can be shown that
%
\begin{align*}
  \bar{\psi}_{1,L} O_i \psi_{2,L} &= \frac{1}{4}\overline{\big(1-\gamma^5\big)\psi}_{1} O_i \big(1-\gamma^5\big)\psi_{2} \\
  &= \frac{1}{4}\Big(\big(1-\gamma^5\big)\psi_{1}\Big)^\dagger \gamma^0 O_i \big(1-\gamma^5\big)\psi_{2} \\
  &= \frac{1}{4}\psi^\dagger_{1}\big(1-\gamma^5\big)^\dagger \gamma^0 O_i \big(1-\gamma^5\big)\psi_{2} \\
  &= \frac{1}{4}\psi^\dagger_{1}\big(1-\gamma^5\big) \gamma^0 O_i \big(1-\gamma^5\big)\psi_{2} \\
  &= \frac{1}{4}\psi^\dagger_{1}\gamma^0\frac{1+\gamma^5}{2} O_i \big(1-\gamma^5\big)\psi_{2} \\
  &= \frac{1}{4}\bar{\psi_{1}}\frac{1+\gamma^5}{2} O_i \big(1-\gamma^5\big)\psi_{2} 
\end{align*}
%
where the relations ${\gamma^5}^\dagger = \gamma^5$ and $\{\gamma^5,\gamma^i\}=0$ were used.

Then, if we define
\begin{equation*}
  O'_i = \frac{1}{4}(1+\gamma^5) O_i (1-\gamma^5),
\end{equation*}
we can calculate how all of the S, V, T, A, and P operators transform. This is shown in table
\ref{tab:operatorPrime}, where we see that only the vector and axial vector terms remain, and they
appear with the same $\gamma^\mu(1-\gamma^5)$ factor up to a sign! Thus, these are the only
currents that can contribute in the weak interaction Hamiltonian \cite{greiner1996}.



%Several experimental efforts (cite experiments) at the time indicated that the emitted particles
%(antiparticles) from weak decay processes were in helicity states of $-1$ ($+1$), where helicity
%is defined as $\Lambda = \boldsymbol{\sigma \cdot p}/p$ and the spin and momentum are both of the emitted
%particle. This means that in $\beta^-$-decay, the emitted electron has negative helicity and the
%antineutrino has positive helicity. This may lead one to believe that the weak force couples only
%to negative (positive) helicity eigenstates for particles (antiparticles), and as such the
%Hamiltonian would include the helicity projection operator $P_{\pm} = (1\pm \Lambda)/2$ to select
%only the proper helicity states from the wavefunction. This has one major issue: the helicity
%operator is not Lorentz invariant for massive particles. Thus a Lorentz boost to a
%reference frame moving in the direction of particle motion but with speed greater than that of
%the particle 

\subsection{Renormalizing the Hadronic Current}

The forms of both the hadronic current ($\bar{\psi}_p \gamma_\mu(1-\gamma^5)\psi_n$) and
leptonic current ($\bar{\psi}_\nu \gamma_e(1-\gamma^5)\psi_{\bar{\nu}}$) in \ref{eq:VminusA} indicate
that the vector and axial vector currents possess the same coupling constants, or that
they contribute equally in the weak interaction. For the leptonic current this is the case
as the particles participating are true point particles (and the equal relative strengths are
what is meant by parity is violated maximally in the weak interaction), but what about the
hadronic current? One must remember that the hadronic process that occurs is actually the decay
of a single quark, and that the other two quarks (and the sea of quarks and gluons confined within the hadron)
are spectators to the process.
It is conceivable then that the spectator quarks could modify the couplings.
To account for potential differences in the hadronic current, the
Hamiltonian is modified slightly,
%
\begin{equation} 
  H = G_F \Big(\bar{\psi}_p \gamma_\mu(g_V+g_A\gamma^5)\psi_n \Big)\Big(\bar{\psi}_{e} \gamma^\mu(1-\gamma^5)\psi_{\bar{\nu}} \Big),
  \label{eq:VminusA2}
\end{equation}
%
which lends itself to the definition
%
\begin{equation}
  \lambda \equiv \frac{g_A}{g_V}.
\end{equation}

The presence of the spectator quarks in the hadron
can effectively renormalize the interaction, which can
precipitate values of $g_V\neq 1$ and $g_A\neq -1$,
%creating values for $|g_V|$ and $|g_A|$ different from unity (as one may guess from inspection of Equation \ref{eq:VminusA2}, $g_A$ is negative
%under this convention),
thus making the hadronic couplings different from the
couplings in the leptonic current.

\subsubsection{Conserved Vector Current Hypothesis}

The observed measurements of the Fermi constant ($G_F$) seemed
universal when comparing measurements from
purely leptonic ineractions and purely Fermi transitions (vector transitions).
Thus the vector portion, in analogy with
electromagnetism, is taken to be conserved, as was postulated by Feynman and Gell-Mann in
1958 \cite{feynman1958}. 

The conservation of the vector current can be imagined by considering the disocciation of the neutron
into a proton and a $\pi^{-}$ for a short period of time, until the proton absorbs the
$\pi^{-}$ and becomes a neutron again. If during this time when the proton and $\pi^{-}$ exist separately
the decay process is not allowed, then we would expect that the current would be renormalized.
Instead, the $\pi^{-}$ can weakly decay via $\pi^- \rightarrow \pi^0+e^- + \overline{\nu}$,
a Fermi transition as $J^\pi=0^- \rightarrow J^\pi=0^-$. This loop diagram and others like it contribute to the
conservation of the vector current by projecting
the potentiality of neutron decay onto the disocciated pion \cite{grotz1990}. So, while one may na\"{i}vely
think that the neutron could not decay while the pion exists, the probability of the pion also decaying
via the same interaction and giving the same decay products conserves the vector current.
This conservation of vector current is now called the
Conserved Vector Current (CVC) hypothesis, and the value of $g_V$ is taken to be unity ($g_V = 1$).

The same conservation of coupling to the axial vector current can not be assumed, as was also 
considered by Feynman and Gell-Mann \cite{feynman1958}:
%
\begin{displayquote}
Now with present knowledge it is not so easy to say
whether or not a pseudovector current like
$\bar{\psi}i\gamma^5\gamma^\mu\tau^+\psi$
can be arranged to be not renormalized. The present
experiments in $\beta$ decay indicate that the ratio of the
coupling constant squared for Gamow-Teller (axial vector) and Fermi
(vector)
is about $1.3\pm0.1$. This departure from 1 might be a
renormalization effect.
\end{displayquote}
%
The value of $g_A$ can then only be determined experimentally\footnote{Today it can also be calculated on
  a lattice as will be discussed in a few sections.} and is directly related to measurements
of the correlation coefficients in Section \ref{ssec:correlations}.


\subsection{CKM Mixing Matrix}
The weak interaction as introduced thus far describes, within the hadronic current, the transitions
between quarks with charge $-1/3$ and $+2/3$. This includes observed interactions between the heavier
strange quark (charge $-1/3$) and the up quark (charge $+2/3$). 
A shortcoming of the interaction in Equation \ref{eq:VminusA2} was the inability to describe
the decay rate of strangeness altering processes with respect to strangeness conserving
decays, where the strangeness is measured by the number of strange quarks in the state of interest.
This led Cabibbo \cite{cabibbo1963} to introduce the following correction to the hadronic current:
%
\begin{equation}
  J^\mu_H = \cos\theta_C J^\mu_H(\Delta S=0) + \sin\theta_C J^\mu_H(|\Delta S|=1),
  \label{eq:cabibbo}
\end{equation}
%
where $J^\mu_H = \bar{\psi}_1 \gamma_\mu(g_V-g_A\gamma^5)\psi_2$ is the hadronic current between
hadrons 1 and 2, and $\theta_C$, called the Cabibbo angle, quantifies the relative strength
of the couplings to the strangeness conserving and strangeness altering processes.
Cabibbo found that $\theta_C \approx 13\degree$ described the observed differences.

At the time Cabibbo introduced his modification, only the up, down and strange quarks were known
to exist. Glashow, Iliopoulos, and Maiani proposed the existence of a fourth quark, the charm quark,
with charge $+2/3$ as a complementary particle in the weak doublet with the strange quark, to explain
an observed discrepancy in the decay rate of $K^0 \rightarrow \mu^+ + \mu^-$ \cite{glashow1970}.
Validation of the GIM mechanism arrived with the discovery of the
charm quark in 1974.

Even prior to the discovery of the charm quark, Kobayashi and Maskawa generalized the ideas of
Cabibbo, Glashow, Iliopoulos, and Maiani to include three generations of quarks \cite{kobayashi1973}. The single
Cabibbo angle was replaced with three angles relating couplings between each generation of quarks and a complex
phase. We now know that three generations of quarks exist, and they appear in the weak interaction as part of weak
doublets rather than particle doublets, i.e.
%
\begin{equation}
  \begin{pmatrix} u \\ d' \end{pmatrix} ,
  \begin{pmatrix} c \\ s' \end{pmatrix} ,
  \begin{pmatrix} t \\ b' \end{pmatrix}.
\end{equation}
%
The weak states for the down ($d$), strange ($s$), and bottom ($b$) quarks are related to the particle states
by the CKM matrix: 
%
\begin{equation}
  \begin{pmatrix} d' \\s' \\ b' \end{pmatrix} =
  \begin{pmatrix} V_{ud} & V_{us} & V_{ub} \\
    V_{cd} & V_{cs} & V_{cb} \\
    V_{td} & V_{ts} & V_{tb} \\
  \end{pmatrix}
  \begin{pmatrix} d \\s \\ b \end{pmatrix} ,
\end{equation}
%
\begin{equation}
  \begin{pmatrix} d' \\s' \\ b' \end{pmatrix} =
  \begin{pmatrix} c_{12}c_{13} & s_{12}c_{13} & c_{13}e^{i\delta} \\
    -s_{12}c_{23}-c_{12}s_{23}s_{13}e^{i\delta} & c_{12}c_{23}-s_{12}s_{23}s_{13}e^{i\delta} & s_{23}c_{13} \\
    s_{12}s_{23}-c_{12}c_{23}s_{13}e^{i\delta} &  -c_{12}s_{23}-s_{12}c_{23}s_{13}e^{i\delta} & c_{23}c_{13} \\
  \end{pmatrix}
  \begin{pmatrix} d \\s \\ b \end{pmatrix},
\end{equation}

\noindent where $c_{ij}$ $(s_{ij})$ stand for $\cos\theta_{ij}$ ($\sin\theta_{ij}$), $\theta_{ij}$
are the three angles $\theta_{12}$, $\theta_{13}$, and $\theta_{23}$ that relate the couplings
between each quark generation, and $\delta$ is a
complex phase.

Each element of the CKM matrix, $V_{ij}$, quantifies the coupling of quark $i$
to quark $j$, or in comparison to Cabibbo's modification to the hadronic current
above, the CKM matrix elements take the place of the $\cos\theta_C$ and $\sin\theta_C$
when extended to three generations of quarks. As a matter of fact, if we assume no mixing
with the third generation of quarks, the CKM matrix becomes the matrix developed by
Glashow, Iliopoulos, and Maiani and the only quantity needed to fully determine
the relationships between the weak states and the quark states is the Cabibbo angle $\theta_{12}=\theta_C$
\cite{griffiths2008}.
Thus, direct comparison with Equation \ref{eq:cabibbo} indicates that the proper
$V_{ij}$ should accompany the hadronic current
of the process being considered in the Hamiltonian, i.e. $V_{ud}$ accompanies the hadronic
current in neutron $\beta$-decay.

The existence of non-zero off diagonal terms means that there are non-zero probabilities for
weak interactions between quarks of different generations, given that it is energetically
allowed. The complex phase ($\delta$) present in the CKM
matrix also accounts for CP-violation in weak interactions.
The elements of the CKM matrix must be measured experimentally, and the unitarity of the matrix,
%
\begin{equation}
  |V_{ud}|^2 + |V_{us}|^2 + |V_{ub}|^2 = 1,
\end{equation}
is an important test of the three quark nature of the Standard Model. If
instead the CKM matrix is found to not be unitary, this would indicate the presence of
new beyond Standard Model physics, an example of which could be a fourth generation of quarks.


\section{What Does This Mean for the Neutron?}

\subsection{Matrix Element}
The matrix element for $\beta$-decay is given by
%
\begin{equation}
  \mathcal{M} = \frac{G_FV_{ud}}{\sqrt{2}} J^\mu L_\mu
  %\langle p |J^\mu| n\rangle
  \label{eq:M}
\end{equation}
%
where $L_\mu = \bar{u}_e \gamma_\mu (1-\gamma^5) u_{\bar{\nu}}$ is the leptonic current
and the hadronic current $J^\mu$, written in the style of \cite{gardner2001}, is given by
%
\begin{multline}
  J^\mu = \bar{u}_p \bigg[ f_1(q^2) - i \frac{f_{2}(q^2)}{M}\sigma^{\mu\nu}q_\nu + \frac{f_3(q^2)}{M}q^\mu \\
    + g_1(q^2)\gamma^5 - i \frac{g_{2}(q^2)}{M}\sigma^{\mu\nu}\gamma^5q_\mu 
    +  \frac{g_3(q^2)}{M}\gamma^5q^\mu \bigg] u_n
  \label{eq:hadronicCurrent}
\end{multline}
%\begin{multline}
%  J^\mu = \bar{u}_p \bigg[ g_V(q^2) - i \frac{g_{WM}(q^2)}{2M}\sigma^{\mu\nu}q_\nu + \frac{g_S(q^2)}{2M}q^\mu
%    + g_A(q^2)\gamma^5 \\- i \frac{g_{T}(q^2)}{2M}\sigma^{\mu\nu}\gamma^5q_\mu 
%    +  \frac{g_P(q^2)}{2M}\gamma^5q^\mu \bigg] u_n
%\end{multline}
%
where the terms included are all those that satisfy translational and Lorentz invariance. Notice the
usual $f_1(0)=g_V$ and $g_1(0) = g_A$ terms from the earlier construction of the weak interaction
(Equation \ref{eq:VminusA2}). The
rest of the couplings
account for modifications to the hadronic current from other mechanisms (nuclear recoil and Coulomb for example)
that may transform in ways not described by the typical axial vector and vector currents. Also note that all terms
included in the current are either axial vector or vector, as is ensured by the proper inclusion of the momentum
transfer $q_\nu$. This must be the case since all terms in the current-current multiplication $J^\mu L_\mu$ must be
scalar or pseudoscalar.

The $f_2$ term is often called the weak magnetism contribution \cite{gellmann1958}, as it, in
the words of Gell-Mann,
``bears the same relation to the allowed Fermi coupling that magnetism bears to electricity''.
As $q^2\rightarrow0$ and assuming the CVC hypothesis,
$f_2(0) = (\mu_p - \mu_n)/2$, where $\mu$ denotes the anomalous magnetic moment.
The $g_3$ term is the induced pseudoscalar term and is predicted to contribute
to the energy spectrum at the order $m_e^2/ME_e \approx 10^{-4}$
\cite{holstein1974recoil,plaster2012}.

The remaining $f_3$ and $g_2$ couplings are considered second-class currents \cite{weinberg1958charge}.
They arise from violations of $G$ parity, defined as $G = Ce^{i\pi I_2}$. If $G$ parity is conserved in
weak interactions as it is in strong interactions, then $f_3=g_2=0$. Even if nonzero, the effects
from either are expected to small, as they also contribute to the energy spectrum at order $m_e^2/ME_e \approx 10^{-4}$
\cite{holstein1974recoil,plaster2012}.


\subsection{Neutron $\beta$-Decay Asymmetry Parameter $A$} \label{ssec:neutronAsymmParam}

In the case of $\beta$-decay of polarized free neutrons and assuming the electron spin is undetectable,
Equation \ref{eq:jackson} simplifies drastically,
%
\begin{multline}
  \frac{d\Gamma}{dE_e dE_\nu d\Omega_e d\Omega_\nu} = \frac{1}{2} \frac{F(\pm Z, E_e)}{\big( 2\pi \big)^5}
  p_e E_e \big( E_0 - E_e \big)^2 \\ \times \xi 
  \Bigg\{ 1 + a\frac{\vec{p}_e \cdot \vec{p}_\nu}{E_e E_\nu} + b\frac{m_e}{E_e} 
  + \frac{\langle \vec{J} \rangle}{J} \cdot \Bigg[ A\frac{\vec{p}_e}{E_e}
    + B\frac{\vec{p}_\nu}{E_\nu} + D\frac{\vec{p}_e \times \vec{p}_\nu}{E_e E_\nu}\Bigg]
  \Bigg\}.
  \label{eq:jacksonSimple}
\end{multline}
%
Now in the case of UCNA, only the polarization of the initial neutron population and the momentum
of the decay electron are observable, which further simplifies Equation \ref{eq:jacksonSimple} to
%
\begin{multline}
  \frac{d\Gamma}{dE_e dE_\nu d\Omega_e d\Omega_\nu} = \frac{1}{2} \frac{F(\pm Z, E_e)}{\big( 2\pi \big)^5}
  p_e E_e \big( E_0 - E_e \big)^2 \times \xi 
  \Bigg\{ 1 + b\frac{m_e}{E_e} 
  + A \frac{\langle \vec{J} \rangle}{J} \cdot \frac{\vec{p}_e}{E_e}
  \Bigg\}.
  \label{eq:jacksonSimple2}
\end{multline}
%

The $b$ term is known as the Fierz interference term, and, if nonzero, it presents itself
within the electron energy spectrum as an energy shift. The other term is the asymmetry parameter
$A$ between the neutron spin and the electron momentum, the measurement of which is the topic of
this thesis.

The Fierz interference term can be written as
\begin{equation}
  b\xi = \pm 2\gamma \mathrm{Re}\bigg[|M_{F}|^2 \lambda_{J'J}\big(C_SC^*_V-C'_SC'^*_V \big) 
    + |M_{GT}|^2\big(C_TC^*_A-C'_TC'^*_A \big) \bigg]
  \label{eq:b}
\end{equation}
where $\gamma = (1-\alpha ^2 Z^2)^{1/2}$, $\alpha$ is the fine structure constant, and $\xi$ is given in
Equation \ref{eq:xi}. From Equation \ref{eq:b}, we see that $b=0$ under the V-A structure
of the weak interaction within the Standard Model, as
$C_S = C'_S = C_T = C'_T = 0$. Measurements of $b$ provide constraints on
beyond Standard Model physics as they probe the existence of scalar and tensor currents,
but these measurements are difficult because they
require precise determination of the shape of the electron energy spectrum. A measurement of
$b$ using the 2010 UCNA energy spectrum exists, with a value of
$b_n = 0.067 \pm 0.005_{\mathrm{stat}}{}^{+ 0.090}_{-0.091}{}_{\mathrm{syst}}$ \cite{hickerson2017}, which is consistent with
$b=0$. A result using the energy
spectra presented in this dissertation is currently under analysis by Xuan Sun of the California Institute of
Technology, with intentions of reducing the
systematic uncertainty from the previous result.

The asymmetry parameter $A$ was given in Equation \ref{eq:XiA}, but under the assumptions that the
weak interaction is V-A and couples to left-handed fermions ($C_A=C'_A$ and $C_V=C'_V$), the couplings
are pure real, and using the fact that
$\Delta J = {\frac{1}{2}}^+ - {\frac{1}{2}}^+ = 0$ we have
%
\begin{align}
  A\xi &= 2\bigg[ \frac{2}{3} |M_{GT}|^2 \Big(-C_A^2\Big) 
    + M_FM_{GT}\Big(\frac{1}{3}\Big)^{\frac{1}{2}}\big(-2C_VC_A \big) \bigg] \\
  &=-4 \bigg( \frac{1}{3}C_A^2|M_{GT}|^2 + \frac{1}{\sqrt{3}} C_VC_A M_FM_{GT} \bigg)
  \label{eq:XiAsimple}
\end{align}
%
and from \ref{eq:xi} under the same assumptions,
%
\begin{equation}
  \xi = 2|M_F|^2C_V^2+2|M_{GT}|^2C_A^2.
\end{equation}
%

For the neutron, the Fermi and Gamow-Teller matrix elements are $M_F=1$ and
$M_{GT} = \sqrt{3}$, with the simplicity resulting from there being no
overlapping nucleon wavefunctions as there would be in nuclear $\beta$-decay
\cite{grotz1990}. Using this in the equations above we have
%
\begin{align}
  A & =\frac{-4 \big( C_A^2 +  C_VC_A \big)}{2 \big(C_V^2 + 3C_A^2\big)} \\
  A & = -2 \frac{ \lambda \big(\lambda+1\big)}{ 1 + 3\lambda^2}
\end{align}
%
where
%
\begin{equation}
  \lambda = \frac{C_A}{C_V} = \frac{g_A}{g_V}.
\end{equation}

This expression for $A$ does not account for recoil order and Coulomb modifications to the asymmetry, and so it is
often referred to as $A_0$. Thus we see that upon measuring the $\beta$-decay
asymmetry parameter $A$, we must make modifications to the measured value to extract
$A_0$. From $A_0$, a direct determination of $\lambda = \frac{g_A}{g_V}$, the ratio of the axial-vector to vector
coupling constants in the hadronic weak interaction, follows.
These corrections are detailed below.

\subsubsection{Recoil Order Corrections}  \label{sssec:ROCorr}

Gardner and Zhang \cite{gardner2001} present the asymmetry parameter $A$ upon considering recoil order effects
from a finite mass nucleon. The treatment is completely general in that it includes all possible couplings
(including induced couplings and second-class currents) from the hadronic current in Equation
\ref{eq:hadronicCurrent}. Defining the parameters
%
\begin{equation*}
  \tilde{f_i}=f_i(0)/f_1(0),\qquad  \tilde{g_i}=g_i(0)/g_1(0), \qquad
  R=\frac{E^{\mathrm{max}}_e}{M_n}, \qquad x = \frac{E_e}{E^{\mathrm{max}}_e}, \qquad \epsilon=\Big(\frac{m_e}{M}\Big)^2
\end{equation*}
where again $f_1(0)=g_V$, $g_1(0)=-g_A$, and $\lambda=g_A/g_V$, Gardner and Zhang report:
%
\begin{multline}
  A = A_0 + \frac{1}{(1+3\lambda^2)^2}\Bigg\{\frac{\epsilon}{Rx}\Big[4\lambda^2
      (1-|\lambda|)(1+|\lambda|+2\tilde{f}_2)+4|\lambda|(1-|\lambda|)
      (|\lambda|\tilde{g}_2-\tilde{f}_3)\Big] \\
    + R\bigg[\frac{2}{3}\big[1+\lambda+2(\tilde{f}_2+\tilde{g}_2)\big]
      (3\lambda^2+2|\lambda|-1)\bigg] \\
      +Rx\bigg[\frac{2}{3}(1+|\lambda| + \tilde{f}_2)
        (1-5|\lambda|-9\lambda^2-3|\lambda|^3)
        +\frac{4}{3}\tilde{g}_2(1+|\lambda|+3\lambda^2+3|\lambda|^3)\bigg]\Bigg\} \\
  +O(R^2,\epsilon).
\end{multline}

The correction to $A_0$ is energy dependent, as the $x$ term is simply the electron energy divided by
the endpoint energy. The above expression simplifies to the recoil order corrections
presented elsewhere \cite{bilenkii1960,wilkinson1982} when the second-class currents
are ignored. The correction as applied within this analysis, shown in Section \ref{sssec:sysROCorr},
also ignores the second-class currents.

Gardner and Zhang provide a similar expression for $a$ (the neutrino-electron correlation),
which is also of the form $d_0R+d_1Rx+d_{-1}\epsilon/(Rx)$ where the coefficients $d_i$ are functions
of the three unknowns $\lambda$, $\tilde{f}_2$, and $\tilde{g}_2$.
They point out that, via a determination of $d_0$ and
$d_1$ (via the $x^0$ and $x^1$ energy dependence) from measurements of both $A$ and $a$:
%
\begin{displayquote}
  The system is over-constrained, so that we can infer the existence of physics
  beyond the SM (Standard Model), namely, presence of non-($V-A$) currents
  \cite{jackson1957a}, if the extracted coupling constants differ from SM bounds or if the
  values of the extracted couplings are not consistent with each other.
\end{displayquote}
Part of the ever-continuing motivation for carrying out precision measurements for
neutron decay correlation parameters like $A$ is to place more stringent limits on
beyond SM physics.



\subsubsection{Electromagnetic Corrections} \label{sssec:RadCorr}

The decay electron is immersed in the Coulomb field of the proton, and thus the observed
decay kinematics can be affected by both exchanging photons with the other
constituents or Bremsstrahlung radiation as the electron moves through the field.
Fermi initially accounted for modifications to the electron wavefunction from the
presence of a point nucleus \cite{fermi1934versuch,wilkinson1982}, which appears
as the Fermi correction to the phase space factor as seen in the
angular distribution in Equation \ref{eq:jackson}.

The radiative correction of order $\alpha$ (the fine structure constant) to the
decay rate of unpolarized nuclei was established
by Sirlin in 1967 \cite{sirlin1967}, and it takes the form:
%
\begin{equation}
  G(E,E_0) = 1+\frac{\alpha}{2\pi}g(E,E_0),
\end{equation}
where the functional form of $g(E,E_0)$ is given in \cite{sirlin1967}. This appears
as a multiplicative factor in the decay rate.

Shann then readdressed the radiative corrections of Sirlin, only he extended the
correction to polarized nuclei in the context of the electron asymmetry,
giving the correction factor \cite{shann1971}:
%
\begin{equation}
  A(E) = A_0\bigg(1+\frac{\alpha}{2\pi}\Big(h(E,E_0)-g(E,E_0)\Big)\bigg),
\end{equation}
where
\begin{multline}
  h-g = 4 \bigg( \frac{E_0-E}{3E\beta^2} \bigg)
  \bigg( \frac{\tanh^{-1}\beta}{\beta}-1 \bigg)
  \bigg(1-\beta^2+ \frac{E_0-E}{8E} \bigg) \\
  + \frac{\tanh^{-1}\beta}{\beta}
  \bigg( 2-2\beta^2-\frac{\left(E-E_0\right)^2}{6E^2} \bigg).
\end{multline}
%
The application of this correction within the context of this analysis is given in
Section \ref{sssec:sysRadCorr}.

\subsection{Neutron Lifetime}
The neutron lifetime is another observable of the free neutron, and it is calculated using the matrix
element in Equation \ref{eq:M} and then integrating over the allowable phase space for the decay.
The neutron lifetime is given by
%
\begin{equation}
  \frac{1}{\tau_n} = \frac{G_F^2 m_e^5}{2\pi^3}V_{ud}^2 \big(1+3\lambda^2)f(1+\mathrm{RC})
\end{equation}
%
where $f$ is the phase space factor and $(1+\mathrm{RC})$ accounts for radiative
corrections \cite{czarnecki2004precision,marciano2006}. If we take $G_F$ to be known from muon decay and $\lambda$ from
measurements of correlation paramaters (like $A$), then we see that upon measuring the
neutron lifetime $\tau_n$, there is only one unknown: the CKM matrix element $V_{ud}$.
We can rewrite the lifetime equation as
%
\begin{equation}
  |V_{ud}|^2 = \frac{4908.7(1.9)\mathrm{s}}{\tau_n}\big(1+3\lambda^2\big),
\end{equation}
where the known values from $G_F$, $f$, and $(1+\mathrm{RC})$ have been combined.
The uncertainty on the numerator comes from the combined uncertainties on
the radiative corrections, the leading uncertainty of which comes from
the $\gamma W$-box diagram, and the small uncertainty on $G_F$. Here we see that
by measuring the neutron lifetime and $\lambda$ from free neutron decay,
one can determine $V_{ud}$ solely from neutron decay. 
To date, the most precise determination of $V_{ud}$ comes from nuclear
$\beta$-decay of $0^+ \rightarrow 0^+$ which is purely a vector interaction
and thus does not contain the axial-vector renormalization $g_A$.



\section{Ultracold Neutrons}

While neutrons have a magnetic moment and mass, the typical energy scale
of a free neutron is much larger than any potential energy associated with
the interaction of the neutron with materials, gravititational fields,
or magnetic fields. But what happens if the neutron is cooled down
to where its kinetic energy almost vanishes? The neutron begins to interact
quite differently than typically expected!

In the UCNA experiment, neutrons
with kinetic energies $<~350$~neV and speeds $<~8$~m/s are utilized to measure
the neutron $\beta$-decay asymmetry parameter $A$. Neutrons of such energies
are called ultracold neutrons, and will often be referred to simply as UCN. The
low energy of these neutrons means they can be trapped in material bottles, manipulated
with magnetic fields, and even affected by gravity.

\subsection{Fermi Potential} \label{sssec:fermipotential}

As mentioned, UCN can be confined within material bottles, meaning that they are
totally internally reflected from certain surfaces no matter the angle of incidence.
Thus the slow moving neutrons behave more like a diffuse gas than a typical neutron
beam.

UCN see an effective Fermi potential when encountering a surface, given by
%
\begin{equation}
  V_{\mathrm{eff}} = \frac{2\pi\hbar^2}{m}Na,
\end{equation}
where $a$ is the bound atom coherent scattering length of the material and $N$ is the number density of
the material. This potential can be treated as a one-dimensional non-relativistic quantum mechanics problem,
and what one finds is that when the energy of the incoming wave is less than $V_{\mathrm{eff}}$, the neutron will be
totally reflected, although there is a transmitted component of the wave function which penetrates the
material. This transmitted component can lead to losses of UCN via interactions with the nuclei through
neutron capture or upscattering \cite{golub1991ultra}. A table of typical effective potentials is seen
in Table \ref{tab:fermipotential}.
By constructing the critical guides and other components of the UCNA experiment using materials with high
Fermi potentials, UCN can be guided to the experimental volume and then held until they decay.

\begin{table}[h]
  \caption{Effective potentials for some UCN reflecting materials.} 
  \centering
  \begin{tabular}{l c }
    \hline \\ [-1.75ex]
    Material & $V_{\mathrm{eff}}$ (neV) \\
    \hline \hline \\ [-1.75ex]
    $^{58}\mathrm{Ni}$ & 342 \\
    Stainless Steel & 189 \\
    Cu & 168 \\
    Fe & 210 \\
    Be & 252 \\
    \hline
  \end{tabular}
  \label{tab:fermipotential}
\end{table}


\subsection{Gravity}

As in classical physics, the neutron posesses gravitational potential energy of the form
%
\begin{equation}
  V_g = m_ngh \approx \big(102 \mathrm{~neV/m}\big) \cdot h.
\end{equation}

Normally this is inconsequential for a neutron with typical energies on the MeV scale,
but for UCN with energies comparable to the potential energy associated with only a few
meters change in height, gravity can have a great influence. One might consider using
a change in height to lower the UCN energy to a regime which can be guided by certain
materials as seen in Section \ref{sssec:fermipotential}, or they can be accelerated using
a decrease in height to assist in detection of the UCN \cite{golub1991ultra}.

\subsection{Magnetic Fields}

The neutron has an instrinsic magnetic moment given by
${\mu}_n = -1.91304272(45){\mu}_N$ \cite{pdg} which arises from the quark makeup of the neutron.
The negative sign indicates that the neutron magnetic moment is anti-aligned with its spin.

A magnetic moment interacts with a magnetic field through the $-\vec{\mu} \cdot \vec{B}$ potential, so for
a neutron we have
%
\begin{equation}
  V_m = -\vec{\mu}_n \cdot \vec{B} = \mu_n \hat{\sigma} \cdot \vec{B} \approx \pm \big(60 \mathrm{~neV/T}\big)
  \hat{\sigma} \cdot \vec{B}, 
\end{equation}
where $\hat{\sigma}$ is a unit vector in the direction of the spin. From this it is apparent that a spin which is
aligned to the field will feel a repulsive force from the potential, while a spin of the opposite orientation
will see an attractive potential. The strength of the potential also indicates that a modest magnet of several
Tesla will completely reflect a spin aligned with the field. We'll return to this in the next chapter.


\section{Motivation in a Nutshell}

The focus of this thesis is describing in detail a new precision
measurement of the neutron $\beta$-decay asymmetry parameter $A_0$.
While a measurement of $A_0$ is alone interesting as it presents
a measure of parity
violation within the weak interaction, the necessity for continued measurements of $A_0$
may not be so obvious, given that we already know parity is violated maximally in the
weak interaction. The motivation
becomes apparent though when considering the relationships between all of the
$\beta$-decay correlations and the coupling constants of Lee and Yang's version of Fermi's
theory of $\beta$-decay
(first derived by Jackson, Treiman, and Wyld \cite{jackson1957a} and
described in Section \ref{ssec:correlations}). From this relationship we see that precision
measurements of any $\beta$-decay observable (i.e. $A_0$) either produces a measurement of the couplings
known to exist ($g_A$ and $g_V$ or the ratio $\lambda=g_A/g_V$), or tests the couplings believed to not exist within the
Standard Model (scalar, pseudoscalar, tensor).

\subsection{Why Determine $\lambda$?}

Determining $\lambda$ from measurements of the correlation coefficients is more fundamental to
weak interaction physics, as the value is not determined within the theory. Therefore,
theoretical calculations involving the weak interaction use as input the experimentally
determined value for $\lambda$, and the precision of the calculation is limited by the
precision of the experimental value. One may have even noticed that the recoil order corrections to
the asymmetry parameter in
Section \ref{sssec:ROCorr} are themselves dependent on $\lambda$.

Precision measurements of $\lambda$ not only allow for more reliable theoretical calculations, but
they also serve as complimentary benchmarks for lattice QCD (LQCD) calculations, where the continuous
field theory of the strong interaction (Quantum Chromodynamics, hence QCD)
is discretized on a finite four-dimensional lattice (space and time). By solving the theory on
the lattice and extrapolating to an infinitely large lattice with lattice spacing of zero 
and physical masses for the particles, one can, for example, extract the coupling constants in the
hadronic current (Equation \ref{eq:hadronicCurrent}). The limiting factor for lattice QCD
determinations of any parameter is the statistical precision from finite Monte Carlo sampling
and systematic uncertainties associated with the extrapolations, but recent improvements in
LQCD calculatations \cite{bhattacharya2016,capitani2017iso,berkowitz2017accurate,chang2017nucleon}
show promise in soon reaching comparable precision for $g_A$ in relation to
experiment. As such, comparison between the measured and LQCD results will allow for limits on
beyond Standard Model right-handed currents in the weak interaction, as what is actually measured experimentally
is $g'_A = g_A(1-2\epsilon_R)$, where $\epsilon_R$ denotes the effective right-handed coupling if present
\cite{cirigliano2013beta,gonzalez2016global,alioli2017right}.

\begin{figure} [h]
  \centering
  \includegraphics[scale=0.5,page=1]{1-Introduction/A_results.pdf}
  \caption{Historical plot of $A_0$ measurements prior to the analysis presented within
    this thesis
    \cite{bopp1986,erozolimskii1991new,yerozolimsky1997,liaud1997,abele2002,mund2013,mendenhall2013}.
    The shaded band indicates the Particle Data Group average value \cite{pdg}
    for the asymmetry parameter, where the solid data points are included in the average. Thank you
    to Dr. Plaster for inclusion of this plot.}
  \label{fig:Ameasurements}
\end{figure}


\subsection{Current Status of $A_0$ Measurements} \label{ssec:Previous_results}

Measurements of $A_0$ provide the most sensitive value of $\lambda = g_A/g_V$,
which we now see has profound impact within the physics community.
Measurements of $a_0$, the correlation between the electron momentum and
neutrino momentum, are of comparable sensitivity, but such measurements are inherently more difficult.
They require either the reconstruction of the neutrino momentum through measurement of the recoil proton
kinetic energy \cite{darius2017measurement}, which is less than 760~eV compared to the $\sim782$~keV endpoint
of the electron, or through direct measurement of
the proton energy spectrum \cite{byrne2002determination}. The most precise measurement of $a_0$ to date
is at the sub-4\% precision level, while current $A_0$ measurements are sub-1\% precision, thus providing
better determination of $\lambda$ for the time being.

Figure \ref{fig:Ameasurements} shows the evolution of $A_0$ measurements for the neutron dating back to
1986 and not including the most recent result which is the topic of this thesis. We see that
with time not only has the precision increased, but a dichotomy has arisen between older measurements
(pre-2002) and more recent measurements. The difference between the two groups may be attributed
to the relatively large systematic corrections applied for the older pre-2002 results ($>10\%$) compared to
all corrections being $<2\%$ for the newer post-2002 results \cite{brown2017}.

The Mendenhall \textit{et al.} result \cite{mendenhall2013} is the UCNA predecessor to the current analysis
presented within this thesis. It should be noted that all measurements prior to Mendenhall
\textit{et al.}\footnote{There are earlier UCNA results \cite{pattie2009first,liu2010determination}, but only the more recent
  Mendenhall result is used in calculating the PDG average of $A_0$.}
used cold neutron beams, whereas UCNA utilizes ultracold neutrons, and thus necessitates far different
systematic corrections to the measured asymmetry compared to the other experiments. This makes the UCNA experiment
a great complement to the other measurements to ensure no unforeseen systematic bias to the experimentally
determined value of $A_0$ and $\lambda$ from cold neutron techniques. Present results from UCNA already
confirm that the shift in $A_0$ values seems to
be motivated by better experimental techniques as a whole and that post-2002 results are perhaps more aligned with the
underlying value. From this point on, we focus on a new analysis of UCNA data in an
attempt at further reducing the uncertainty on the $\beta$-decay asymmetry parameter $A_0$ using UCN, thereby improving
the precision on $\lambda=g_A/g_V$.









%\section{Previous measurements of $A_{0}$}
%\label{sec:Previous_results}

%\subsection{2010 UCNA Analysis}
%\subsection{PERKEO}

%\section{Summary}



