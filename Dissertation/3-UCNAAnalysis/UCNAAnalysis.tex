\chapter{UCNA Analysis}
\label{ch:UCNA_Analysis}
%%%%%%%%%%%%%%%%%%%%%%%%%%%%%%%%%%%%%%%%%%%%%%%%%%%%%%%%%%%%%%%%%%%%%%%%%%%%%%%
%%%%%%%%%%%%%%%%%%%%%%%%%%%%%%%%%%%%%%%%%%%%%%%%%%%%%%%%%%%%%%%%%%%%%%%%%%%%%%%
%%%%%%%%%%%%%%%%%%%%%%%%%%%%%%%%%%%%%%%%%%%%%%%%%%%%%%%%%%%%%%%%%%%%%%%%%%%%%%%

Designing an experiment and collecting the right data are non-trivial alone,
but calibrating the system and analyzing the data holds the meat and bones of
any trustworthy scientific result. The calibration is a beautiful combination
of simulation and data manipulation which allows one to extract the energy of
an event based solely on some electronic signal. Imagine a baseball pitcher
throwing a fastball into a sheet, and the observer behind the sheet must
determine the velocity of the ball from only seeing the impression the pitch
made on the sheet. This is the task every nuclear physics experiment is faced
with, only the baseball is a particle and the sheet is our detecotr system.
Below we focus on the energy calibration of our apparatus.


\section{Simulation}
\label{sec:Simulation}

The simulation work for this thesis was completed using the Geant4 simulation software
package. The geometry of the apparatus was duplicated to the best precision possible
and benchmarked against data. The structure of the simulation code did not change from
those in the previous analysis as described in \cite{mpmThesis}, although minor changes
were made to the geometry.


\section{Gain correction and pedestal subtraction}

\section{Activated Xenon}
\subsection{Wirechamber Calibration}

Blah Blah

\subsection{Position Dependence}

Blah Blah

\subsection{Trigger Thresholds}

\section{Energy Calibration}
\subsection{Electron Conversion Sources}
\subsection{Linearity Curves}








