\chapter{UCNA Analysis}
\label{ch:UCNA_Analysis}
%%%%%%%%%%%%%%%%%%%%%%%%%%%%%%%%%%%%%%%%%%%%%%%%%%%%%%%%%%%%%%%%%%%%%%%%%%%%%%%
%%%%%%%%%%%%%%%%%%%%%%%%%%%%%%%%%%%%%%%%%%%%%%%%%%%%%%%%%%%%%%%%%%%%%%%%%%%%%%%
%%%%%%%%%%%%%%%%%%%%%%%%%%%%%%%%%%%%%%%%%%%%%%%%%%%%%%%%%%%%%%%%%%%%%%%%%%%%%%%

Designing an experiment and collecting the right data are non-trivial alone,
but interpreting the results lends itself to a whole new train of thought.
The rest of this thesis is spent describing the details of such an
analysis, with this chapter summarizing important aspects,
such as terminology to be used later and the
model used to characterize our detector response.

%--------------------------------------------------------------


\section{Interpreting Detector Response}
While all pertinent components of the experimental apparatus were included
in the simulation, the detector response is not inherent within our
output. You may notice from (figure with quenched energy) that this
doesn't resemble the gaussian peaks normally seen in a detector signal.
The gaussian-like peaks arise from finite resolution effects and
must be put in defacto using parameters calculated
from real data. The better we understand the conversion from a pure simulation
energy output to a response which mimics that from our actual detectors, the
more credence we can put in our calibration. For the time being, the model
will be introduced with parameter determination addressed as it arises. 

\subsection{Energy Response} \label{sssec:EnergyResponse}
In the experiment, we have rather precise access to the energy signal via the
four PMTs coupled to each scintillator. The energy manifests itself initially as a voltage
which is converted to a digital signal. The system is designed such that this channel
read-out is proportional to the energy seen by each PMT. The visible energy $E_{vis}$ deposited
in the scintillator as seen by a single PMT, say PMT $i$, for an event at position $(x,y)$ 
is given by the following: 

\begin{equation} \label{eq:EvisResponse}
E_{vis} = \eta^-1(x,y) \cdot f_i\left( \left( ADC_i - (p_i(t) + \delta p_i(t))\right) \cdot g_i(t) \right)  ,
\end{equation}

\noindent where 
\begin{align*}
&f_i^{-1} = \textrm{inverse of the linearity relation from ADC channels to Energy,}\\
&\eta_i(x,y) = \textrm{PMT correction factor for position dependence,} \\
&p(t) = \textrm{mean pedestal value for PMT } i,\\
&\delta p(t) = \textrm{randomly sampled pedestal width for PMT }i,\\
&g(t) = \textrm{gain correction factor for PMT }i.
\end{align*}

This expression is exact in the case where all values are determined with infinite
precision and without stochastic fluctuations. Unfortunately, each parameter on the right
side of this equation is either stochastic in itself (as is the ADC response), or it was
determined via observation of a stochastic process (the gain and pedestal). Thus what we
really resolve is an approximation to the energy, which comes with some uncertainty. This
uncertainty will be addressed later.


\section{Time-dependent Detector Corrections}

Obviously the system is not immune to drifts in signals due to variations
in time. There are many sources of such drifts, ranging from simple
electronic noise to changes in temperature. We deal with such time-dependent
effects using pedestal subtraction, gain correction, and constant monitoring
of backgrounds.

\subsection{Pedestal Subtraction}
The pedestal is a measure of the inherent detector signal, or baseline, 
upon which all other data signals lie. In terms of PMT signals, you can imagine 
the pedestal as a non-zero $ADC$ value corresponding to zero input, or an offset.
You might say that the experiment can be run without caring about an offset
because the calibration will take this into account, which would be the case 
if the pedestals were constant or if we calibrated each run against itself, but 
neither is the case. We use a collection of subsequent runs to form our 
calibration sets, and these sets then calibrate data which is often taken hours,
or even days, earlier or later. Thus changing pedestals can be worrisome, and care
must be taken to determine the pedestals and subtract them from data.

To determine a pedestal, events must be chosen where there was a global trigger, but
the PMT of interest does not trigger and preferably there is no photon signal
whatsoever in the scintillator on that side. Obvious choices for these events are
UCN monitor triggers, opposite side 2-fold PMT triggers, and high-threshold $^{207}$Bi
pulser triggers. Once there is a global trigger, the TDC for the PMT of interest can 
be cut on to be sure there was no individual trigger and the events can be 
histogrammed. The mean of this peak can be taken as the average pedestal over whatever
time period the data was taken (nominally a single run), and this value can be 
subtracted from every subsequent reading of this PMT.

One interesting thing to note is that the discriminators for all PMTs are housed 
together, which leads to correlations between the PMT signals. In a perfect world, 
each PMT would have one pedestal, and that pedestal wouldn't care about other PMTs.
Instead, what we see in figure \ref{fig:peds_types} is that the pedestals
are substantially dependent on the type of events that are chosen 
to construct the pedestal. 

The influence of event type on pedestal values heavily influenced the decision made
to eliminate the UCN monitor events and only use $^{207}$Bi pulser event 
triggers to construct the pedestals in this analysis. While the best 
choice would be UCN monitor events due to there being zero signal 
in the electronics box housing the PMT electronics, these aren't readily 
available for most source runs due to beam being off, which could lead to 
us introducing pedestal drifts based solely on the events chosen to 
construct them. To further reduce correlations in signals across PMTs, the pulser
events were also limited to those coming from opposite side PMTs.
Consistency amongst all runs 
is key, mainly due to the fact that we are using pedestal subtracted source runs
to calibrate the detectors, and then we are applying these calibrations to the 
$\beta$-decay data. A pedestal shift of ~5-10 channels maps to an offset of roughly
~5-10 keV as well as the PMTs show a 1:1 correspondence between ADC and keV on 
average. 

\begin{figure}[h] \label{fig:peds_types}
\centering
\includegraphics[scale=.25]{3-UCNAAnalysis/ImageHolder.pdf}
\caption{Pedestal values for a $\beta$-decay run determined using different 
types of events to illustrate the cross-talk between PMTs. (UCN Monitors, 
Bi triggers, Opposite side triggers, same side 2-fold triggers. Also shown 
is the dependence of the pedestal on which PMT triggers in the Bi Pulser. NOTE:
Choose West PMT4 in an early 2012/2013 beta run) }
\end{figure}

\begin{figure}[h] \label{fig:peds_timeDep}
\centering
\includegraphics[scale=.25]{3-UCNAAnalysis/ImageHolder.pdf}
\caption{Pedestal means as a function of run number. Error bars are the
RMS of the measured pedestal. }
\end{figure}

\begin{figure}[h] \label{fig:peds_timeDep}
\centering
\includegraphics[scale=.25]{3-UCNAAnalysis/ImageHolder.pdf}
\caption{Example pedestals from all 8 PMTs}
\end{figure}

\subsection{Gain Correction}

\subsection{Time-dependent backgrounds}
Here I just want to elude to the fact that these are taken into account in
the super-ratio.

%----------------------------------------------------

\section{Position Dependence}

\subsection{Activated Xenon}


\subsection{Position Maps}

%----------------------------------------------------------

\section{Trigger Thresholds}
An integral piece of the PMT Response Model from section \ref{sssec:pmtModel} 
is the sampling of the threshold functions to determine whether or not a detector
triggered. I'll repeat here that a 2-fold PMT trigger from one of the two electron
detectors is required to create a global electron trigger from the DAQ. In simulation
we only see the energy deposition as a whole from an individual scintillator, and then
we model the four-pmt response for that side. At low energies, this response is 
intimately entwined with the trigger threshold.

\subsection{General Model for Trigger Determination} \label{ssec:genTrigModel}
Obviously one must rely on data to determine the trigger thresholds of a detector, since
the non-step-functional shape is directly related to the stochastic nature 
of the detector and electronics. If we knew with infinite precision and 
accuracy the signal produced in a detector
and could read out the response in real time with infinite precision, 
there wouldn't be muchneed for a model to estimate trigger probabilities. 
Instead, to understand the trigger threshold, a functional 
form for the probability of a trigger need be determined using real data which may or 
may not have created a trigger in a detector/PMT. 

The most important part of determining the trigger threshold shape for 
any detector is the availability of data which was collected no matter if 
the detector or component (PMT) produced a trigger. If such a subset of 
data is available and plentiful, it is straightforward to estimate
the trigger probability by binning the data in some unit proportional to 
energy (whether in energy or something like it isn't important) and taking 
the bin-by-bin ratio of those events that triggered to all of the events in the 
sample. Plotting these ratios as a function of whatever metric was chosen tells 
you how probable an event of some value is to create a trigger.
Once you have mapped this probability, you can then sample these curves within 
simulation to apply your true trigger threshold to simulated data.

The not-so transparent part of the trigger probability comes when calculating the error of the estimate in each bin, since the numerator and denominator in the ratio are correlated. But as demonstrated in \cite{casadei2009efficiency}, the ROOT analysis framework handles efficiency errors effectively via use of Bayesian Statistics. 

PUT IN DISCUSSION OF BAYESIAN STATISTICS FOR EACH BIN AND ASYMMETRIC ERROR BARS
AND CONFIDENCE LIMITS.

\subsection{Trigger Data Selection}
As mentioned before, the data used for constructing trigger thresholds must not be 
biased towards triggering the PMT of interest. Thus that PMT must not be a mandatory 
component of the global trigger for that event, so care must be taken to choose only 
events which would trigger regardless of the behavior of the PMT of interest. One
other stipulation placed on these events is that they have an opportunity to 
deposit energy in a particular scintillator. The best choice of events which 
satisfy these conditions are those which have a two-fold trigger on the opposite side 
and then backscatter and those which trigger at least three PMTs on the side of 
interest, which guarantees that the scintillator would have triggered with or without 
whatever PMT one is interested in.	

\subsection{Determining the Trigger Probability}
One option for determining the trigger probability function (and probably the 
most straightforward) is to calculate the trigger probability for an entire detector as 
a whole as a function of the energy deposited by an event. What you get is a 
function that provides the probability that an event of energy $ E_i $ 
produces some sort of trigger, either 2-fold, 3-fold, or 4-fold, in that 
detector. Initially this method was used for sake of simplicity, and it produced 
reasonable agreement between simulation and data, but there is one 
glaring concern: Determining this trigger function from data requires that the data be 
calibrated first. At first glance this may not seem like much of an issue, but the 
calibration hinges upon the 
simulated peaks at low energy, which in turn rely on the trigger functions. This 
cyclical dependence hinders one from truly understanding any discrepancy between 
simulation and data at low energies, which is exactly the reason this method was 
abandoned.  

Instead, similarly to previous analyses, we decided to calculate the trigger
function on a PMT-by-PMT basis as a function of ADC channels above threshold. This
encompasses a true characteristic of each component of the detector rather than some
average effect as seen by a detector package, which is what the aforementioned 
method produces. A typical trigger threshold is seen in figure \ref{fig:trigger_thresh}.
As illustrated in section \ref{ssec:genTrigModel}, the ratio of triggering events
to all events was taken in each ADC bin and then fit using the method described
in the following section.
  

\begin{figure}[h] \label{fig:trigger_thresh}
\centering
\includegraphics[scale=.25]{3-UCNAAnalysis/ImageHolder.pdf}
\caption{Typical trigger threshold functions with Bayesain errors 
applied. }
\end{figure}

\subsubsection{Functional Fit of the Trigger Threshold}

\subsubsection{Effects of cross-talk on trigger thresholds}

A subset of runs in the 2012/2013 data set show severe correlation between 
which PMTs cause the global trigger and the side for which the trigger thresholds 
are being determined. In short, an electron backscattered from the primary trigger
side samples a very different trigger threshold when impinging on the opposite detector 
than it would have if it only struck the opposite detector. This is depicted 
in figure \ref{fig:evtTypeTriggers}, where we see ~(insert number)
ADC shift in the $P=0.5$ ADC channel for backscattering events which triggered the
opposite detector and those which only triggered the detector of interest.

\begin{figure}[h] \label{fig:evtTypeTriggers}
\centering
\includegraphics[scale=.25]{3-UCNAAnalysis/ImageHolder.pdf}
\caption{Trigger threshold functions for Type 0/2/3 events vs. Type 1 events for the 
West detector in an early 2012/2013 beta decay run. }
\end{figure}

%-----------------------------------------------------------


\section{Calibration Overview}

\subsection{PMT Calibration}

\subsection{Wirechamber Calibration}

 
\section{Backscattering} \label{sec:backscattering}

Backscattering identification plays
an important role in our Monte Carlo corrections and asymmetry extraction.
Based on which detector components trigger, we classify events
into those that do not backscatter
(Type 0) and those that do backscatter (Types 1, 2, and 3) \cite{plaster12}. Type 0 events
trigger one scintillator and one MWPC on the same side, while Type 1 events trigger
both scintillators and both MWPCs. For such events, we assign the initial
direction to the triggering detector for Type 0 and to the earlier triggering detector
for Type 1. Type 2/3 events comprise a class of events that backscatter and trigger both
MWPCs, but only trigger a single scintillator.
The initial direction of such events can
not be determined from trigger logic alone, as can be demonstrated by considering two events
which look identical under trigger logic.
For example, let ``Event 1'' denote an event that initially backscatters off
of MWPC 1 before reaching scintillator 1
and then traverses the length of the decay
trap to trigger both MWPC 2 and scintillator 2 on the opposite side.
Then, suppose ``Event 2'' denotes another event emitted
in the opposite direction to event 1. Suppose this event
triggers MWPC 2 and scintillator 2
only to backscatter from the scintillator and travel to MWPC 1 and stop short
of scintillator 1. Both events trigger MWPC 1 and 2 and scintillator 2,
but the two events had opposite initial directions, so inclusion of the
two events without further knowledge of their initial direction creates
a dilution to the asymmetry.
An important distinction, however, does exist between Type 2 and Type 3 events:
Type 2 events only pass through the MWPC on the
triggering scintillator side once, whereas Type 3 events scatter from
the scintillator, and therefore pass through the MWPC twice on
the triggering side. We can consequently apply a cut on the energy deposited in
the MWPC on the triggering side to statistically assign
Type 2/3 events to the correct side.
This drastically reduces
Monte Carlo corrections for such backscattering events as simulation indicates we
properly identify $>80\%$ of all Type 2/3 events across all energies using this
technique, a marked
improvement over the roughly $50\%$ misidentification rate without separation.

Show the schematic of the different backscattering types. Also explain the 
angular dependence, maybe even show plots of backscattering types vs energy 
and angle from simulation.

\begin{figure}[h]
\centering
\includegraphics[scale=.25]{3-UCNAAnalysis/ImageHolder.pdf}
\caption{Different event types as defined by the type of backscattering.}
\end{figure}

\section{Data Structure}

\begin{figure}[h]
\centering
\includegraphics[scale=.25]{3-UCNAAnalysis/ImageHolder.pdf}
\caption{Schematic of the Octet run sequence.}
\end{figure}

Discuss the use of the octet analysis

\section{Polarimetry} \label{sec:polarimetry}







