\chapter{UCNA Calibrations}
\label{ch:UCNA_Calibrations}
%%%%%%%%%%%%%%%%%%%%%%%%%%%%%%%%%%%%%%%%%%%%%%%%%%%%%%%%%%%%%%%%%%%%%%%%%%%%%%%
%%%%%%%%%%%%%%%%%%%%%%%%%%%%%%%%%%%%%%%%%%%%%%%%%%%%%%%%%%%%%%%%%%%%%%%%%%%%%%%
%%%%%%%%%%%%%%%%%%%%%%%%%%%%%%%%%%%%%%%%%%%%%%%%%%%%%%%%%%%%%%%%%%%%%%%%%%%%%%%

Detector calibrations are a beautiful combination
of simulation and data manipulation which allows one to extract the energy of
an event based solely on some electronic signal. Imagine a baseball pitcher
throwing a fastball into a sheet, and the observer behind the sheet must
determine the velocity of the ball from only seeing the impression the pitch
made on the sheet. This is the task every nuclear physics experiment is faced
with, only the baseball is a particle and the sheet is our detector system.
Below we focus on the energy calibration of our apparatus.

This chapter will begin by discussing the position reconstruction algorithm,
which relies on signals from the wirechamber (MWPC). Then we will describe
the energy calibration of the scintillators which utilizes the
position of the events, as each individual PMT response is highly position
dependent. Last of all, an energy calibration of the wirechamber will be
discussed, which, while not imperative to the analysis, provides the ability to
distinguish between a Type 2 and Type 3 backscattering event more
effectively.

%--------------------------------------------------------------


%----------------------------------------------------------

\section{Wirechamber Position Reconstruction}

\subsection{Wirechamber signals}

The wirechamber signals for each detector include a summed anode signal, and two collections of
cathode signals consisting of 16 individual ``wire'' readings. Again we use the term
wire loosely, as there are technically 64 wires in a plane, but they are read out in groups
of four. The position of each wire group is taken to be the midpoint of the grouping (between
the two center wires). The two sets of cathode signal collections come from perpendicular
planes, so the position can be reconstructed along both the $x-axis$ and $y-$axis.

The signal is read out by a peak sensing ADC, so the maximum value of the signal is recorded
for each event. The pedestal is determined using the bismuth pulser events (different from the
PMT pedestals) and is subtracted from the ADC signal for every event. For the cathodes, a software
threshold for each wire is set at 100~channels above pedestal, so only wires above this threshold
will be used in the position reconstruction.

\subsubsection{Wirechamber trigger} \label{sssec:mwpctrigg}

As mentioned in sections \ref{sssec:ExpMWPC} and \ref{sec:backscattering}, a wirechamber
software trigger is set and used to eliminate the gamma background and to identify different
backscattering events. One could use either the summed anode signal or a signal formed from
the cathode signals. The best trigger is one that separates the pedestal furthest from the
triggering data. For this analysis, this was found to be the sum of the maximum cathode
signals from the two wire planes.

\subsubsection{Cathode wire clipping}

Low energy events deposit a substantial amount of energy in the MWPC, and thus create large
signals in one or more wires in cathode planes. This can create a signal which is beyond the
range of the ADC, producing an overflow event which is read out as the max value of the ADC
regardless of how much actual charge was deposited in the wirechamber.
These events are called ``clipped'' events. A plot of the typical ratio of clipped events and the
number of clipped wires can be seen in Figure \ref{fig:nClipped}.

\begin{figure}[h]
  \centering
  \includegraphics[scale=0.6,page=1]{4-UCNACalibrations/mwpc_position.pdf} 
  \caption{Number of clipped wires in the $x$-plane of the East detector. Most of the events
    (\~92~\%) exhibit no clipping.}
  \label{fig:nClipped}
\end{figure}


A clipped event poses a problem in the position reconstruction because the true signal
is no longer known, only that it was above the maximum ADC value. Using this wire in
the position reconstruction will not properly account for the strength of the
signal at the position of the clipped wire.

\subsection{Wirechamber position reconstruction}

Determining the position of the spiraling electrons is vital for every component of the analysis
that follows. Before now, the gain, pedestals, and trigger thresholds are determined with no
knowledge of the position of an event, but rather basic trigger logic. As will be seen in
the Section \ref{ssec:posmaps}, the actual response of the PMT is position dependent, and so we introduce the
position reconstruction algorithm now.

\subsubsection{Events with no cathode clipping}

The algorithm developed for reconstructing the positions of the events for this analysis
is meant to be as simple of possible, with more complex measures taken for special events.
If we have an event that passes through the MWPC, creates a software trigger, and has no
clipped wires, then the position is calculated as the average position of the signals:
%
\begin{equation}
  \bar{x} = \frac{\sum q_i x_i}{q_i},
\end{equation}
%
where the sum runs over wires above the individual pedestal subtracted threshold, $q_i$ are the
pedestal subtracted wire signals, and $x_i$ are the positions of the wires.  This can
be interpreted either as a weighted average of the wire positions with the weights equal to
the wire signals, or it can be seen as a simple average where each unit of the signal is seen
as an event entering into the average with a value equal to the wire position. The final position
of the event is then given by $(\bar{x}, \bar{y})$, where $\bar{y}$ is calculated in the same manner.

\begin{figure}[h]
  \centering
  \begin{tabular}{cc}
    \subfloat[East Detector]{\includegraphics[scale=0.37,page=2]{4-UCNACalibrations/mwpc_position.pdf}} &
    \subfloat[West Detector]{\includegraphics[scale=0.37,page=3]{4-UCNACalibrations/mwpc_position.pdf}}
    \end{tabular}
    \caption{Wirechamber reconstructed positions for all electron events in a single octet of
      $\beta$-decay data. The black dashed line indicates the 50~mm fiducial cut applied to the
      data during asymmetry extraction.}
  \label{fig:wirechamberPos}
\end{figure}

For well-behaved wires (no clipping, no missing or ``dead'' wire segments), this method produces
a continuous reconstruction of the events based on where they pass through the wirechamber.
An example of the position dependence can be seen in Figure \ref{fig:wirechamberPos}.

\subsubsection{Events with cathode clipping or missing wire signals}

In the event of a clipped wire or the rare case of a missing wire signal, the basic average method
shown above will not suffice. Using the clipped wire in the average can systematically shift the
reconstructed position if the signal is not symmetric about the clipped wire, and the same can
be said of a missing signal. While this may not appear as a visible distortion to the
position distribution, the method
loses its integrity.

To handle these types of events, a method similar to that used in the previous analysis
\cite{mpmThesis} was adopted. By assuming that the wirechamber charge cloud (or
ionization cloud) takes a Gaussian shape in the MWPC, we can expect that the signals
in the wires should also sit on a Gaussian given by
%
\begin{equation}
  q = Ne^{(x-\bar{x})^2/\sigma^2},
\end{equation}
%
where $q$ is the signal as a function of position, $x$ is the position of the signal, and $N$ is
a scaling parameter. If we take the log of this
expression we have
%
\begin{align} \label{eq:gaussPos}
   &\ln(q) = \ln(N) + \frac{(x-\bar{x})^2}{\sigma^2} \\
   &\ln(q) = \frac{1}{\sigma^2}x^2 - \frac{2\bar{x}}{\sigma^2}x + \Big(\ln(N)+\frac{\bar{x}^2}{\sigma^2}\Big) \\
   &\ln(q) = Ax^2 + Bx + C 
\end{align}
%
with
\begin{equation}
  A = \frac{1}{\sigma^2}, \qquad B = - \frac{2\bar{x}}{\sigma^2}, \qquad C = \ln(N)+\frac{\bar{x}^2}{\sigma^2}.
\end{equation}
This equation has three unknowns, thus with three ``good'' wires giving three points on the Gaussian, the unknowns
are fully determined. The three chosen wires are the maximum non-clipped wire and the next two largest signals
surrounding this maximum signal. These two wires can be on one side of the maximum wire, as would be the case if the
maximum wire were the edge wire. The extraction of the mean is the approximation of the center of the event.

\begin{figure}[h]
  \centering
  \includegraphics[scale=0.5,page=7]{4-UCNACalibrations/mwpc_position.pdf} 
  \caption{Collection of extraced $\sigma$ values for the Gaussian determination of event position. The
    average of this distribution is an estimate of the characteristic width of the wirechamber charge cloud,
    $\sigma_c$, used when only two wires are usable in the position reconstruction.}
  \label{fig:meanSigma}
\end{figure}

Now there is also the special case where there are only two usable wires above the individual cathode
wire threshold. If they are consecutive wires, the average method is applied, but if they are not consecutive wires
(so they are separated by a clipped wire(s)), a new approximation must be applied. The expression in
equation \ref{eq:gaussPos} requires three wire signals to estimate a position, so we must eliminate one of the unknowns
to use this method. Again, if we assume that all signals follow a typical Gaussian shape, then the width of the
charge cloud should be roughly constant. Thus we can determine a characteristic width, $\sigma_c$, from the
mean extracted $\sigma$ from the rest of the events (including the non-clipped events). A histogram of the
extracted $\sigma$ values can be seen in Figure \ref{fig:meanSigma}. A value of $\sigma_c = 6\text{~mm}$ was used
for these type events, and then the mean was calculated by solving \ref{eq:gaussPos} with two unknowns.

Last of all, there is the situation where only one wire was above the individual threshold. For this type of event,
the only reasonable choice is to set the position of the event to the position of this wire.

\subsection{Simulation of wirechamber positions} \label{ssec:simWirePos}

In order to reduce any unforeseen systematic effects, an attempt at including
every aspect of the experiment within the simulated data is made. Thus for the wirechamber
position reconstruction, we would like to employ the methods outlined above for every
simulation event also. 

Embedded within the simulation by M. Mendenhall during the previous analysis
is a model for the charge collection within the wirechamber based on the work
from \cite{mathieson1991induced}. In summary, based on the wirechamber geometry, an estimate
of the charge cloud as seen by the cathode and anode can be calculated for an event
that deposits energy $E_{\mathrm{MWPC}}$ in the wirechamber. The agreement between data and simulation was
shown to be good (\cite{mpmThesis} section 6.3.6), and thus the charge collection
model was used in this analysis.

A new contribution to the model is the application of the observed wirechamber thresholds
and clipping conditions to each of the cathode wires. The model already included is simply an estimation of the charge
collection on the range from $0$ to $\infty$. This could be used in the simple average method for ``good'' events
with no clipping and no missing wires, but then any systematic effects from wire clipping would go unnoticed.

\subsubsection{Wire model}

For each wire in each plane, the response can be characterized by two values: a clipping threshold and a trigger threshold.
Recall that we apply an individual software trigger threshold of 100 ADC channels for each wire, so a model parameter must
be determined for this. Also, in the simulation model, the signal on each cathode is not bound, so an artificial
clipping parameter must be introduced.

To determine the trigger threshold $E_T$ for a single cathode wire in a given plane, the ratio of
events with signal above the software trigger threshold for that wire
to the total number of electron events identified by that detector is calculated as
\begin{equation}
  R_{T} = \frac{N_T}{N_{ALL}},
\end{equation}
where $N_T$ indicates ``trigger'' events and $N_{ALL}$ refers to the total number of electron events.

Then for the corresponding simulation of this wirechamber cathode wire, the trigger threshold $E_T$
is applied starting at $E_T=0.01$~keV and the same ratio as above is calculated for the simulation,
$R^{\mathrm{sim}}_{T}$. The
threshold is incrementally increased by $0.01$~keV and the ratio is recalculated until
$R^{\mathrm{sim}}_{T} = R_T$. The value for $E_T$ is saved for application within the new simulation
model.

Now with the knowledge of the low energy threshold, a similar method for the high energy clipping
threshold $E_C$ can be applied for this wire. The ratio of events from data becomes
\begin{equation}
  R_{C} = \frac{N_C}{N_{T}},
\end{equation}
so that this is now the ratio of clipped events to triggered events for the wire of interest.
Then within the simulation, the clipping threshold can be scanned down from an arbitrarily high
threshold until $R^{\mathrm{sim}}_{C} = R_C$. It was found that starting 
the clipping threshold at $E_C=9$~keV and incrementing by $-0.1$~keV provided nice enough
agreement while not using exceptionally high computation times, as this process is carried out
for every wire grouping (64/run) in every $\beta$-decay run ($\sim1000$).

\begin{figure}[h]
  \centering
  \begin{tabular}{cc}
  \subfloat[Multiplicity]{\includegraphics[scale=0.3,page=5]{4-UCNACalibrations/mwpc_position.pdf}}  &
  \subfloat[Number of clipped wires]{\includegraphics[scale=0.3,page=4]{4-UCNACalibrations/mwpc_position.pdf}}
  \end{tabular}
  \caption{Comparisons between data (blue line) and simulation (red dashed line) after
    the application of the single wire trigger and clipping model. There is a discrepancy in the multiplicity,
    but the general features are captured. The number of clipped wires in data and simulation is in better agreement.}
  \label{fig:multDataSim}
\end{figure}


\subsubsection{Results of individual wire model}

The agreement between data and simulation from application
of the trigger threshold can be seen in Figure \ref{fig:multDataSim} a.), where
the multiplicity of triggered wires for both data and simulation is shown. Without the
the application of a nonzero individual trigger threshold, the multiplicity would
generally be much higher for the simulation.


\begin{figure}[h]
  \centering
  \includegraphics[scale=0.6,page=6]{4-UCNACalibrations/mwpc_position.pdf} 
  \caption{Position dependence of clipped events for data (blue line) and simulation (red dashed line) after
    the application of the single wire trigger and clipping model to the simulation. The model captures the
    position dependence of the clipping nicely.}
  \label{fig:clippedPos}
\end{figure}

The application of the clipping threshold to the simulation also provides nice agreement
as seen in Figure \ref{fig:multDataSim} b.). More importantly, the position dependence
of the clipped events is properly accounted for in the simulation as shown in figure
\ref{fig:clippedPos}.

With these effects accounted for within the simulation, effects regarding position
reconstruction should be accounted for within the systematic corrections. Any subsequent
MWPC systematic effects from the efficiency of the MWPC trigger described in section
\ref{sssec:mwpctrigg} is accounted for separately as will be shown in Section \ref{sssec:mwpcEff}.

%The disagreement
%in the zero bin is attributed to the possibility of the MWPC generating a trigger given the
%summed max cathode trigger cut from Section \ref{sssec:mwpctrigg} while having no signal above
%threshold
%----------------------------------------------------






\section{Scintillator Energy Calibration}

\subsection{Method}

The goal of the energy calibration is to provide a means for taking as input an ADC value
from a PMT and returning as output an estimate of the energy deposited in the scintillator
to which the PMT is coupled. For a given event, one will have up to four estimates from a
single detector from the four PMTs on each side (less than four if a PMT is not functioning
properly). Then, given the event type and the method for combining multiple PMT responses in
Section \ref{sec:EnergyResponse}, we can produce a single estimate of the initial energy of
the event.

The energy calibration is briefly highlighted in Sections \ref{sssec:scintEnCal} and
\ref{ssec:PMTCal}. Figure \ref{fig:sourceTranslate} nicely depicts in a cartoon how
the conversion electron sources are translated across the detector face using a source
paddle. The different source locations sample different regions of the PMT light transport
maps, thus providing markedly different PMT responses from an individual source. This allows
one to characterize the ADC response to a broad range of light exposure using only three conversion sources.
The entire process takes place without breaking vacuum, so the calibrations can be done
intermittently during $\beta$-decay running.

Up to this point, the parameters that enter the calibration have been well-defined and only need
to be determined once (pedestals, gain, event positions, etc.). To determine
the relationship between the ADC response in a PMT and the energy deposited, we instead turn to a cyclical process which
depends on both the relationship between ADC and the scintillation light that reaches a PMT,
what we will call the linearity curves, and the
PMT resolution factors $\alpha_i$ (first introduced in Section \ref{ssec:combinePMT}), which relate
the number of photoelectrons produced by a PMT to the incoming light from an event. While the data
inherently includes the resolution factors given the observed widths of the source peaks, the detector
response model takes the resolution factors as input, and therefore the linearity curves, which compare
simulation to data, depend on the $\alpha_i$ values. An easy way to determine the $\alpha_i$ values
is to compare the observed calibrated data peak widths to the simulated widths, wherein lies the problem
that the data needs to first be calibrated. To mitigate this inherent dependence of one aspect on the other,
we simply guess at an initial set of calibration parameters (both the linearity curves and resolution factors
for each PMT), and then we analyze the resulting simulated peak widths with respect to the calibrated data peak widths.
By tweaking the $\alpha_i$
values in such a way as to create better agreement between simulation and data peak widths, the simulation data can
then be re-processed in the
detector response model, and we can calculate a new set of linearity curves. This process is repeated until the agreement is
satisfactory, as will be discussed in the upcoming sections.

One last component that was left out of the above cyclical discussion is the determination of the position
dependent response of each PMT to events that occur at different $(x,y)$ locations of the scintillator.
In the past, these position dependent light transport maps were also part of the cycle, as they were
determined from an $E_{\mathrm{recon}}$ spectrum of xenon which is dependent on the calibration, but in this analysis
we decided to make a single determination of the position dependent response. The position dependent response is thus
another ``well-defined'' input parameter of the calibration and is the topic of the next section.

In general, the entire scintillator energy calibration process proceeds as follows: 
\begin{itemize}
\item Determine the position dependent light transport maps for each PMT, $\eta_i(x,y)$.
\item Approximate the PMT resolution factors, $\alpha_i$.
\item For each source peak in a given calibration period,
  \begin{itemize}
  \item Fit the ADC peak from the data.
  \item Apply the detector response model to the simulation to emulate the data taking conditions.
  \item Fit the $E_{\mathrm{vis}}$ peak from the simulation.
  \end{itemize}
\item Relate the ADC peaks to the simulation $E_{\mathrm{vis}}$ to produce an energy calibration.
\end{itemize}
Every step beyond the first is then repeated to converge on the best approximation of the resolution factors
and linearity curves.


\subsection{Position Dependent Light Transport Maps} \label{ssec:posmaps}

As mentioned in the experimental description, each PMT is coupled to a quadrant of
the scintillator and collects the most light from this quadrant. The light collection
is therefore position dependent, and an individual PMT will receive a different
amount of light for an event of energy $E_i$ depending on where that event strikes
the scintillator. To properly map the PMT signal to energy, this position dependence
must be accounted for on a PMT-by-PMT basis. The collection of values that correct for this
dependence will often be called position maps from here on.


\subsubsection{Activated Xenon}

To map the position response of the scintillator, signals must be present across the
entire face of the detector. The $\beta$-decay spectrum
is an obvious option as the UCN fill the entire decay volume, and was used for these position maps prior to 2010,
but the event rate is low when divided into small position bins across the scintillator.
Prior to running in 2010, a method using activated xenon was developed to provide
a higher event rate with the required full fiducial coverage.

The xenon is activated by placing a small amount of natural xenon in the volume that
normally holds the $\mathrm{SD}_2$ source, freezing it, and then exposing it to the moderated
neutron flux for several minutes. This produces a plethora of radioactive isotopes
with various half-lives. The xenon is then warmed up to a gas and stored. The gas
is then released into the spectrometer during position mapping periods, and the decay products
are detected \cite{mpmThesis}. The various radioactive isotopes provide several features to fit for comparison
with one another across the entire
detector surface.

\begin{figure}[h] 
\centering
\includegraphics[scale=.5]{4-UCNACalibrations/xenonSpectrum.pdf}
\caption{Example energy spectrum of the neutron-activated xenon used for position map determination. }
\label{fig:xenonSpectrum}
\end{figure}

The spectral shape of the activated xenon changes with time due to the different half-lives
of the isotopes, but this is not a concern as the PMT will see the same shape at all positions
across the detector, just with different ADC scales due to the position dependence of the light
collection. Therefore one only needs to choose a feature of the spectrum to fit in different
positions to map the relative response.

\subsubsection{Position Maps} \label{sssec:posmaps}

The scintillator is divided into a grid of $5\times5\mathrm{~mm}^2$ squares (in the 1~T
decay trap coordinates) with one square directly in the center,
and the xenon events are collected for each of these ``pixels''. A key feature
from the spectrum is then chosen and fit in every pixel, with the position dependent response
factor in pixel $i$ for a single PMT defined as
%
\begin{equation}
  \eta_i = \frac{Q_i}{Q_0},
\end{equation}
%
where $Q_i$ is the fitted ADC value of the feature in pixel $i$ and $Q_0$ is the fitted ADC value of the
feature in the center pixel. This normalizes the position response for a PMT to the center pixel. The
position response at some position $(x,y)$, referred to as the continuous variable $\eta(x,y)$,
is then calculated via a two-dimensional Catmull-Rom
cubic interpolating spline \cite{catmull1974}. The same interpolation is used to produce the smooth
plots of the
position dependence in Figure \ref{fig:posmaps}.


\begin{figure}[hp] 
\centering
\subfloat[East Detector]{\includegraphics[page=1,scale=.55]{4-UCNACalibrations/position_map_4_5mm_endpoint.pdf}} \\
\subfloat[East Detector]{\includegraphics[page=2,scale=.55]{4-UCNACalibrations/position_map_4_5mm_endpoint.pdf}}
\caption{Typical set of position maps from a xenon position mapping period. The position maps
  remain fairly constant throughout the two run periods.}
\label{fig:posmaps}
\end{figure}

\begin{figure}[h] 
\centering
\includegraphics[scale=.55]{4-UCNACalibrations/posmapComp_4_5mm_endpoint_vs_peak.pdf}
\caption{Ratio of $\eta_{\mathrm{peak}} / \eta_{\mathrm{endpoint}}$ for the East side PMTs. The differences
  are most pronounced in areas of high and low light collection. The West side shows more
  consistency when comparing maps calculated using the different features.}
\label{fig:posmapCompare}
\end{figure}

A typical xenon energy spectrum can be seen in Figure \ref{fig:xenonSpectrum}. The two obvious features one could
fit are the peak between 100~keV and 200~keV or the 915~keV $\beta$-decay endpoint. The peak is a
superposition of several isotopes, while the endpoint comes from the
$^{135}\mathrm{Xe~}\frac{3}{2}+$ isotope. The position maps are fit in terms of pedestal and gain corrected ADC,
so unfortunately the luxury of knowing an initial guess for the feature position is not afforded as it would
be if the fit was done in the energy domain. This makes
fitting the peak more reliable upon first inspection, as the endpoint fit (done via a Kurie plot as described
in Section \ref{sssec:endpoint}) is more sensitive to the range of the fit, especially when the spectrum is not
purely a $\beta$-spectrum. There is a problem with using the peak though, as areas with low light collection
for a given PMT lose a portion of the peak below the trigger threshold. This changes the feature shape
compared to regions of higher light output, thus biasing the position map. The better choice is then the
$^{135}\mathrm{Xe~}\frac{3}{2}+$ endpoint. 

The problem with fitting the endpoint in pixels with different light collection efficiencies is illustrated by
imagining that every pixel sees the same xenon energy spectrum (as in Figure \ref{fig:xenonSpectrum}), but that
the spectrum is compressed or stretched when compared to the spectrum in the center pixel depending on where
the pixel is located. Since each pixel has the same ADC range, choosing the proper fit range becomes difficult
as it is different in every pixel. To avoid this issue, a secondary feature was derived to be used as a
seed to the endpoint fit range. This secondary feature is calculated by first fitting the peak with a Gaussian
and extracting the mean ($\mu$) and sigma ($\sigma$), and then calculating the average ADC value, $\xi$, of the spectrum
from $\mu+1.5\sigma$ and beyond. Then the endpoint fit is done over the range $(\xi,2\xi)$. The range used to
calculate $\xi$ and the range over which the endpoint were fit were determined via trial and error, and produce
consistent results across the entire detector.

It should also be noted that the position maps were calculated using both the peak and the endpoint as the key
feature, and the differences are noticeable for the East detector. This is illustrated in Figure \ref{fig:posmapCompare}, where the
ratio of the two methods is plotted.



\subsection{Linearity Curves} \label{ssec:linCurves}
After determining the position dependent light transport maps for each PMT,
the next step in in the scintillator energy calibration process is relating a gain and pedestal subtracted ADC signal
to an energy. In our case, we choose to relate the signal as seen in a single PMT to a quantity that is
proportional to the light that would have reached the PMT. This quantity is of course related to the deposited
energy by the position response factor, and thus we calibrate to $\eta_i(x,y)E_{\mathrm{vis}}$, where $E_{\mathrm{vis}}$ in this
case comes from simulation.

\subsubsection{Fitting the source peaks}
Typical single PMT energy spectra for simulation and data from the $^{137}\mathrm{Ce}$, $^{113}\mathrm{Sn}$, and $^{207}\mathrm{Bi}$ sources can be
found in Figures \ref{fig:Ce_spectra}, \ref{fig:Sn_spectra}, and \ref{fig:Bi_spectra} respectively. These spectra were produced using the final
calibration, and at this point should only be inspected for their general shape to understand the type of fit required.

An iterative peak fitting procedure like that from Section \ref{ssec:BiGain} is used to extract the final mean
of each peak to be used in the calculation of the linearity curves,
where the fit is performed five times taking the previous mean and sigma as input parameters
for the next fit. For the  $^{137}\mathrm{Ce}$ and $^{113}\mathrm{Sn}$ sources, which have a single
electron peak, a single Gaussian with an asymmetric fit range extended farther above the mean is used. Again this is to
weight the fit more heavily on the portion of the spectrum which exhibits the most Gaussian-like structure. The
lower tail of the energy distribution has a contribution from events detected that have lost energy elsewhere
in the spectrometer and is thus broadened compared to the upper tail. For the $^{207}\mathrm{Bi}$ source,
the two Gaussian peaks are fit simultaneously by
the summation of two Gaussian functions, so that the effect of one on the other is taken into account by the fit, and
the proper mean is extracted from each. This helps remove effects from possible use of inaccurate
resolution factors. %Figures \ref{fig:ADC_source_spectra} and \ref{fig:En_source_spectra} also shows the fits to each spectrum.

\subsubsection{Extracting the PMT response}

As mentioned above, we calibrate the PMT ADC response to a quantity proportional to the light incoming into
the PMT, or $\eta_i(x,y)E_{\mathrm{vis}}$, so we need to plot $\eta_i(x,y)E_{\mathrm{vis}}$ vs. ADC response.
Above we described how we fit the mean of the expected visible energy
$E_{\mathrm{vis}}$ from simulated spectra and the observed ADC response for each PMT.
The value of $\eta$ used for each source peak is determined from the average position of the events in the peak.
For each source calibration run period, all of the data points are plotted for each PMT and the response
is fit with a quadratic response function,
%
\begin{equation} \label{eq:linCurveParam}
  f(x) = C_1 + C_2x+ C_3x^2,
\end{equation}
%
where $x$ is the gain corrected ADC signal and $f$ is equivalent to $\eta(x,y)E_{\mathrm{vis}}$. The quadratic term
is constrained to $|C_3|<7.5\times10^{-5}$, which is small compared to $C_2\approx1$. The $C_1$ offset allows
for imperfections in the pedestal subtraction. If the pedestal subtraction were perfect, one would expect
$C_1\approx0$. Allowing $C_1$ to be a free parameter generally produces better Monte Carlo to Data agreement
at low energies.

\begin{figure} [p]
  \centering
  \begin{tabular}{cc}
    \subfloat[PMT East 0]{\includegraphics[scale=0.35,page=1]{4-UCNACalibrations/linCurves_SrcPeriod_3.pdf}} &
    \subfloat[PMT East 1]{\includegraphics[scale=0.35,page=2]{4-UCNACalibrations/linCurves_SrcPeriod_3.pdf}} \\
    \subfloat[PMT East 2]{\includegraphics[scale=0.35,page=3]{4-UCNACalibrations/linCurves_SrcPeriod_3.pdf}} &
    \subfloat[PMT East 3]{\includegraphics[scale=0.35,page=4]{4-UCNACalibrations/linCurves_SrcPeriod_3.pdf}}
  \end{tabular}
  \caption{East linearity response curves for a single calibration run period in 2011-2012. The residuals
    reported are calculated as $100\times(fit-data)/fit$. The error bars on the points are the error on the
    extracted mean from the fits of the data and simulation peaks, but they are smaller than the markers.}
  \label{fig:eastLinCurves}
\end{figure}

\begin{figure} [p]
  \centering
  \begin{tabular}{cc}
    \subfloat[PMT West 0]{\includegraphics[scale=0.35,page=5]{4-UCNACalibrations/linCurves_SrcPeriod_3.pdf}} &
    \subfloat[PMT West 1]{\includegraphics[scale=0.35,page=6]{4-UCNACalibrations/linCurves_SrcPeriod_3.pdf}} \\
    \subfloat[PMT West 2]{\includegraphics[scale=0.35,page=7]{4-UCNACalibrations/linCurves_SrcPeriod_3.pdf}} &
    \subfloat[PMT West 3]{\includegraphics[scale=0.35,page=8]{4-UCNACalibrations/linCurves_SrcPeriod_3.pdf}}
  \end{tabular}
  \caption{West linearity response curves for a single calibration run period in 2011-2012. The residuals
  reported are calculated as $100\times(fit-data)/fit$. The error bars on the points are the error on the
    extracted mean from the fits of the data and simulation peaks, but they are smaller than the markers.}
  \label{fig:westLinCurves}
\end{figure}

Example calibration fits for a single calibration run period in 2011-2012
are illustrated in Figures \ref{fig:eastLinCurves} and \ref{fig:westLinCurves} for the East and West
PMTs respectively. Beneath each calibration plot are the calibration residuals for each data point
compared to the fit line reported as a percent residual. The parameters of each linearity curve are
written to file to be accessed when applying the calibration to data.

\subsection{PMT Resolution Factors} \label{ssec:PMTresolution}

The PMT resolution factors are key inputs into the detector response model, making them especially
important to the calibration. They directly affect the observed width of the simulated conversion
electron peaks, and improper simulated peak widths present a possible bias in the calibration by changing
the shape of the peak being fit.

As mentioned earlier, a ``guess'' is made for the initial resolution factors for the first iteration
of the energy calibration. This guess is set to a higher resolution for each PMT than is typically observed so that
the simulation peaks do not suffer from extreme broadening which can make fitting unreliable, especially for
the double Gaussian peak of $^{207}\mathrm{Bi}$. After the calibration curves are determined, we can assess
whether or not the resolution factors are correct for each individual calibration run period.
This involves comparing the widths
of the data peaks to the widths of the simulation peaks, both in the energy domain. The widths of the simulation
peaks come directly from the fits used in calculating the linearity curves, as the simulated $E_{\mathrm{vis}}$
spectra were already fitted, but recall that the PMT ADC spectra were used in determining the lineariry curves. Thus, in order
to extract the data $E_{\mathrm{vis}}$ widths, the energy calibration for each run period must be applied
to the conversion electron events themselves (the method for applying the
calibration is briefly described in the following section). This provides $E_{\mathrm{vis}}$ spectra for each PMT as well as
an $E_{\mathrm{recon}}$ spectrum from the combination of the individual PMT energies.

\begin{figure} [h]
  \centering
  \begin{tabular}{cc}
    \subfloat[PMT East 3]{\includegraphics[scale=0.35,page=4]{4-UCNACalibrations/final_width_comp_3.pdf}} &
    \subfloat[PMT East 0]{\includegraphics[scale=0.35,page=5]{4-UCNACalibrations/final_width_comp_3.pdf}} 
  \end{tabular}
  \caption{Plots of the data source widths vs. the simulated source widths for a single PMT on each side of the
    detector used for determination of
    the proper PMT resolution factors. When the slope$\approx$1, the resolution factors are recorded to be used
    in the final calibration.}
  \label{fig:PMTwidths}
\end{figure}

With the source widths from both the data and simulation in hand, we plot the simulated widths vs. the data widths.
If the resolution factors are correct, there will be a 1:1 ratio of the peaks. On the contrary, if the $\alpha_i$ values are
incorrrect, the slope of the relationship indicates whether or not the resolution is set too high or to low.
Figure \ref{fig:PMTwidths} exemplifies the desired relationship between the simulation and data peak widths. The slope
is extracted from a linear fit, and this is carried out for each PMT in a calibration period. The data points
represent every source run in the respective source calibration period.

\begin{figure} [h]
  \centering
  \begin{tabular}{cc}
    \subfloat[PMT East 3]{\includegraphics[scale=0.35,page=1]{4-UCNACalibrations/widthsErecon_runPeriod_3.pdf}} &
    \subfloat[PMT East 0]{\includegraphics[scale=0.35,page=2]{4-UCNACalibrations/widthsErecon_runPeriod_3.pdf}} 
  \end{tabular}
  \caption{Plots of the data source widths vs. the simulated source widths for the final $E_{\mathrm{recon}}$
    source spectra on each side of the
    detector. This is a measure of how well the individual PMT resolution factors combine to produce
    agreement between the final reconstructed energy spectra. The slopes of nearly unity indicate the
    final detector resolutions were very good for this calibration period.}
  \label{fig:EvisWidths}
\end{figure}

A slope of $\sim1$ is not achieved after one iteration of the calibration, but rather several. After extracting
the slopes, $m_i$, for each PMT following an intermediate iteration of the calibration, a new set of of resolution
factors are determined in the same manner as in the previous analysis, namely from
%
\begin{equation}
  \alpha_i = m^2 \alpha_i',
\end{equation}
%
where $\alpha_i'$ are the new estimates for the resolution factors.

Once the slopes approach unity for all PMTs, the resolution factors are recorded and the linearity curves
are calculated as the last step so that they use the best $\alpha_i$ values. This completes the calibration,
and at this point final $E_{\mathrm{recon}}$ spectra for each source peak exist. The widths of these
spectra, which are the culmination of all of the PMT information combined, provide a idea as to how well the
final combined widths agree between simulation and data. Figure \ref{fig:EvisWidths} shows this
agreement for a single run period.

%The slope is squared in this case
%because
%\begin{equation*}
%  m = \Big\langle \frac{\mathrm{Data Width}}{\mathrm{Simulated Width}} \Big\rangle,
%\end{equation*}
%and the resolution factors are proportional to the square of the width.

\subsection{Applying the calibration} \label{ssec:appCal}

Application of the calibration to an electron event of any kind requires first reading from the collection
of calibration variables for each PMT: the position dependent response (interpolated to the exact position of the event),
the resolution parameters $\alpha_i$, and the linearity curve parameters $C_1$, $C_2$, and $C_3$.
Then, by inserting the gain corrected ADC
response of each PMT into Equation \ref{eq:linCurveParam} with the proper parameters, the linearity curve returns
an estimate of the scintillation light as viewed from a given PMT. To reach an estimate of $E_{\mathrm{vis}}$
from the returned value $f$ (remember that $f = \eta(x,y)E_{\mathrm{vis}}$), one simply divides by the
position map value for that event. With a value for $E_{\mathrm{vis}}$ from each PMT, a final
$E_{\mathrm{recon}}$ value is constructed using the methods of Section \ref{ssec:combinePMT}, which rely on $\alpha_i$.
Examples of the individual calibrated PMT responses for simulation and data
can be found in Figures \ref{fig:Ce_spectra}, \ref{fig:Sn_spectra}, and \ref{fig:Bi_spectra}.
These spectra came from a single calibration run that included all three sources within the fiducial volume at the same time.

\subsection{Monte Carlo to Data Agreement}
A final measure of the agreement between Monte Carlo and data is best observed by comparing the $E_{\mathrm{recon}}$
spectra of the data (after applying the calibration) to the $E_{\mathrm{recon}}$ spectra of the simulation (with
the full detector response model applied). Figure \ref{fig:EreconSpectra} shows the typical agreement for each source.
The agreement of all of the source peaks is further analyzed when determining the energy uncertainty contribution
to the final asymmetry in Section \ref{ssec:energyRecon}.

\begin{sidewaysfigure}
  \centering
  \includegraphics[scale=1,page=2]{4-UCNACalibrations/run_17523.pdf}
  \caption{Example spectra from each PMT for simulation (red dashed line) and data (blue line) for $^{137}\mathrm{Ce}$ conversion electron source
    after application of the calibration. This is for a random run that included all three sources within the fiducial
    volume at the same time.}
  \label{fig:Ce_spectra}
\end{sidewaysfigure}

\begin{sidewaysfigure}
  \centering
  \includegraphics[scale=1,page=3]{4-UCNACalibrations/run_17523.pdf}
  \caption{Example spectra from each PMT for simulation (red dashed line) and data (blue line) for $^{113}\mathrm{Sn}$ conversion electron source
    after application of the calibration. This is for a random run that included all three sources within the fiducial
    volume at the same time.}
  \label{fig:Sn_spectra}
\end{sidewaysfigure}

\begin{sidewaysfigure}
  \centering
  \includegraphics[scale=1,page=4]{4-UCNACalibrations/run_17523.pdf}
  \caption{Example spectra from each PMT for simulation (red dashed line) and data (blue line) for $^{207}\mathrm{Bi}$ conversion electron source
    after application of the calibration. This is for a random run that included all three sources within the fiducial
    volume at the same time.}
  \label{fig:Bi_spectra}
\end{sidewaysfigure}

\begin{sidewaysfigure}
  \centering
  \includegraphics[scale=1,page=1]{4-UCNACalibrations/run_17523.pdf}
  \caption{Example $E_{\mathrm{recon}}$ spectra from combined PMT response
    for simulation (red dashed line) and data (blue line) for all conversion electron sources
    after application of the calibration. This plot is from the same run as the plots of the previous individual
    PMT spectra, showing how the spectra combine to produce a single final energy estimate.}
  \label{fig:Erecon_spectra}
\end{sidewaysfigure}



%------------------------------------------------------------------------

\section{Wirechamber Energy Calibration}

\subsection{Method}

The goal of the wirechamber energy calibration is to provide an estimate of
the energy deposited within the MWPC, $E_{\mathrm{MWPC}}$, depending on the
response of the wirechamber. The available information from the wirechamber includes the
summed anode signal, which is proportional to the total amount of ionization
within the MWPC gas and thus the total energy deposited, and the reconstructed position from the cathode signals.
Also available is the reconstructed initial energy of all events, $E_{\mathrm{recon}}$, as a result of the
previously described scintillator energy calibration. This is important as higher energy events deposit
less energy in the wirechamber than lower energy events, and thus subsets of events with
different $E_{\mathrm{recon}}$ will provide different distributions of $E_{\mathrm{MWPC}}$.
With this in mind, the general idea for calibrating the wirechamber is as follows:

\begin{itemize}
\item Map out the position dependence of the anode signals in a similar manner to the
  PMT light transport maps.
\item Separate data into subsets based on $E_{\mathrm{recon}}$
  to provide several different distributions of wirechamber signals (anode for data and $E_{\mathrm{MWPC}}$
  for simulation).
\item For each $E_{\mathrm{recon}}$ subset, fit the wirechamber ADC signals for the data.
\item For each $E_{\mathrm{recon}}$ subset, fit the simulated wirechamber deposited energy, $E_{\mathrm{MWPC}}$.
\item Plot the extracted ADC values from the data vs. the extracted $E_{\mathrm{MWPC}}$ from the simulation to
  determine the parameterization from anode signal to wirechamber deposited energy.
\end{itemize}

The energy calibration of the wirechamber is carried out for every individual $\beta$-decay run,
rather than using the periodic
source calibration runs as was done for the scintillator energy calibration. This is advantageous as
it automatically corrects for any changes in the anode gain on the run-by-run time scale. 


\subsection{Position dependence of anode signal}

While the wirechamber is constructed to be as homogeneous as possible, a residual
position dependence to the anode signal still exists. The cause of such position dependence is undetermined,
but it only affects the energy calibration of the wirechamber, which is
a non-essential component of the analysis. Also, by mapping the position dependence in a manner
similar to the PMT position maps (Section \ref{sssec:posmaps}), we can remove any effects altogether.

\begin{figure}[h]
  \centering
  \includegraphics[scale=0.5,page=1]{4-UCNACalibrations/MWPCPositionMapSpectrum.pdf}
  \caption{Typical wirechamber anode signal for a xenon calibration run in a
    single pixel. The fit function is a TMath::Landau function from the ROOT Data Analysis
    Framework, and the extracted value is the MPV (most probable value) from the
    distribution.}
  \label{fig:xenonMWPCsignal}
\end{figure}

\begin{figure}[h]
  \centering
  \begin{tabular}{cc}
    \subfloat[East MWPC]{\includegraphics[scale=0.3,page=1]{4-UCNACalibrations/MWPC_posmap.pdf}}  &
  \subfloat[West MWPC]{\includegraphics[scale=0.3,page=2]{4-UCNACalibrations/MWPC_posmap.pdf}}
  \end{tabular}
  \caption{Example position dependent anode response for each MWPC. These MWPC position maps were
    created using electrons with $300\mathrm{~keV}<E_{\mathrm{recon}}<350\mathrm{~keV}$.}
  \label{fig:mwpcPosMap}
\end{figure}

The dedicated xenon calibration runs were used to determine the MWPC position maps.
The wirechamber anode signal was histogrammed in 5~mm~$\times$~5~mm position bins (pixels) across the detector,
giving distributions like the one seen in Figure \ref{fig:xenonMWPCsignal}. Only Type 0 events were used
for the position dependence of the wirechamber, as these only pass through the wirechamber once and are a better
representation of the amount of energy deposition as a function of initial event energy. The distribution
in each pixel is fit with a TMath::Landau function from the ROOT Data Analysis Framework.
The most probable value of
the distribution is returned from the fit for every pixel, and then the position dependent response factor is formed
via
%
\begin{equation}
  \eta^{\mathrm{MWPC}}_i = \frac{\mathrm{MPV}_i}{\mathrm{MPV}_0},
\end{equation}
where $\mathrm{MPV}_0$ is the most probably value of the center pixel.
The continuous values plotted in Figure \ref{fig:mwpcPosMap} are determined as they were in
Section \ref{sssec:posmaps}. The position dependent response factor is divided out of the anode signal
prior to determining the wirechamber calibration, and thus also prior to applying the wirechamber
calibration to the data.

One should note that the MWPC position maps can be determined for different initial energies of the
electrons, since the scintillator calibration has already been applied. The position response plots
in Figure \ref{fig:mwpcPosMap} were made using electrons with $300\mathrm{~keV}<E_{\mathrm{recon}}<350\mathrm{~keV}$,
although the position maps are stable for any 50~keV window chosen. For ease of application, the MWPC position
maps from the above energy range were used for electrons of all initial energies.


\subsection{Relating Anode Signal to $E_{\mathrm{MWPC}}$}

As mentioned above, the wirechamber energy calibration is carried out on a run-by-run basis, so
the discussion that follows applies to every individual $\beta$-decay run.
After dividing out the MWPC position dependent response from the anode signals of all electron events,
we can attempt to relate the anode response to the expected wirechamber response from simulation (again
we only use Type 0 events for the energy calibration). 
To do this, we first separate the electron events by their $E_{\mathrm{recon}}$ into 50~keV energy bins beginning at 100~keV and ending at
800~keV, giving twelve separate collections of electron events. The lower bound is chosen to avoid any possible
effects from differences between the simulated and actual scintillator thresholds. The upper bound is just beyond
the endpoint energy of the $\beta$-decay electrons.

\begin{figure}[h]
  \centering
  \begin{tabular}{cc}
    \subfloat[Simulated MWPC response]{\includegraphics[scale=0.35,page=1]{4-UCNACalibrations/MWPCcalSim_300-400keV.pdf}}  &
  \subfloat[Actual MWPC response]{\includegraphics[scale=0.35,page=1]{4-UCNACalibrations/MWPCcalData_300-400keV.pdf}}
  \end{tabular}
  \caption{TMath::Landau fit of both simulation and data for electron events which deposited between 300-400 keV
    in the west scintillator. The fitted MPV (p1 in the statistics box) from each plot make up a single data point
    in the wirechamber energy calibration plot below (Figure \ref{fig:mwpcCal}).}
  \label{fig:landauFits}
\end{figure}

\begin{figure}[h]
  \centering
  \begin{tabular}{cc}
    \subfloat[East MWPC]{\includegraphics[scale=0.3,page=1]{4-UCNACalibrations/MWPC_cal_17126.pdf}}  &
  \subfloat[West MWPC]{\includegraphics[scale=0.3,page=2]{4-UCNACalibrations/MWPC_cal_17126.pdf}}
  \end{tabular}
  \caption{Example MWPC calibration curve for a single run in 2011-2012. A calibration like this one
  is carried out for every $\beta$-decay run.}
  \label{fig:mwpcCal}
\end{figure}

The twelve electron energy groupings provide twelve different distributions
to be fit with a Landau distribution, and the MPV of each should be different. This is a result
of the electrons of interest
lying in an energy range such that interactions with the wirechamber fill gas produce more energy loss
per unit length for less energetic electrons, or in other words the lower energy electrons fall below
the minimum ionizing particle energy for the fill gas. A nice summary of particle interactions with matter
can be found in \cite{pdg}, but the important takeaway from this is that the twelve different initial energy
groupings yield twelve different energy deposition distributions in the wirechamber, which provides
twelve data points to use when comparing simulation with data.

For the simulation, the MPV is extracted in the energy domain, where the energy is the
amount of energy deposited within the wirechamber gas. For the data, the MPV is an ADC value.
Examples of each fit for a single $E_{\mathrm{recon}}$ grouping can be seen in Figure \ref{fig:landauFits}.
If we
have modeled the wirechamber correctly and as long as the MWPC anode signal is proportional to
the total ionization created in the wirechamber, then we expect a linear relationship between the
simulated energy depostion and the anode signal for each of the $E_{\mathrm{recon}}$ groupings. By plotting
the MPVs from data and simulation and fitting with a straight line, we define the conversion
from (position corrected) MWPC anode signal to energy deposited within the wirechamber. An example
calibration is demonstrated in Figure \ref{fig:mwpcCal}. The calibration fit is of the form
%
\begin{equation} \label{eq:pureLin}
  E_{\mathrm{MWPC}}(\mathrm{ADC}) = C_1 + C_2\cdot \mathrm{ADC},
\end{equation}
and the fit is extrapolated to the origin in a continuous manner using
\begin{equation}
  E_{\mathrm{MWPC}} = C_3 \cdot \mathrm{ADC}^{C_4}
\end{equation}
with the parameters $C_3$ and $C_4$ determined by ensuring the two equations are equal at the
transition point along with their first derivatives. The transition point is defined as 50~ADC channels
below the lowest data point.

\begin{figure}[h]
  \centering
  \begin{tabular}{cc}
    \subfloat[East MWPC]{\includegraphics[scale=0.35,page=1]{4-UCNACalibrations/mwpc_energy_comp.pdf}}  &
  \subfloat[West MWPC]{\includegraphics[scale=0.35,page=2]{4-UCNACalibrations/mwpc_energy_comp.pdf}}
  \end{tabular}
  \caption{Simulation vs. Data after application of the wirechamber calibration
  for a single run in 2011-2012. This plot includes only Type 0 events.}
  \label{fig:simVsDataMWPC}
\end{figure}

Including the $C_1$ parameter ($y-$intercept) allows for
imperfections in the MWPC anode pedestal subtraction. The extrapolation to the origin produces better agreement
with simulation, as data events with very low energy deposition in the wirechamber could otherwise be
assigned negative energies if only Equation \ref{eq:pureLin} is used and the $y-$intercept is negative.
Comparison between simulation and data of the final calibrated energy deposition for a single
run is shown in Figure \ref{fig:simVsDataMWPC}, where we see very nice agreement for both detectors.




\subsection{Backscattering Separation for Type 2/3 Events} \label{sssec:backscSep}
The primary reason for performing the wirechamber energy calibration is to allow
for comparison with simulation in order to separate the Type 2 and Type 3 backscattering events. Recall from Section
\ref{sec:backscattering} that the Type 2/3 backscattering events are
indistinguishable using only detector trigger logic, but that their energy
deposition in the primary wirechamber is different due to Type 2 events passing
through the primary MWPC once and Type 3 passing through twice. Thus there should exist an energy cut that most effectively
distinguishes the Type 2 from the Type 3 events. 

\begin{figure}[h]
  \centering
  \begin{tabular}{cc}
    \subfloat[East]{\includegraphics[scale=0.35,page=1]{4-UCNACalibrations/Thesis_sepPlot_2011-2012.pdf}}  &
  \subfloat[West]{\includegraphics[scale=0.35,page=2]{4-UCNACalibrations/Thesis_sepPlot_2011-2012.pdf}}
  \end{tabular}
  \caption{2011-2012 simulated wirechamber energy deposition of Type 2 and Type 3 events with
    $200\mathrm{~keV}<E_{\mathrm{recon}}<300\mathrm{~keV}$. The vertical dashed line indicates the optimal
    cut on the wirechamber energy deposition to separate the two otherwise identical backscattering
    events. The position of this cut varies with $E_{\mathrm{recon}}$. The distributions for the other
    geometries are similar.}
  \label{fig:two&threeDist}
\end{figure}

\subsubsection{Monte Carlo Study}
Monte Carlo studies provide the
means for applying the most efficient cuts on the energy deposition in the wirechamber
to separate the Type 2/3 backscattering events due to the fact that the true intitial direction of each event
is known, and thus so is the true type of the event. With this knowledge, one can analyze the fraction of events
that are assigned the proper side as a function of different separation cuts. When we say we are applying cuts to
separate the two event types, we mean choosing a wirechamber deposited energy where
any events above the cut value will be assigned one event type (and thus a certain side), while an event below
the chosen cut will be identified as the other event type (and the opposite side). An example of the
wirechamber energy deposition for the two types of events can be seen in
Figure \ref{fig:two&threeDist}, with the location of the optimal cut indicated by the
vertical line. Note that this plot uses only electron events satisfying
$200\mathrm{~keV}<E_{\mathrm{recon}}<300\mathrm{~keV}$.
From this plot we see that an event with $E_{\mathrm{MWPC}}$
below the cut would be identified as a Type 2 event, while an event lying above the cut
would be a Type 3 event. One can also see from this plot that the separation is not perfect,
as the two event type distributions overlap.

\begin{figure}[h]
  \centering
  \includegraphics[scale=0.55,page=1]{4-UCNACalibrations/properly_identified_2011_200-300keV.pdf}
  \caption{Fraction of properly identified events where
    $200\mathrm{~keV}<E_{\mathrm{recon}}<300\mathrm{~keV}$ as a function of the separation cut.
    The polynomial fit to the maximum is indicated by the solid red line, from which the location
    of the most effective separation cut is extacted.}
  \label{fig:propID}
\end{figure}

The position of the optimal cut in Figure \ref{fig:two&threeDist}, while arguably obvious
to the keen observer, is determined quantitatively for eight different groups of $E_{\mathrm{recon}}$ events in
100~keV increments from $0-800$~keV. For each $E_{\mathrm{recon}}$ group,
the cut is incremented from 0~keV to 20~keV in 1~keV steps, and then the fraction
of properly identified events given each different cut is calculated. The fraction of properly identified events for
each value of the separation cut is plotted as a function of the position of the cut as shown
in Figure \ref{fig:propID}. The goal is then to extract the maximum of this distribution, which
indicates the optimal cut. To determine the maximum, the peak is fit with a second-order polynomial
of the form
%
\begin{equation}
  f(x) = C_1 + C_2\big(x-C_3\big) + C_4\big(x-C_3\big)^2,
\end{equation}
%
and then the maximum is extracted using the derivative of the fit function. The location
of this maximum is the optimal cut for separating Type 2/3 events within the $E_{\mathrm{recon}}$  
range being analyzed.



\subsubsection{Determining a continuous separation cut}

After fitting the location of the optimal cuts of the eight $E_{\mathrm{recon}}$ groups, a
continuous wirechamber cut can be extracted by plotting the location of the cuts vs. the
midpoint of the $E_{\mathrm{recon}}$ group, i.e. for the electron events with
$200\mathrm{~keV}<E_{\mathrm{recon}}<300\mathrm{~keV}$ we use 250~keV. Such plots are shown in
Figure \ref{fig:separationFit} for the 2011-2012
and 2012-2013 data sets. Due to the consistency between the two detectors, a fit to the
average of the data points from each side is carried out, with the fit taking the
functional form 
%
\begin{equation}
  f(x) = C_1 + C_2e^{-C_3x}.
\end{equation}
Then, with the fit parameters saved for each geometry, a continuously varying separation
cut can be calculated depending on the $E_{\mathrm{recon}}$ of an indivudual event.

\begin{figure}[h]
  \centering
  \begin{tabular}{cc}
    \subfloat[2011-2012]{\includegraphics[scale=0.38,page=1]{4-UCNACalibrations/EreconVsCut_2011-2012.pdf}}  &
    \subfloat[2012-2013]{\includegraphics[scale=0.38,page=1]{4-UCNACalibrations/EreconVsCut_2012-2013.pdf}}
  \end{tabular}
  \caption{Plot of optimal cuts for each detector from 2011-2012 and 2012-2013. The eight data points for
    each detector are a result of dividing the simulation data into 100~keV $E_{\mathrm{recon}}$ groups from
    $0-800$~keV. The average of the two detector sides are fit to determine a continuous function
    describing the optimal Type 2/3 separation cuts.}
  \label{fig:separationFit}
\end{figure}

\begin{figure}[h]
  \centering
  \begin{tabular}{cc}
    \subfloat[2011-2012]{\includegraphics[scale=0.35,page=1]{4-UCNACalibrations/backscSepEfficiency_Octet0.pdf}} &
    \subfloat[2012-2013]{\includegraphics[scale=0.35,page=1]{4-UCNACalibrations/backscSepEfficiency_Octet95.pdf}} \\
  \end{tabular}
  \subfloat[2012-2013 Isobutane in MWPC]{\includegraphics[scale=0.35,page=1]{4-UCNACalibrations/backscSepEfficiency_Octet75.pdf}} 
  \caption{Fraction of properly identified events when the Type 2/3
    separation is applied compared to when it is not applied. The results are shown for a single octet from all possible
    geometries. The shaded error band is purely statistical.}
  \label{fig:sepResults}
\end{figure}

\subsubsection{Results}

By applying the continuous cut to the processed simulation, we can assess the
effectiveness of the Type 2/3 separation procedure. 
Figure \ref{fig:sepResults} shows the fraction of the events that are assigned to the correct
side of the detector as a function of reconstructed initial energy. The plot shows a vast improvement in
properly assigning the initial direction of the events when using the Type 2/3 separation
highlighted above (solid line), rather than using no separation at all (dashed line).
Upon integrating across all energies, the Type 2/3 separation method yields $>81\%$ proper identification
of the Type 2/3 events,
compared to the $\sim53\%$
when no separation is applied. This improvement is very important if all event types
are to be included in the final analysis, as it drastically reduces the
Monte Carlo systematic corrections that must be applied to the backscattering events.
This will be shown in detail in the following chapter.




%----------------------------------------------------------






