\chapter{UCNA Calibrations}
\label{ch:UCNA_Calibrations}
%%%%%%%%%%%%%%%%%%%%%%%%%%%%%%%%%%%%%%%%%%%%%%%%%%%%%%%%%%%%%%%%%%%%%%%%%%%%%%%
%%%%%%%%%%%%%%%%%%%%%%%%%%%%%%%%%%%%%%%%%%%%%%%%%%%%%%%%%%%%%%%%%%%%%%%%%%%%%%%
%%%%%%%%%%%%%%%%%%%%%%%%%%%%%%%%%%%%%%%%%%%%%%%%%%%%%%%%%%%%%%%%%%%%%%%%%%%%%%%

Detector calibrations are a beautiful combination
of simulation and data manipulation which allows one to extract the energy of
an event based solely on some electronic signal. Imagine a baseball pitcher
throwing a fastball into a sheet, and the observer behind the sheet must
determine the velocity of the ball from only seeing the impression the pitch
made on the sheet. This is the task every nuclear physics experiment is faced
with, only the baseball is a particle and the sheet is our detector system.
Below we focus on the energy calibration of our apparatus.

%--------------------------------------------------------------

\section{Simulation}
\label{sec:Simulation}

The simulations and application of the
detector response model provide a suitable starting
point for discussing the analysis, as the calibration of the data hinges
strongly on the simulation results. There is a notable dependence of the
response model on data, but this will be addressed as needed
when describing the calibration.

The simulation work for this thesis was completed
using the Geant4 simulation software package. The geometry of the
apparatus was duplicated
to the best precision possible and benchmarked against data after
the detector response model was applied. The structure
of the simulation code did not change from those in the previous analysis
as described in \cite{mpmThesis}, although minor changes were
made to the geometry to reflect real adjustments to the experimental
apparatus. 

\subsection{Input}
\subsubsection{Conversion Electron Sources}
make sure to show figures or a table with the conversion lines
\subsubsection{Activated Xenon}
\subsubsection{$\beta$-decay Electrons} \label{sssec:betaSim}

\subsection{Output}
The Geant4 simulation provides adequate trajectory tracking along with
the energy deposition along these tracks. Obviously we have access to far
more information in the simulation than we do with actual data, so the first
task is to construct observables related to those coming from the detectors.

\subsubsection{Energy Deposition}
By tallying the energy deposited along an entire track within some subset
of the geometry, one can reconstruct the energy deposited anywhere within
the SCS. The areas of highest interest for the sake of analysis are the
wirechambers and the scintillators. The energy deposited in the scintillator
is of utmost importance, as the analysis can be completed without knowledge
of the energy deposition in the wirechamber, where determination of energy
deposition is only used for type II/III separation. From here on,
the energy lost in the scintillator, as determined from simulation, will be
referred to as $E_{dep}$. This is the maximum energy which could be detectable
for an event in the data if no inconspicuous energy losses existed.

\subsubsection{Quenched Energy}
Show Birk's law. Talk about MPM and jianglai's analysis to determine
quenching parameters. Then talk about the modification using BradF's note
and show the shift to the functional form. Also show a figure which shows
the quenched energy vs Edep.

\subsubsection{Position of Detector hits}
Talk briefly of importance of position dependence, and maybe address
the way the position for the wirechamber and scintillator are calculated.
I believe this was a weighted average of the track positions where the
weights are given by the energy deposition.

\subsection{Simulating Detector Response}
While all pertinent components of the experimental apparatus were included
in the simulation, the detector response is not inherent within our
output. You may notice from (figure with quenched energy) that this
doesn't resemble the gaussian peaks normally seen in a detector signal.
The gaussian-like peaks arise from finite resolution effects and
must be put in defacto using parameters calculated
from real data. The better we understand the conversion from a pure simulation
energy output to a response which mimics that from our actual detectors, the
more credence we can put in our calibration. For the time being, the model will
be introduced with parameter determination addressed as it arises. 

\subsubsection{PMT Response Model}

Recall from section \ref{sssec:pmtModel} that we can relate a digital signal
in a PMT, $ADC_i$, to the energy deposited in the scintillator
using Equation \ref{eq:EnergyResponse}.
By reverse engineering this process, we can use a purely simulated $E_Q$ and
simulate an ADC channel, so that we can process the simulated data in
exactly the same way we process real data. The following equation comes directly
from \ref{eq:EnergyResponse}:

\begin{equation} \label{eq:pmtResponse}
ADC_i = f_i^{-1}\left(\eta_i(x,y) \cdot E_{Q} \right)/g_i(t) + p_i(t) + \delta p_i(t) 
\end{equation}

This relationship is somewhat incomplete though, as it does not address the
stochastic nature of the PMTs themselves. Let's rewrite ************To properly incorporate this into our
model we must utilize the fact that a PMT consists of several dynodes, each of which
introduces stochasticity into the final voltage read out. 

\subsubsection{$E_{true}$ reconstruction}
%----------------------------------------------------------

\subsection{Wirechamber Calibration}

Blah Blah

%----------------------------------------------------------

\section{Energy Calibration}
\subsection{Electron Conversion Sources}
\subsection{Linearity Curves}
\subsection{PMT Resolution}








