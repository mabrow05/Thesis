\chapter{UCNA Results}
\label{ch:UCNA_Results}
%%%%%%%%%%%%%%%%%%%%%%%%%%%%%%%%%%%%%%%%%%%%%%%%%%%%%%%%%%%%%%%%%%%%%%%%%%%%%%%
%%%%%%%%%%%%%%%%%%%%%%%%%%%%%%%%%%%%%%%%%%%%%%%%%%%%%%%%%%%%%%%%%%%%%%%%%%%%%%%
%%%%%%%%%%%%%%%%%%%%%%%%%%%%%%%%%%%%%%%%%%%%%%%%%%%%%%%%%%%%%%%%%%%%%%%%%%%%%%%


\section{Constructing an Asymmetry}

\subsection{Decay Rate Model}
Nominally the decay rate for polarized neutrons is expressed in terms of the spin
vector and all possible correlations to the products. For simplicity,
we begin by writing the decay rate only as a function of the
electron asymmetry: 

\begin{equation} \label{eq:simpleRate}
\Gamma\left(E,\Omega\right)=C \cdot S(E) \cdot \left[ 1+A(E)\beta\cos\theta \right],
\end{equation}

\noindent where $S(E)$ is the Standard Model decay rate for unpolarized beta decay, $C$
is a constant which encompasses anything not included in $S(E)$ and serves the 
purpose of absorbing other constants, $A(E)$ is the 
energy dependent electron asymmetry parameter, $\beta$ is the normal $v/c$, and 
$\theta$ is the angle between the spin of the neutron and the electron momentum.

We can also introduce a polarization, giving

\begin{equation}
\Gamma\left(E,\Omega\right)=C \cdot S(E) \cdot \epsilon \cdot \left[ 1+ P\cdot A(E)\beta\cos\theta \right],
\end{equation}

\noindent where $\epsilon$ is the loading efficiency of neutrons in this spin state which I haven't absorbed
into the constant for reasons to be seen, and 
$P$ is the polarization, i.e. a number between 0 and 1 (very close to 1
in our case).

At this point, assumptions can be made regarding the experimental setup utilized
to further express the decay rate in terms of detector rates. Due to the polarization
being aligned with the 1 T magnetic field in the decay trap, electrons emitted with
a momentum component along the spin will spiral towards one detector, while electrons
with a component of momentum opposite the spin will be detected in the opposite 
detector. If we fix the z-axis in our decay rate using the static position of our
apparatus, we can call the detector at $\theta=0$ detector
1 and the opposite detector 2. 

Now, all electrons with $0 < \theta < \pi/2$ will head towards detector 1 and those 
with $\pi/2 < \theta < \pi$ will be directed towards detector 2. The solid angle can
then be integrated out for each detector by integrating $\phi$ from $(0,2\pi)$ and $\cos\theta$ over the intervals given, where the integral of $\cos\theta$ over detectors
1 and 2 yields $\pm1/2$ respectively: 

\begin{equation} 
\Gamma_{1,2}\left(E\right)= 2\pi \cdot C \cdot S(E) \cdot \epsilon \cdot \eta_{1,2}(E) \left[ 1+ P\cdot A(E)\beta \left(\pm_{1,2}\frac{1}{2}\right) \right],
\end{equation} 

\begin{equation} 
\Gamma_{1,2}\left(E\right)=C' \cdot S(E) \cdot \epsilon \cdot \eta_{1,2}(E) \left[ 1 \pm_{1,2} P\cdot A(E) \frac{\beta}{2} \right],
\end{equation} 

\noindent with $\eta_{1,2}(E)$ signifying energy dependent electron detection efficiencies for detectors 1 and 2. 

Thus far we have assumed a fixed polarization, but in UCNA we have the ability to 
flip the spin of the neutrons prior to loading. We call these states flipper on ($+$) 
and flipper off ($-$), denoted by $P_{\pm}$. The flipper on state introduces a negative in front of the polarization efficiency:

\begin{equation}
\Gamma_{1,2}^{\pm}\left(E\right)=C' \cdot S(E) \cdot \epsilon_{\pm} \cdot \eta_{1,2} \left[ 1 \pm_{1,2} 
\left(\mp P_{\pm}\right) \cdot A(E) \frac{\beta}{2} \right],
\end{equation} 

\noindent where $\epsilon_{\pm}$ accounts for differing loading efficiencies for 
different spin states.

If we now assume that we are equally as good at polarizing in either spin state, we
can say $P_{+}=P_{-}=P$:

\begin{equation}
\Gamma_{1,2}^{\pm}\left(E\right)=C' \cdot S(E) \cdot \epsilon_{\pm} \cdot \eta_{1,2} \left[ 1 \pm_{1,2} 
\left(\mp P \right) \cdot A(E) \frac{\beta}{2} \right].
\end{equation}


\subsection{Super-Ratio}

A simple asymmetry over some energy bin for a single 
polarization state can be defined as

\begin{equation} 
A_{\mathrm{simple}} = \frac{\Gamma_1 - \Gamma_2}{\Gamma_1 + \Gamma_2}, 
\end{equation}

\noindent and assuming that the efficiencies $\eta_1$ and $\eta_2$ are the same one
finds

\begin{equation} \label{eq:Asimple}
A_{\mathrm{simple}} = P \cdot A(E) \cdot \frac{\beta}{2},
\end{equation}

\noindent where $\beta$ is calculated as either the velocity associated with the energy 
at the center of the bin or the average energy of the bin.   

In reality we can't assume with certainty that the detectors are identical, but we can 
utilize the fact that we can flip the spins to cancel differing efficiencies. If we 
take two runs with opposite polarizations, we can define the super-ratio asymmetry in 
some energy range as:

\begin{equation}
A_{\mathrm{SR}} = \frac{1-\sqrt{R}}{1+\sqrt{R}} ,
\end{equation}
\noindent where 
\begin{equation} \label{eq:finalA_SR}
 R = \frac{\Gamma_{1}^+ \cdot \Gamma_{2}^-}{\Gamma_{1}^- \cdot \Gamma_{2}^+}.
\end{equation}

Some algebra yields an identical expression to equation \ref{eq:Asimple},

\begin{equation*}
R = \frac{ \left[ C'  S(E)  \epsilon_{+}  \eta_{1} \left[ 1 + 
      \left(- P \right)  A(E) \frac{\beta}{2} \right] \right] 
\left[ C'  S(E)  \epsilon_{-}  \eta_{2} \left[ 1 - 
    \left(+ P \right)  A(E) \frac{\beta}{2} \right] \right] }
{ \left[ C'  S(E)  \epsilon_{-}  \eta_{1} \left[ 1 + 
      \left(+ P \right)  A(E) \frac{\beta}{2} \right] \right] 
\left[ C'  S(E)  \epsilon_{+}  \eta_{2} \left[ 1 - 
    \left(- P \right)  A(E) \frac{\beta}{2} \right] \right] },
\end{equation*}

\begin{equation*}
R = \frac{ \left[ 1 + \left(- P \right)  A(E) \frac{\beta}{2} \right] 
\left[ 1 - \left(+ P \right)  A(E) \frac{\beta}{2} \right]  }
{  \left[ 1 +  \left(+ P \right)  A(E) \frac{\beta}{2}  \right] 
 \left[ 1 -  \left(- P \right)  A(E) \frac{\beta}{2} \right]},
\end{equation*}

\begin{equation*}
R = \frac{ \left[ 1 -  P  A(E) \frac{\beta}{2} \right]^2 }
{  \left[ 1 + P  A(E) \frac{\beta}{2}  \right]^2 },
\end{equation*}

\noindent and plugging into $A_{\mathrm{SR}}$

\begin{equation*}
  A_{\mathrm{SR}} = \frac{1-\frac{ \left[ 1 -  P  A(E) \frac{\beta}{2} \right] }
{  \left[ 1 + P  A(E) \frac{\beta}{2}  \right] } }
  {1+\frac{ \left[ 1 -  P  A(E) \frac{\beta}{2} \right] }
{  \left[ 1 + P  A(E) \frac{\beta}{2}  \right] }},
\end{equation*}

\begin{equation*}
A_{\mathrm{SR}} = \frac{2 P A(E) \frac{\beta}{2}}{2},
\end{equation*}


\begin{equation}
A_{\mathrm{SR}} = P \cdot A(E) \cdot \frac{\beta}{2}.
\end{equation}

\subsection{Extracting $A_0$}

The quantity of interest isn't the raw measured asymmetry, or even $A(E)$, but rather $A_0$, which is also directly proportional to $\lambda=g_A/g_V$, yielding a direct 
measurement of the axial vector coupling constant if the CVC hypothesis is assumed. 

$A_0$ manifests itself in our measured asymmetry as:

\begin{equation}
A(E) = A_0 \cdot \left( 1 + \Delta_{\mathrm{Th}}(E) \right),
\end{equation}

\noindent where $\Delta_{\mathrm{Th}}(E)$ is the energy dependent systematic shift in the
asymmetry due to theory corrections like finite mass effects and radiative 
corrections to name a few. The energy dependence will be dropped for neatness from here on.

If one were to assume we could perfectly measure our super-ratio asymmetry, free of
systematic effects with only statistical uncertainty, we could write down from 
equation \ref{eq:finalA_SR}:

\begin{equation}
A_{\mathrm{SR}}^{\mathrm{pure}} = P \cdot A(E) \cdot \frac{\beta}{2}, 
\end{equation}

\begin{equation}
A_{\mathrm{SR}}^{\mathrm{pure}} = P \cdot A_0 \cdot \left(1 + \Delta_{\mathrm{Th}} \right) \cdot \frac{\beta}{2}, 
\end{equation}

\noindent and finally

\begin{equation}
A_0 = \frac{A_{\mathrm{SR}}^{\mathrm{pure}}}{P \cdot \left( 1+\Delta_{\mathrm{Th}} \right) \cdot \beta/2}. 
\end{equation}

Obviously our measurement of the super-ratio is not free of systematic errors, so 
this equation alone provides us with little, but we can assume that, providing our
simulation and detector response model has been properly benchmarked against data (both conversion electron and 
$\beta$-decay), we can characterize the systematic effects on a bin-by-bin basis
by comparing pure simulation super-ratios to those from processed simulation data. The 
processed simulation data refers to data which has been run through the detector 
response model and thus treated like ``real" data. If this is the case, then we can say:

\begin{equation}
A_{\mathrm{SR}}^{\mathrm{pure}} = A_{\mathrm{SR}}^{\mathrm{Exp}} \cdot \left(1+\Delta_{\mathrm{Exp}}(E) \right).
\end{equation}

At this point it's straighforward to write down $A_0$ as a function of quantities 
which are either measured directly, well defined in the theory, or determined via simulation studies as:

\begin{equation}
A_0 = \frac{A_{\mathrm{SR}}^{\mathrm{Exp}} \cdot \left(1 + \Delta_{\mathrm{Exp}} \right) }{P \cdot \left( 1 + \Delta_{\mathrm{Th}} \right) \cdot \beta/2},
\end{equation}

\noindent where the energy dependence on $\Delta_{\mathrm{Exp}}$ was also dropped for 
aesthetics. The super-ratios are calculated and systematic effects applied 
over discrete energy bins, and ultimately one can fit over a well chosen energy range to determine $A_0$.
The optimization of this energy range is discussed elsewhere.


%--------------------------------------------------------------------------------------------------
\section{Systematic Correctons}

\subsection{Extracting $\Delta_{\mathrm{Exp}}$}

\subsubsection{Determining Relative Contributions to $\Delta_{\mathrm{Exp}}$}

\subsection{$\Delta_{\mathrm{Th}}$ Contributions}

\subsection{Polarimetry}

\subsection{Energy Reconstruction}

%--------------------------------------------------------------

\section{Analysis Choices}












