\documentclass[final]{ukthesis}
%you must include these 2 packages.
\usepackage{amsmath}

\usepackage[pdfauthor={Michael A.-P. Brown},
  pdftitle={Determination of the Neutron $\beta$-Decay Asymmetry
    Parameter $A_0$ Using Polarized Ultracold Neutrons},
            pdfsubject={The Subject},
            pdfkeywords={Some Keywords},
            pdfproducer={Latex with hyperref},
            pdfcreator={latex->dvips->ps2pdf},
            pdfpagemode=UseOutlines,
            bookmarksopen=true,
            bookmarksnumbered=true,
            hidelinks]{hyperref}

\usepackage{memhfixc}
\usepackage{float}
\usepackage{graphicx}
\usepackage{subfig}
\usepackage{csquotes}
\usepackage{mathtools}
\usepackage{gensymb}
\usepackage{rotating}
\usepackage{caption}

%\usepackage[small]{titlesec}
%\setcounter{secnumdepth}{4}
\setcounter{tocdepth}{3}

%\changetocdepth{3}
%\maxsecnumdepth{subsection}
\setsecnumdepth{subsubsection}


\usepackage[
backend=bibtex,
%style=nature,
style=alphabetic,
%sorting=ynt
]{biblatex}
\addbibresource{references}


%%%%%%%%%%%%%%%%%%%%%%%%%%%%%%%%%%%%%%%%%%%%%%%%
\begin{document}
%author data
\author{Michael A.-P. Brown}
\title{Determination of the Neutron $\beta$-Decay Asymmetry
    Parameter $A_0$ Using Polarized Ultracold Neutrons}
\abstract{
  The UCNA Experiment at the Los Alamos Neutron Science Center (LANSCE) is the first measurement
  of the $\beta$-decay asymmetry parameter $A_0$ using polarized ultracold neutrons (UCN). $A_0$,
  which represents the parity-violating angular correlation between the direction of the initial
  neutron spin and the emitted decay electron's momentum, determines $\lambda=g_{A}/g_{V}$, the ratio
  of the weak axial-vector and vector coupling constants. A high-precision determination of
  $\lambda$ is important for weak interaction physics, and when combined with the neutron lifetime
  it permits an extraction of the CKM matrix element $V_{ud}$ solely from neutron decay. At LANSCE,
  UCN are produced in a pulsed, spallation driven solid deuterium source and then polarized via
  transport through a 7 T magnetic field. Their spins can then be flipped via transport through
  an Adiabatic Fast Passage spin flipper located in a low-field-gradient 1 T field region prior
  to transport to a decay storage volume situated within a 1 T solenoidal spectrometer. Electron
  detector packages located at each end of the spectrometer
  provide for the measurement of decay electrons. Previous
  UCNA results (based on data collected in 2010 and earlier) were limited by systematic
  uncertainties, in particular those from the UCN polarization, calibration of the electron
  energy, electron backscattering, and angular acceptance of events.
  This dissertation will present a background of Neutron
  Decay, an overview of the UCNA Experiment, followed by a detailed report on the entire analysis
  process for data acquired during run periods in 2011-2012 and 2012-2013.}

\advisor{Dr. Bradley Plaster}
\keywords{UCN, Neutrons}
\dgs{Dr. Chris Crawford}
%the title pages
\frontmatter
\maketitle


\begin{dedication}
  I dedicate this to my daughter, Scarlett Marie, for whom I am completing this thesis in a timely manner, and
  also, to those who I have lost, my brother Kenny, stepfather Bill, and stepmother Diana. 
  
\end{dedication}

\begin{acknowledgments}
  First and foremost, this work was made possible through the
  guidance provided by Dr. Brad Plaster. I thank you and appreciate not only the countless
  hours we spent talking about physics, but also the friendship that has developed. I also would
  like to thank Dr. Renee Fatemi, Dr. Susan Gardner, and Dr. Kevin Donohue for serving on
  my dissertation committee and seeing this process through. 

  I am very thankful to my entire family for constantly taking interest in everything I do.
  To my parents, Cindy Spurlin and John Wade Brown, I thank you from the bottom of
  my heart for pushing me to always be my best. I especially want to thank my loving
  wife Kirstie, as she not only celebrated the good times over the past seven years, but she
  endured the difficult times as well. You are my rock, and you are the reason I made it
  this far. Also to my wife's parents, Joe and Lisa Brangers, I would like to thank you for
  the immense amount of support you give me and Kirstie. We appreciate you more than you could know.
  Last but not least, I thank my best dog friend Piper for keeping me sane.

  I have been blessed with friends, old and new, who I want to thank for making life more enjoyable
  throughout this process.
  To all of my Crossfit friends in Frankfort and to my college golf teammates, the time
  we have spent together has been a necessary and welcomed distraction from my research. To my
  childhood friends, thank you for the love you show even when we are separated by thousands of miles
  at times. To all of the UCNA collaborators who I have had the privelege of working alongside,
  thank you for always being willing to help and for making research enjoyable.
  The list of friends would not be complete without giving a special thank you to Dr. Nima Nouri
  and Dr. Kevin Adkins. Nima, we sat together in the same lab for five years, and we made every effort
  to not mention physics to each other, which led to a friendship built for a lifetime.
  And Kevin, cheers to ten years of doing physics together, from
  undergrad through graduate school, through your wedding and mine, and finally to a daughter for each
  of us. Thank you for the time we spent learning physics together, but more importantly thank you for
  your friendship.
  
\end{acknowledgments}

\tableofcontents\clearpage
\listoffigures\clearpage
\listoftables\clearpage

%----------------------------------------------

\mainmatter
%\DoubleSpacing
\OnehalfSpacing

% Chapter 1
%\chapter{Introduction to the theory of neutron $\beta$-decay}
\chapter{Introduction to the theory of neutron $\beta$-decay}
\label{ch:Introduction}

%%%%%%%%%%%%%%%%%%%%%%%%%%%%%%%%%%%%%%%%%%%%%%%%%%%%%%%%%%%%%%%%%%%%%%%%%%%%%%%
%%%%%%%%%%%%%%%%%%%%%%%%%%%%%%%%%%%%%%%%%%%%%%%%%%%%%%%%%%%%%%%%%%%%%%%%%%%%%%%

\section{The Standard Model}
The Standard Model of particle physics encompasses everything we know about
interactions between particles and describes nature using the most fundamental
building blocks yet discovered. This section is used as an introduction
to these building blocks, the quarks and leptons, and to the mediators of the
interactions between them, to preface the upcoming descriptions of the neutron
and $\beta$-decay.

\subsection{The Building Blocks}

\subsection{The Forces}

\subsection{Symmetries}

\section{Properties of the neutron}
\label{sec:neutronProperties}
Let us take a moment to give a very brief introduction to the neutron to motivate the
upcoming sections. This will also serve as a useful tool for those who are
unfamiliar with nuclear and field theories, as the following sections become
quite technical. The majority of this dissertation can be read and mostly understood
without much knowledge of the theoretical description behind neutron $\beta$-decay,
so the properties of the neutron are a natural starting point.

The neutron is a net neutrally charged composite particle. The terms net and
composite hint at the inner structure of the neutron, made up of fundamental
particles called quarks, in this case two down ($d$) quarks and one up ($u$) quark.
The quarks do carry charge, with the up charge $+1/2$ and the down charge
$-2/3$, so that the net charge of the neutron is zero. Four other quarks exist,
the charm, strange, bottom, and top in order of increasing mass. The neutron
is the second lightest three-quark composite particle (baryon) one can imagine
behind only the proton.

The neutron undergoes $\beta$-decay, defined as a transition from the neutron
to a proton, electron, and electron anti-neutrino,
\begin{equation*}
  n\rightarrow p + e^- + \bar{\nu}_{e}.
\end{equation*}
\noindent The three-body decay gives rise to a continuous energy spectrum
for the proton, electron, and anti-neutrino by conservation of energy and momentum
as seen in figure \ref{fig:spectralShapes}. The lifetime of the free neutron is
approximately fifteen minutes. 

%%%%%%%%%%%%%%%%%%%%%%%%%%%%%%%%%%%%%%%%%%%%%%%%%%%%%%%%%%%%%%%%%%%%%%%%%%%%%%%
%%%%%%%%%%%%%%%%%%%%%%%%%%%%%%%%%%%%%%%%%%%%%%%%%%%%%%%%%%%%%%%%%%%%%%%%%%%%%%%

\section{$\beta$-decay of the neutron}

\subsection{The weak interaction}
Talk about the existence of the weak interaction within the standard model and
the vertices and maybe write down the leptonic and hadronic current.

\subsection{Correlation Coefficients}
\subsection{Parity violation}
Maximally violates parity, talk about Wu's experiment. Emphasize that this is
precisely what we are measuring. Connect back to the relationship in the hadronic
current

\section{Ultracold Neutrons}




\copyrightnotice

% Chapter 2
%\chapter{UCNA Experiment}
\chapter{UCNA Experiment}
\label{ch:UCNA_Experiment}
%%%%%%%%%%%%%%%%%%%%%%%%%%%%%%%%%%%%%%%%%%%%%%%%%%%%%%%%%%%%%%%%%%%%%%%%%%%%%%%
%%%%%%%%%%%%%%%%%%%%%%%%%%%%%%%%%%%%%%%%%%%%%%%%%%%%%%%%%%%%%%%%%%%%%%%%%%%%%%%
%%%%%%%%%%%%%%%%%%%%%%%%%%%%%%%%%%%%%%%%%%%%%%%%%%%%%%%%%%%%%%%%%%%%%%%%%%%%%%%

The UCNA Experiment is a rather mature experiment, having taken data from
2007-2013, that aims to precisely determine a value of $A_{0}$, the neutron
$\beta$-decay asymmetry parameter. Here an overview of the experimental
apparatus and critical components is given, as well as previous measurements
of $A_{0}$ from other experiments. 


\section{Overview}
\label{sec:Overview}

Most neutrons utilized for experimental purposes possess energies upwards of
(insert number here) and are what we call fast neutrons. As (insert energy
spectrum figure for neutrons) figure blah shows, we define ultracold neutrons
as thos which have energies on the neV scale and typical speeds of ${<8m/s}$.
The goal of UCNA is to produce UCN in a solid deuterium source which is fed
higher energy neutrons via a spallation source at the end of an $800$ MeV
proton beam. These UCN are then guided towards a material trap where they can
decay. During travel, the UCN pass through a series of polarizing magnets
which allows the experimenter to control the spin state of the neutrons in
the trap during any run. Utilizing a strong ${~1 T}$ magnetic field in the
decay volume, decay electrons are guided towards detectors at either end of
the decay volume where their energy can be reconstructed. From knowledge about
the initial direction and the energy of the neutron, one can construct an
energy dependent asymmetry and determine a value for $A_{0}$.

\section{Los Alamos Neutron Science Center}
\subsection{800 MeV Proton Accelerator}

Blah Blah

\subsection{Ultracold Neutron Source}

Blah Blah

\subsection{Experimental Hall}

Here we highlight some of the other experiments taking place.


%%%%%%%%%%%%%%%%%%%%%%%%%%%%%%%%%%%%%%%%%%%%%%%%%%%%%%%%%%%%%%%%%%%%%%%%%%%%%%%
%%%%%%%%%%%%%%%%%%%%%%%%%%%%%%%%%%%%%%%%%%%%%%%%%%%%%%%%%%%%%%%%%%%%%%%%%%%%%%%
\section{Experimental Setup}
\label{sec:Setup}

BLAH BLAH


%%%%%%%%%%%%%%%%%%%%%%%%%%%%%%%%%%%%%%%%%%%%%%%%%%%%%%%%%%%%%%%%%%%%%%%%%%%%%%
\section{Previous measurements of $A_{0}$}
\label{sec:Previous_results}

\subsection{2010 UCNA Analysis}
\subsection{PERKEO}







\copyrightnotice

% Chapter 3
%\chapter{UCNA Analysis}
\chapter{UCNA Analysis}
\label{ch:UCNA_Analysis}
%%%%%%%%%%%%%%%%%%%%%%%%%%%%%%%%%%%%%%%%%%%%%%%%%%%%%%%%%%%%%%%%%%%%%%%%%%%%%%%
%%%%%%%%%%%%%%%%%%%%%%%%%%%%%%%%%%%%%%%%%%%%%%%%%%%%%%%%%%%%%%%%%%%%%%%%%%%%%%%
%%%%%%%%%%%%%%%%%%%%%%%%%%%%%%%%%%%%%%%%%%%%%%%%%%%%%%%%%%%%%%%%%%%%%%%%%%%%%%%

\iffalse
Designing an experiment and collecting the right data are non-trivial alone,
but interpreting the results lends itself to a whole new train of thought.
The rest of this thesis is spent describing the details of such an
analysis, with this chapter summarizing important aspects,
such as terminology to be used later and the
model used to characterize our detector response.
\fi

This chapter is dedicated to introducing important aspects of the analysis that
is not strictly tied to calibrations and asymmetry extraction, but rather
data collection and processing to turn raw detector signals into values that
can be calibrated and analyzed. 





%--------------------------------------------------------------

\section{Outline of Analysis Steps} \label{sec:outline}

To preface the rest of this chapter, we can highlight
the general process of the analysis beginning with a raw detector
ADC value from each PMT and finishing with an asymmetry.
Below are the general steps:

\begin{itemize}
\item Determine pedestals for each PMT and subtract the pedestal from each data event.
\item Measure the gain of each PMT and divide the drifts out of the signal.
\item Apply a PMT-by-PMT calibration to determine the expected position dependent
  energy deposited in the scintillator
  for each event.
\item Correct for the position dependent response of each PMT
  to return a visible energy as seen by each PMT, $E_{\mathrm{vis},i}$.
\item Combine the four PMT energies into a single deposited energy, $E_{\mathrm{vis}}$.
\item Convert this combined estimate of the energy deposition to a final
  reconstructed energy, $E_{\mathrm{recon}}$, to use in analysis.
\item Calculate an asymmetry and apply all systematic corrections.
\end{itemize}

It is important to point out early on the difference between $E_{\mathrm{vis}}$ and
$E_{\mathrm{recon}}$. The deposited or visible energy, $E_{\mathrm{vis}}$, is an estimate of the energy
physically deposited in the
scintillator by a particle, while the reconstructed energy, $E_{\mathrm{recon}}$, is an estimate of
the true initial energy of an event. The two are different due to the electron losing energy
as it traverses through the windows of the decay trap, the windows of the MWPC,
the MWPC itself, and the dead layer of the scintillator. Of course the analysis could be
done in terms of $E_{\mathrm{vis}}$, but it is more convenient to express the results
in terms of the true electron energy spectrum, and also the energy dependent theory modifications
are in terms of the true initial energy of an event.


\section{Energy Response from Detector Response} \label{sec:EnergyResponse}
Following the process outlined in the previous section, we can derive an
expression which takes a detector signal to an energy estimate for each PMT.
The visible energy $E_{\mathrm{vis},i}$ deposited
in the scintillator as seen by a single PMT $i$ for an event at position $(x,y)$ 
is given by the following: 

\begin{equation} \label{eq:EvisResponse}
E_{\mathrm{vis},i} = \eta_i^{-1}(x,y) \cdot f_i\bigg( \Big( \mathrm{ADC}_i - p_i(t) \Big) \cdot g_i(t) \bigg)  ,
\end{equation}

\noindent where 
\begin{align*}
&f_i = \textrm{linearity relation from ADC channels to light output (approximately energy),}\\
&\eta_i(x,y) = \textrm{PMT position map factor to correct for position dependence of light response,} \\
&p(t) = \textrm{mean pedestal value for PMT } i,\\
&g(t) = \textrm{gain correction factor for PMT }i.
\end{align*}

This expression is exact in the case where all values are determined with infinite
precision and without stochastic fluctuations. Unfortunately, each parameter on the right
side of this equation is either stochastic in itself (as is the ADC response), or it was
determined via observation of a random process (the gain and pedestal), and so
the underlying value for any given event may not be the same as the value applied in the
above expression. Thus what we
really resolve is an approximation to the energy, which comes with some uncertainty. This
uncertainty will be addressed later in section \ref{ssec:energyRecon}.
For the time being, keep in mind the above list of
input parameters which take us from an ADC signal (detector signal) to an energy signal. 

\subsection{Combining PMT Responses}

For every two-fold trigger, all four PMTs from each side (eight in total) are read out by the
DAQ. Upon application of the individual detector calibrations, this yields
eight visible energies, $E_{\mathrm{vis},i}^\mathrm{E,W}$, where $i$ runs from 1 to 4 for the
four PMTs on each side. For each detector, the four available energies should
be combined to create a visible energy for each side, $E_{\mathrm{vis}}^\mathrm{E,W}$. While the
majority of events only strike one scintillator so only one of E/W visible energies will
be useful, the Type 1 events will have a usable energy on each side.

Now not all PMTs are created equal. Some PMTs have better resolution than others, thus these PMTs
should contribute more to the average energy. A simple average would not account for this, but
a weighted average, upon definition of the weights, will suffice:
%
\begin{equation}
  E_{\mathrm{vis}}^\mathrm{E,W} = \frac{\sum_{i=1}^{4} w_i E_{\mathrm{vis},i}^{\mathrm{E,W}}}{\sum_{i=1}^{4} w_i},
\end{equation}
%
where $w_i=\frac{1}{\sigma_i^2}$ are the weights for each $E_{\mathrm{vis},i}^{\mathrm{E,W}}$. We will drop the superscript
E/W for now, understanding that what follows is done for each detector.

Now let's assume that the $E_{\mathrm{vis},i}$ from equation \ref{eq:EvisResponse} is an approximation
for an event which deposited exactly the energy $E_Q$ (the $Q$ stands for ``quenched'', which will
be described in section \ref{sssec:Equenched}). A certain amount of this $E_Q$ is visible to each
PMT, given by $\eta_i(x,y)E_Q$, as each PMT only captures a portion of the total number of photons produced
in the scintillator.
Now if we want to relate this to the signal read out by the DAQ
from the PMT, we must multiply by a PMT resolution factor $\alpha_i$ to convert from energy
to photoelectrons, giving $N_i = \alpha_i \eta_i(x,y) E_Q$.
These PMT resolution factors are calculated during the calibration process
and will be discussed in section \ref{ssec:PMTresolution}, but for now, assume they are known.

The number of photoelectrons produced by a PMT is a stochastic process, so we can write the actual
measured number of photoelectrons as $\overline{N_i} = N_i \pm \sqrt{N_i}$. The fractional uncertainty
is then $1/\sqrt{N_i} = 1/\sqrt{\alpha_i \eta_i(x,y) E_Q}$. Now, since the measured number of photoelectrons,
$\overline{N_i}$, is directly proportional to the ADC signal of the PMT and thus also the visible energy
of the PMT by equation \ref{eq:EvisResponse}, we have
%
\begin{equation}
  E_{\mathrm{vis},i} = E_Q \pm \frac{E_Q}{\sqrt{\alpha_i \eta_i(x,y) E_Q}} =  E_Q \pm \sqrt{\frac{E_Q}{\alpha_i \eta_i(x,y)}}.
\end{equation}
%

The measured value of $E_{\mathrm{vis},i}$ is actually sampled from a distrubution with $\mu$ and $\sigma$ given by the
above equation, but we can assume that the best weight is given by $w_i = \frac{1}{\sigma_i^2} = \frac{\alpha_i \eta_i(x,y)}{E_Q}$.
Then the final combined $E_{\mathrm{vis}}$ for a single detector becomes
%
\begin{align}
  E_{\mathrm{vis}} &= \frac{\sum_{i=1}^{4} \frac{\alpha_i \eta_i(x,y)}{E_Q} E_{\mathrm{vis},i}}{\sum_{i=1}^{4} \frac{\alpha_i \eta_i(x,y)}{E_Q}}
  = \frac{\sum_{i=1}^{4} \alpha_i \eta_i(x,y) E_{\mathrm{vis},i}}{\sum_{i=1}^{4} \alpha_i \eta_i(x,y)} 
  = \frac{\sum_{i=1}^{4} \alpha_i f_i(q)}{\sum_{i=1}^{4} \alpha_i \eta_i(x,y)}
\end{align}
%
where we have used equation \ref{eq:EvisResponse} with $q=\big( \mathrm{ADC}_i - p_i(t) \big) \cdot g_i(t)$.
The uncertainty on the final weighted average above is given by
\begin{equation}
  \delta E_{\mathrm{vis}} = \frac{1}{\sqrt{\sum_{i=1}^{4} w_i}} = \frac{1}{\sqrt{\sum_{i=1}^{4} \frac{\alpha_i \eta_i(x,y)}{E_Q}}}
    \approx  \sqrt{\frac{E_{\mathrm{vis}}}{\sum_{i=1}^{4}\alpha_i \eta_i(x,y)}}.
\end{equation}

As pointed out in \cite{mpmThesis}, the position map values only enter into the weighted average
as part of the sum in the denominator and multiplied by the $\alpha_i$ conversion factors, which is a smoother
function of position than the individual position map values, and thus sensitivity to uncertainty in position map
values $\eta_i$ should be minimized.

\subsection{$E_{\mathrm{vis}}$ to $E_{\mathrm{recon}}$}

As mentioned in section \ref{sec:outline}, $E_{\mathrm{vis}}$ is an estimate of the
energy visible to the PMT from deposition in the scintillator. The more meaningful
initial energy of the particle must be determined using a separate parameterization.
To determine such a parameterization, simulated data is used so that the initial
energies of the events are known precisely, and then the response of the detector
to different event energies and types can be modeled.

\begin{figure}[h]
  \centering
  \includegraphics[scale=0.78]{3-UCNAAnalysis/2011-2012_Evis_to_Erecon.pdf}
  \caption{Parameterization between $E_{\mathrm{vis}}$ and $E_{\mathrm{recon}}$ for
    2011-2012. The mapping for the other geometries are very similar and thus are
    not shown.}
  \label{fig:Erecon}
\end{figure}

The mapping between $E_{\mathrm{vis}}$ and $E_{\mathrm{recon}}$ is determined separately
for 2011-2012, 2012-2013, and 2012-2013 with isobutane in the wirechamber.
Using the simulated $\beta$-decay spectrum and a characteristic set of detector
response variables, the simulated events are processed in the same manner as
they would be for data. They are identified as Type 0,1, or 2/3, and their
$E_{\mathrm{vis}}$ values for each detector are determined. Then the events are
grouped into histograms using their initial energy (we will call it $E_{\mathrm{recon}}$,
but really this is the true energy of the events), with the histograms corresponding
to 10~keV groups from 0~keV to 790~keV. Then each of these histograms are fit to
determine the average $E_{\mathrm{vis}}$ value for a given $E_{\mathrm{recon}}$. The results
of these fits for each event type in each detector for 2011-2012 is shown as the data
points in figure \ref{fig:Erecon}. The other geometries look very similar and thus are
left out. Data points with low statistics at low energies were dropped.

The fit to the data is of the form
\begin{equation}
  E_{\mathrm{recon}} = C_1 + C_2E_{\mathrm{vis}} + \frac{C_3}{E_{\mathrm{vis}}} + \frac{C_4}{{E_{\mathrm{vis}}}^2},
\end{equation}
and then the fit is extrapolated continuously to zero from the lowest energy data point using
\begin{equation}
  E_{\mathrm{recon}} = C_5{E_{\mathrm{vis}}}^{C_6}
\end{equation}
with the parameters $C_5$ and $C_6$ determined by ensuring the two equations are equal at the
transition point along with their first derivatives.

Using the fitted parameters, an $E_{\mathrm{recon}}$ value is determined
for every event by plugging its weighted average $E_{\mathrm{vis}}$ into the proper equation.
It should be noted that for Type 1 events, the visible energies of both the East and West
detectors are added together when determining the parameterization, so
the same is done when applying the above parameterization to data. This is done to utilize as much
detector information as possible. The side for a Type 1 event is set to
the detector with the earlier trigger, making it the primary detector.
Thus, for a Type 1
event, one would plug $E_{\mathrm{vis}} = E_{\mathrm{vis}}^E + E_{\mathrm{vis}}^W$ into the $E_{\mathrm{recon}}$
equation for the side that triggered first.


\section{Time-dependent Detector Corrections}

Obviously the system is not immune to drifts in signals due to variations
in time. There are many sources of such drifts, ranging from simple
electronic noise to changes in temperature. We deal with such time-dependent
effects using pedestal subtraction, gain correction, and constant monitoring
of backgrounds.

\subsection{Pedestal Subtraction}
The pedestal is a measure of the inherent detector signal, or baseline, 
upon which all other data signals lie. In terms of PMT signals, you can imagine 
the pedestal as a non-zero ADC value corresponding to zero input, or an offset.
You might say that the experiment can be run without caring about an offset
because the calibration will take this into account, which would be the case 
if the pedestals were constant or if we calibrated each run against itself, but 
neither is the case. We use a collection of subsequent runs to form our 
calibration sets, and these sets then calibrate data which is often taken hours,
or even days, earlier or later. Thus changing pedestals can be worrisome, and care
must be taken to determine the pedestals and subtract them from data.

To determine a pedestal, events must be chosen where there was a global trigger, but
the PMT of interest does not trigger and preferably there is no signal
whatsoever in the scintillator on that side. Obvious choices for these events are
UCN monitor triggers, opposite side two-fold PMT triggers, and high-threshold $^{207}\mathrm{Bi}$
pulser triggers from other PMTs. Once there is a global trigger, we can use the individual TDC
to ensure there was no individual trigger, and the events can be 
histogrammed for the PMT of interest. The mean of pedestal
peak can be taken as the average pedestal for a single run,
and this value can be 
subtracted from every subsequent reading of this PMT.

One interesting thing to note is that the discriminators, which determine whether
a component triggers,
for all PMTs are housed 
together, which leads to correlations between the PMT triggers. In a perfect world, 
each PMT would have one pedestal, and that pedestal wouldn't care about other PMT's signals.
Instead, what we see is that the pedestals
can be dependent on the type of events that are chosen 
to construct the pedestal, and the effect can be \~10~channels for some PMTs.
This indicates that the pedestal for one PMT may be dependent
on the signal present in another PMT. These shifts are important as a
pedestal shift of ~5-10 channels maps to an offset of roughly
~5-10 keV, as the PMTs show roughly 1:1 correspondence between ADC and keV.

The influence of event type on pedestal values means we must carefully choose which events
to use when calculating the pedestal.
The best choice would be UCN monitor events due to there being zero signal 
in the electronics box housing the PMT electronics, and these thus would give
the cleanest measurement of the PMT pedestal.
These are unfortunately the first event type we can eliminate as they are only present
during $\beta$-decay runs (when UCN are produced and thus create UCN monitor triggers)
and not during calibration runs (taken during the day when the beam is off). 
Of the remaining two options, the choice was made to use the
two-fold PMT triggers from the opposite detector rather than $^{207}\mathrm{Bi}$ pulser
events. The choice is somewhat arbitrary, because what is important is that we choose
a consistent subset of data for both calibration and $\beta$-decay data, but the
opposite side two-fold triggers do better represent the baseline present in each PMT
for data events when compared to the much higher signal present from the $^{207}\mathrm{Bi}$
pulser.


\iffalse
\begin{figure}[h] 
\centering
\includegraphics[scale=.25]{3-UCNAAnalysis/ImageHolder.pdf}
\caption{Pedestal values for a $\beta$-decay run determined using different 
types of events to illustrate the cross-talk between PMTs. (UCN Monitors, 
Bi triggers, Opposite side triggers, same side 2-fold triggers. Also shown 
is the dependence of the pedestal on which PMT triggers in the Bi Pulser. NOTE:
Choose West PMT4 in an early 2012/2013 beta run) }
\label{fig:peds_types}
\end{figure}

\begin{figure}[h] 
\centering
\includegraphics[scale=.25]{3-UCNAAnalysis/ImageHolder.pdf}
\caption{Example pedestals from all 8 PMTs}
\label{fig:peds_ind}
\end{figure}
\fi

\begin{figure}[p]
\centering
\includegraphics[page=1,scale=0.8]{3-UCNAAnalysis/2011-2012_pedestals.pdf}
\caption{Pedestal means as a function of run number for 2011-2012 East Detectors. Error bars are the
  RMS of the measured pedestal. The red lines indicate what ranges of runs belong to
  different calibration periods, and the red marker is the calibration reference run,
  which will be discussed in later sections.}
 \label{fig:peds_timeDep}
\end{figure}

\begin{figure}[p] 
\centering
\includegraphics[page=2,scale=0.8]{3-UCNAAnalysis/2011-2012_pedestals.pdf}
\caption{Pedestal means as a function of run number for 2011-2012 West Detectors. Error bars are the
  RMS of the measured pedestal. The red lines indicate what ranges of runs belong to
  different calibration periods, and the red marker is the calibration reference run,
  which will be discussed in later sections.}
\label{fig:peds_timeDep}
\end{figure}

With the event type chosen, we extract the mean and RMS of the pedestal peak
for each PMT in every run. The pedestal mean (referred to as simply
the pedestal) is then subtracted from the ADC values for all events. This effectively
removes the time-dependent baseline from the detector signals. The time dependence of
the pedestals can be seen for the East PMTs from 2011-2012 in figure \ref{fig:peds_timeDep}.
Most of the PMTs have pedestals which remain quite stable, but PMT East 3 shows the
importance of a run-by-run pedestal subtraction.



\subsection{Gain Correction}

\subsubsection{$^{207}\mathrm{Bi}$ Pulser}
%The goal of the $^{207}\mathrm{Bi}$ gain monitoring pulser system is to create a standard
%candle type signal present in all PMTs to be tracked over time. 
The primary gain monitoring system consists of a small amount $^{207}\mathrm{Bi}$
deposited within a small block of scintillator. The scintillator was surrounded by
light reflecting material on three sides, with the fourth side covered with an optical
attenuator to attempt to match the light output of the \~1~MeV conversion line in the $^{207}\mathrm{Bi}$
to the light output of 1~MeV of energy deposited in the detector scintillator. The pulser was then
attached directly to the PMT next to where the light guides attached to the PMT.
To allow for a single-PMT high threshold trigger, the signal was split off to a different
discriminator than the one used when determining a two-fold trigger. These high threshold
discriminators then allowed for pulser triggers with a distinct pulser identification \cite{mpmThesis}.

An unfortunate but low impact issue with the pulser involves the amount of attenuation applied to the
pulser signal. The pulser peak lies well beyond the equivalent of 1~MeV of light as would be produced
in the detector scintillator, and therefore far outside the range of the $\beta$-decay spectrum. This
is not a serious issue as the PMTs seem to be quite linear, so even a peak well outside the
energy range of interest should suffice.

\begin{figure}[h] 
\centering
\includegraphics[scale=.60]{3-UCNAAnalysis/gain_bismuth2.pdf}
\caption{Example $^{207}\mathrm{Bi}$ spectrum and fit for a single PMT over
  the course of a run. The upper and lower bismuth conversion electron peaks are
  seen, along with a distribution from decay gammas. Note that the gamma rays are
  visible because the pulser is attached directly to PMT and there is no type
  of veto (like the MWPC) to remove them.}
\label{fig:biPulser}
\end{figure}

The $^{207}\mathrm{Bi}$ pulser peak is fit by a Gaussian on a run-by-run basis, allowing for gain corrections on the time
scale of a single run. The fit was done iteratively, with the initial guesses for mean and fit range determined
by stepping backwards from the last bin and using a self-written algorithm to search for the peak. Then the peak
was fit five times consecutively, with each successive fit being fed the previous fits mean and sigma. This
made sure the fit converged as best as possible on the mean of the pulser peak. An example pulser peak and fit
can be seen in figure \ref{fig:biPulser}. The asymmetric fit (range extends farther above the peak than below)
is actually a characteristic part of the iterative
fitting algorithm developed for use throughout this analysis. For Gaussian like peaks occurring as a result of
a process like energy loss in a scintillator followed by PMT amplification, the upper end of the peak is
inherently more Gaussian due to the lower end having a tail from extraneous energy losses. 

The method for applying the gain correction is as follows. First, a reference gain must be determined
to normalize all other gains against. This was chosen to be what is called the ``reference run'', and it
typically consists of a manually inspected source run within each source calibration period. The gain factor
is then calculated as the ratio of the pulser peak in a given run divided by the pulser peak in the reference
run, or
%
\begin{equation}
  g_i = \frac{\mu_i}{\mu_{\mathrm{ref}}}
\end{equation}
%
This automatically defines the gain of the reference run to be $g_{\mathrm{ref}}=1$. Then all other runs which are
calibrated by a certain run period have gain factors which vary based on the fitted pulser value. The time
dependence of the gain values in 2011-2012 can be seen in figures \ref{fig:2011-2012pulser_East}
and \ref{fig:2011-2012pulser_West}. The behavior is similar in 2012-2013.

Some problems with the $^{207}\mathrm{Bi}$ pulser did occur. There were several periods where the pulser
simply did not work for a certain PMT. This was always limited to a single PMT not working, and when this
was the case the PMT without a pulser signal was not used when reconstructing the energy. This has
minimal effect on the energy reconstruction though, as the ``bad'' PMT is still used when determining
a two-fold trigger, and the remaining three PMTs contain sufficient information for reconstructing the
energy deposited. Periods where $^{207}\mathrm{Bi}$ pulser information is missing are evident in
figures \ref{fig:2011-2012pulser_West} where there is missing data for certain PMTs over extended ranges.
It should be noted that in 2012-2013, West PMT4 never had a functioning pulser.

\begin{figure}[p] 
  \centering
  \includegraphics[page=3,scale=.8]{3-UCNAAnalysis/2011-2012_gain.pdf}
  \caption{Gain factors, $g_i$, as a function of run number for 2011-2012 East Detectors.
    The red lines indicate what ranges of runs belong to
    different calibration periods, and the red marker is the calibration reference run.}
  \label{fig:2011-2012pulser_East}
\end{figure}

\begin{figure}[p] 
  \centering
  \includegraphics[page=4,scale=.8]{3-UCNAAnalysis/2011-2012_gain.pdf}
  \caption{Gain factors, $g_i$, as a function of run number for 2011-2012 West Detectors.
    The red lines indicate what ranges of runs belong to
    different calibration periods, and the red marker is the calibration reference run.}
  \label{fig:2011-2012pulser_West}
\end{figure}

\subsubsection{Endpoint Stabilization} \label{sssec:endpoint}

There are unexplained longer term gain fluctuations that do not seem to be captured by the
$^{207}\mathrm{Bi}$ gain monitoring system that can be seen by monitoring the endpoint of the
$\beta$-decay spectrum. These are corrected by applying a second gain factor, $g_{\mathrm{ep},i}$ for PMT $i$,
as the last step in the calibration process. This endpoint gain factor is determined
by comparing the endpoints as seen by each
calibrated PMT to the expected endpoint from the simulation. The final energy for each PMT
is then multiplied by this factor. This does not force the final reconstructed energy endpoint to
match the final reconstructed simulation endpoint exactly,
as the final spectra are the weighted average of the four PMT responses, but rather it corrects 
some systematically shifted periods of data which consistently exhibited endpoints $>30$~keV
away from the expected endpoint.

The method for calculating $g_{\mathrm{ep}}$ was developed previously by M. Mendenhall in section
5.2.2 of \cite{mpmThesis}. For the sake of clarity, the process is repeated here.

\begin{figure}
  \centering
  \includegraphics[scale=0.75]{3-UCNAAnalysis/kuriePlot.pdf}
  \caption{Example Kurie plot with linear fit to extract the y-intercept, which is
    the endpoint energy. The deviations from a straight line are due to the
    trigger efficiency at low energies and finite resolution at high energies. Care
    must be taken to fit in a region which more appropriately characterizes the true
    underlying electron energy spectrum.}
  \label{fig:kuriePlot}
\end{figure}
  

The endpoints are fit using a Kurie plot \cite{kurie1936radiations}, which linearizes the
energy response making for easy determination of the endpoint energy. If we approximate
the decay rate as the phase space factor for the neutron, then we can write down
the decay rate as a function of the electron total energy $E$, endpoint
energy $E_0$, and momentum $p$ as
%
\begin{equation}
  S(E) = pE\big(E_0-E\big)^2
\end{equation}
%
From this we see that a linear equation (the Kurie plot) can be formed:
%
\begin{equation}
  K(E) = \sqrt{\frac{S(E)}{pE}} = E_0-E
\end{equation}
%
where the $y$-intercept and $x$-intercept both determine the endpoint energy $E_0$.
An example Kurie plot can be seen in figure \ref{fig:kuriePlot}.

Now we assume that the measured kinetic energy spectrum $T_{\mathrm{meas}}$ is different from the expected
kinetic energy spectrum $T_{\mathrm{sim}}$ by the gain factor $g_{\mathrm{ep}}$ such that
%
\begin{equation}
  T_{\mathrm{sim}} = g_{\mathrm{ep}} T_{\mathrm{meas}}.
\end{equation}
Then we can write the
measured total energy as
\begin{equation}
  E_{\mathrm{meas}} = m_e + g_{\mathrm{ep}}T_{\mathrm{meas}}
\end{equation}
where $c = 1$ is used. The extracted endpoint for the data is now a function of $g_{\mathrm{ep}}$
since
\begin{equation}
  K(E_{\mathrm{meas}}) = \sqrt{\frac{S(E_{\mathrm{meas}})}{pE_{\mathrm{meas}}}} = E_{0,\mathrm{meas}}-E_{\mathrm{meas}},
\end{equation}
and upon proper choice of $g_{\mathrm{ep}}$, $E_{\mathrm{meas}}=E_{\mathrm{sim}}$. To solve for
the proper gain factor, the data endpoint is iteratively fit with the gain factor adjusted upon each iteration
according to
%
\begin{equation}
  g'_{\mathrm{ep}} = \frac{ E_{0,\mathrm{meas}}- m_e}{ E_{0,\mathrm{sim}}- m_e}g_{\mathrm{ep}} =  \frac{ T_{0,\mathrm{meas}}}{ T_{0,\mathrm{sim}}}g_{\mathrm{ep}}, 
\end{equation}
%
where $E_{0,\mathrm{meas}}$ is the extracted endpoint energy from the data with gain factor $g_{\mathrm{ep}}$ applied,
$E_{0,\mathrm{sim}}$ is the expected endpoint energy extracted from simulation, $T_0$ is the endpoint kinetic energy,
and $g'_{\mathrm{ep}}$ is the guess for the next iteration of endpoint fitting.
Once the condition $1-\frac{g'_{\mathrm{ep}}}{g_{\mathrm{ep}}}<10^{-7}$ is met, the value of $g'_{\mathrm{ep}}$ is taken as the final
endpoint gain factor and is saved to the calibration database. This process is carried out for each PMT for every
$\beta$-decay run, and then every event is reprocessed with the new gain factor applied to the visible energy from PMT
$i$ according to
%
\begin{equation}
  E_{\mathrm{vis},i}^{\mathrm{FINAL}} = g_{\mathrm{ep},i} E_{\mathrm{vis},i}.
\end{equation}
Then $E_{\mathrm{recon}}$ is calcuated using the final visible energies from the available PMTs.



\subsection{Time-dependent backgrounds}
Background events which may have some time dependence are removed from the analysis
via dedicated background runs that accompany every $\beta$-decay run.
Subtracting these background rates from the data rates accounts for backgrounds
with roughly a one hour time variation.
Any backgrounds that vary at the sub one hour level may go unnoticed, but
with a signal to background better than 50:1 the contribution from such is minimal.
The background run scheme will be addressed later in this chapter, with the background
subtraction addressed in chapter 5. 

\section{Trigger Thresholds}

Looking ahead to the detector response model that will be implemented when processing
the simulated data, we need to determine the trigger thresholds for each of the
PMTs. With each PMT attached to a leading edge discriminator, the
amount of charge in a pulse that can create a trigger is randomized. The ADC signal
is an integrated signal, so two signals with integrated charge
at roughly the trigger threshold may have
different amplitudes and therefore different likelihoods of passing the discriminator threshold.
Measurement of these thresholds is important for the detector response model within the simulation
in order to properly model the low energy behavior of the measured spectrum.


\subsection{General Model for Trigger Determination} \label{ssec:genTrigModel}

The most important part of determining the trigger threshold shape for 
any detector is the availability of data which was collected no matter if 
the detector or component (PMT) produced a trigger. If such a subset of 
data is available and plentiful, it is straightforward to estimate
the trigger probability by binning the data in some unit proportional to 
energy (whether in energy or something like it isn't important) and taking 
the bin-by-bin ratio of those events that triggered to all of the events in the 
sample. Plotting these ratios as a function of whatever energy-like metric was chosen tells 
you how probable an event of some value is to create a trigger.
Once you have mapped this probability, you can then sample these curves within 
simulation to apply your true trigger threshold to simulated data.


\subsection{Trigger Data Selection}
As mentioned before, the data used for constructing trigger thresholds must not be 
biased towards triggering the PMT of interest. Thus that PMT must not be a mandatory 
component of the global trigger for that event, so care must be taken to choose only 
events which would trigger regardless of the behavior of the PMT of interest. One
other stipulation placed on these events is that they have an opportunity to 
deposit energy in a particular scintillator. The best choice of events which 
satisfy these conditions are those which have a two-fold trigger on the opposite side 
and then backscatter and those which trigger at least three PMTs on the side of 
interest, which guarantees that the scintillator would have triggered with or without 
the PMT of interest.	

\subsection{Determining the Trigger Probability}
One option for determining the trigger probability function (and probably the 
most straightforward) is to calculate the trigger probability for an entire detector as 
a whole as a function of the energy deposited by an event. What you get is a 
function that provides the probability that an event of energy $ E_i $ 
produces some sort of trigger, either 2-fold, 3-fold, or 4-fold, in that 
detector. Initially this method was used for sake of simplicity, and it produced 
reasonable agreement between simulation and data, but there is one 
glaring concern: Determining this trigger function from data requires that the data be 
calibrated first. At first glance this may not seem like much of an issue, but the 
calibration hinges upon the 
simulated peaks at low energy, which in turn rely on the trigger functions. This 
cyclical dependence hinders one from truly understanding any discrepancy between 
simulation and data at low energies, which is exactly the reason this method was 
abandoned.  

Instead, similarly to previous analyses, we decided to calculate the trigger
function on a PMT-by-PMT basis as a function of ADC channels above pedestal. This
encompasses a true characteristic of each component of the detector rather than some
average effect as seen by a detector package, which is what the aforementioned 
method produces. A typical trigger threshold is seen in figure \ref{fig:trigger_thresh}.
As illustrated in section \ref{ssec:genTrigModel}, the ratio of triggering events
to all events was taken in each ADC bin and then fit using the method described
in the following section.
  

\begin{figure}[h] 
\centering
\includegraphics[scale=.6]{3-UCNAAnalysis/triggerThresholds.pdf}
\caption{Typical trigger thresholds as a function of pedestal subtracted
  ADC values from the West PMTs with fits shown in red. The y-axis
  is a probability of triggering given some ADC signal.}
\label{fig:trigger_thresh}
\end{figure}

\subsubsection{Functional Fit of the Trigger Threshold}
From the sigmoid shape of the threshold data in figure \ref{fig:trigger_thresh}, one might guess
the shape of the curve to be a hyperbolic tangent or an error function. The data tends to show
a sharper turn on at lower ADC channels and a softer levelling off to unit probability, which
prompted the use of both the $\mathrm{erf}(x)$ (lower end of transition region)
and $\tanh(x)$ (upper end of transition region),
with a continuous transition between the
two provided by a smoothing function $S_\pm$ defined as
%
\begin{equation}
  S_\pm\big(x;x_0,R\big) \cdot f(x) = \frac{1}{2}\bigg(1\pm\tanh\Big(\frac{x-x_0}{R}\Big)\bigg) \cdot f(x),
\end{equation}
%
which acts to ``turn on/off'' ($+/-$) a function $f(x)$ around the pivot point $x_0$ and with the severity of
the on/off transition determined by the width parameter $R$. The functional fit $F(q)$ then becomes (note that the $\mathrm{erf}(x)$ and $\tanh(x)$ 
have the range (-1,1), so they must be shifted and the range halved to accomodate $0<F(q)<1$):
%
\begin{equation}
  F(q) = \frac{1}{2} \Bigg[ S_-\big(q;\mu,R\big) \bigg(1 + \mathrm{erf}\Big(\frac{q-\mu}{w_1}\Big)\bigg) +
  S_+\big(q;\mu,R\big) \bigg(1+\tanh\Big(\frac{q-\mu}{w_2}\Big)\bigg)\Bigg]
\end{equation}
%
where $q$ is the ADC value and
the free parameters are $\mu$ ($f(q=\mu)\approx 0.5$), $w_1$ (width of $\mathrm{erf}$), $w_2$ (width of $\tanh$), and $R$ (severity of turn on/off).
This function, while motivated purely by inspection of the shape of the trigger threshold, fits the data
quite well with only these four free parameters. An example of the fits for a single $\beta$-decay
run can be seen in figure \ref{fig:trigger_thresh}. 

%-----------------------------------------------------------


\section{Wirechamber Position Reconstruction}

Determining the position of the spiraling electrons is vital for every component of the analysis
that follows. Before now, the gain, pedestals, and trigger thresholds are determined with no
knowledge of the position of an event, but rather basic trigger logic. As will be seen in
the next section, the actual response of the PMT is position dependent, and so we introduce the
position reconstruction algorithm now.

\subsection{Wirechamber signals}

The wirechamber signals for each detector include a summed anode signal, and two cathode signal
collections consisting of 16 individual ``wire'' readings. Again we use the term
wire loosely, as there are technically 64 wires in a plane, but they are read out in groups
of four. The position of each wire group is taken to be the midpoint of the grouping (between
the two center wires). The two sets of cathode signal collections come from perpendicular
planes, so the position can be reconstructed along both the $x-$ and $y-$axis.

The signal is read out by a peak sensing ADC, and so the maximum value of the signal is recorded
for each event. The pedestal is determined using the bismuth pulser events (different from the
PMT pedestals) and is subtracted from the ADC signal for every event. For the cathodes, a software
threshold for each wire is set at 100~channels above pedestal, so only wires above this threshold
will be used in the position reconstruction.

\subsubsection{Wirechamber trigger} \label{sssec:mwpctrigg}

As mentioned in sections \ref{sec:ExpMWPC} and \ref{sec:backscattering}, a wirechamber
software trigger is set and used to eliminate the gamma background and to identify different
backscattering events. One could use either the summed anode signal or a signal formed from
the cathode signals. The best trigger is one that separates the pedestal furthest from the
triggering data. For this analysis, this was found to be the sum of the maximum cathode
signals from the two wire planes. (Add figure of the two signals?)

\subsubsection{Cathode wire clipping}

Low energy event deposit a substantial amount of energy in the MWPC, and thus create large
signals in one or more wires in cathode planes. This can create a signal which is beyond the
range of the ADC, thus creating an overflow event which is read out as the max value of the ADC.
These events are called ``clipped'' events. A plot of the typical ratio of clipped events and the
number of clipped wires can be seen in figure \ref{fig:nClipped}.

\begin{figure}[h]
  \centering
  \includegraphics[scale=0.38]{3-UCNAAnalysis/ImageHolder.pdf} 
  \caption{Clipping in the x-plane. Could do a 2d clipping plot?}
  \label{fig:nClipped}
\end{figure}


A clipped event poses a problem in the position reconstruction because the true signal
is no longer known, only that it was above the maximum ADC value. Using this wire in
the position reconstruction will not properly account for the strength of the
signal at the position of the clipped wire.

\subsection{Method of reconstructing position}

\subsubsection{Events with no cathode clipping}

The algorithm developed for reconstructing the positions of the events for this analysis
is meant to be as simple of possible, with more complex measures taken for special events.
If we have an event that passes through the MWPC, creates a software trigger, and has no
clipped wires, then the position is calculated as the average position of the signals:
%
\begin{equation}
  \bar{x} = \frac{\sum q_i x_i}{q_i},
\end{equation}
%
where the sum runs over wires above the individual pedestal subtracted threshold, $q_i$ are the
pedestal subtracted wire signals, and $x_i$ are the positions of the wires.  This can
be interpreted either as a weighted average of the wire positions with the weights equal to
the wire signals, or it can be seen as a simple average where each unit of the signal is seen
as an event entering into the average with a value equal to the wire position. The final position
of the event is then given by $(\bar{x}, \bar{y})$, where $\bar{y}$ is calculated in the same manner.

\begin{figure}[h]
  \centering
  \includegraphics[scale=0.38]{3-UCNAAnalysis/ImageHolder.pdf} 
  \caption{1d wirechamber position for events with no clipping.}
  \label{fig:wirechamberPosNoClip}
\end{figure}

For well-behaved wires (no clipping, no missing or ``dead'' wire segments), this method produces
a continuous reconstruction of the events based on where they pass through the wirechamber, as shown
in figure \ref{fig:wirechamberPosNoClip}.

\subsubsection{Events with cathode clipping or missing wire signals}

In the event of a clipped wire or the rare case of a missing wire signal, the basic average method
shown above will not work. Using the clipped wire in the average can systematically shift the
reconstructed position if the signal is not symmetric about the clipped wire, and the same can
be said of a missing signal. While this may not show up in the plotted distribution, the method
loses its integrity.

To handle these types of events, a method similar to that used in the previous analysis
\cite{mpmThesis} was adopted. By assuming that the wirechamber charge cloud (or
ionization cloud) takes a Gaussian shape in the MWPC, then we can expect that the signals
in the wires should also sit on a Gaussian given by
%
\begin{equation}
  q = Ne^{(x-\bar{x})^2/\sigma^2},
\end{equation}
%
where $q$ is the signal and $x$ is the position of the signal. If we take the log of this
expression we have
%
\begin{align} \label{eq:gaussPos}
   &\ln(q) = \ln(N) + \frac{(x-\bar{x})^2}{\sigma^2} \\
   &\ln(q) = \frac{1}{\sigma^2}x^2 - \frac{2\bar{x}}{\sigma^2}x + \Big(\ln(N)+\frac{\bar{x}^2}{\sigma^2}\Big) \\
   &\ln(q) = Ax^2 + Bx + C 
\end{align}
%
with
\begin{equation}
  A = \frac{1}{\sigma^2}, \qquad B = - \frac{2\bar{x}}{\sigma^2}, \qquad C = \ln(N)+\frac{\bar{x}^2}{\sigma^2}.
\end{equation}
This equation has three unknowns, thus with three ``good'' wires giving three points on the Gaussian, the unknowns
are fully determined. The three chosen wires are the maximum non-clipped wire and the next two largest signals
surrounding this maximum signal. These two wires can be on one side of the maximum wire, as would be the case if the
maximum wire were the edge wire. The extraction of the mean is the approximation of the center of the event.

\begin{figure}[h]
  \centering
  \includegraphics[scale=0.38]{3-UCNAAnalysis/ImageHolder.pdf} 
  \caption{Collection of extraced $\sigma$ values for the Gaussian determination of event position. The
    average of this distribution is an estimate of the characteristic width of the wirechamber charge cloud,
    $\sigma_c$, used when only two wires are usable in the position reconstruction.}
  \label{fig:wirechamberPosNoClip}
\end{figure}

Now there is also the special case where there are only two usable wires above the individual cathode
wire threshold. If they are consecutive wires, the average method is applied, but if they are not consecutive wires
(so they are separated by a clipped wire(s)), a new approximation must be applied. The expression in
equation \ref{eq:gaussPos} requires three wire signals to estimate a position, so we must eliminate one of the unknowns
to use this method. Again, if we assume that all signals follow a typical Gaussian shape, then the width of the
charge cloud should be roughly constant. Thus we can determine a characteristic width, $\sigma_c$, from the
mean extracted $\sigma$ from the rest of the events (including the non-clipped events). An histogram of the
extracted $\sigma$ can be seen in figure \ref{fig:meanSigma}. A value of $\sigma_c = 8\text{~channels}$ was used
for these type events, and then the mean was calculated by solving \ref{eq:gaussPos} with two unknowns.

Last of all, there is the situation where only one wire was above the individual threshold. For this type of event,
the only reasonable choice is to set the position of the event to the position of this wire.

\subsection{Simulation of wirechamber positions}

In order to reduce any unforeseen systematic effects, an attempt at including
every aspect of the experiment within the simulated data is made. Thus for the wirechamber
position reconstruction, we would like to employ the above methods for every event. 

Embedded within the simulation by M. Mendenhall during the previous analysis
is a model for the charge collection within the wirechamber based on the work
from \cite{mathieson1991induced}. In summary, based on the wirechamber geometry, an estimate
of the charge cloud as seen by the cathode and anode can be calculated for an event
that deposits energy $E_{\mathrm{MWPC}}$ in the wirechamber. The agreement between data and simulation was
shown to be good (\cite{mpmThesis} section 6.3.6), and thus the charge collection
model was used in this analysis.

A new contribution to the model is the application of the observed wirechamber thresholds
and clipping conditions to each of the cathode wires. The model already included is simply an estimation of the charge
collection on the range from $0$ to $\infty$. This could be used in the simple average method for ``good'' events
with no clipping and no missing wires, but then any systematic effects from wire clipping would go unnoticed.

\subsubsection{Wire model}

For each wire in each plane, the response can be characterized by two values: a clipping threshold and a trigger threshold.
Recall that we apply an individual software trigger threshold of 100 ADC channels for each wire, so a model parameter must
be determined for this. Also, in the simulation model, the signal on each cathode is not bounded, so an artificial
clipping parameter must be introduced.

To determine the trigger threshold $E_T$ for a single cathode wire in a given plane, the ratio of
events with signal above the software trigger threshold for that wire
to the total number of electron events identified by that detector is calculated as
\begin{equation}
  R_{T} = \frac{N_T}{N_{ALL}},
\end{equation}
where $N_T$ indicates ``trigger'' events and $N_{ALL}$ refers to the total number of electron events.

Then for the corresponding simulation of this wirechamber cathode wire, the trigger threshold $E_T$
is applied starting at $E_T=0.01~keV$ and the same ratio as above is calculated for the simulation,
$R^{\mathrm{sim}}_{T}$. The
threshold is incrementally increased by $0.01$~keV and the ratio is recalculated until
$R^{\mathrm{sim}}_{T} = R_T$. The value for $E_T$ is saved for application within the new simulation
model.

Now with the knowledge of the low energy threshold, a similar method for the high energy clipping
threshold $E_C$ can be applied for this wire. The ratio of events from data becomes
\begin{equation}
  R_{C} = \frac{N_C}{N_{T}},
\end{equation}
so that this is now the ratio of clipped events to triggered events for the wire of interest.
Then within the simulation, the clipping threshold can be scanned down from an arbitrarily high
threshold until $R^{\mathrm{sim}}_{C} = R_C$. It was found that starting the 
the clipping threshold at $E_C=9$~keV and incrementing by $-0.1$~keV provided nice enough
agreement while not using exceptionally high computation times, as this process is carried out
for every wire grouping (64/run) in every $\beta$-decay run (\~1000).

\begin{figure}[h]
  \centering
  \begin{tabular}{cc}
  \subfloat[Multiplicity]{\includegraphics[scale=0.3]{3-UCNAAnalysis/ImageHolder.pdf}}  &
  \subfloat[Number of clipped wires]{\includegraphics[scale=0.3]{3-UCNAAnalysis/ImageHolder.pdf}}
  \end{tabular}
  \caption{Comparisons between data (blue open squares) and simulation (red closed circles) after
    the application of the single wire trigger and clipping model.}
  \label{fig:multDataSim}
\end{figure}


\subsubsection{Results of individual wire model}

The agreement between data and simulation from application
of the trigger threshold can be seen in figure \ref{fig:multDataSim} a.), where
the multiplicity of triggered wires for both data and simulation is shown. Without the
the application of a nonzero individual trigger threshold, the multiplicity would
generally be much higher for the simulation.


\begin{figure}[h]
  \centering
  \includegraphics[scale=0.38]{3-UCNAAnalysis/ImageHolder.pdf} 
  \caption{Position dependence of clipped events for data (blue open squares) and simulation (red closed circles) after
    the application of the single wire trigger and clipping model.}
  \label{fig:clippedPos}
\end{figure}

The application of the clipping threshold to the simulation also provides nice agreement
as seen in figure \ref{fig:multDataSim} b.). More importantly, the position dependence
of the clipped events is properly accounted for in the simulation as shown in figure
\ref{fig:clippedPos}.

With these effects accounted for within the simulation, effects regarding position
reconstruction should be accounted for within the systematic corrections. Any subsequent
MWPC systematic effects from the efficiency of the MWPC trigger described in section
\ref{sssec:mwpctrigg} is accounted for separately as will be shown in section \ref{sssec:mwpcEff}.

%The disagreement
%in the zero bin is attributed to the possibility of the MWPC generating a trigger given the
%summed max cathode trigger cut from section \ref{sssec:mwpctrigg} while having no signal above
%threshold
%----------------------------------------------------


\section{Position Dependent Light Transport Maps}

As mentioned in the experimental description, each PMT is coupled to a quadrant of
the scintillator and collects the most light from this quadrant. The light collection
is therefore position dependent, and an individual PMT will receive a different
amount of light for an event of energy $E_i$ depending on where that event strikes
the scintillator. To properly map the PMT signal to energy, this position dependence
must be accounted for on a PMT-by-PMT basis. The collection of values which correct for this
dependence will appropriately be called position maps from here on.


\subsection{Activated Xenon}

To map the position response of the scintillator, signals must be present across the
entire face of the detector. The $\beta$-decay spectrum
is an obvious option, and was used for these position maps prior to 2010,
but the event rate is low when divided into small position bins across the scintillator.
Prior to running in 2010, a method using activated xenon was developed to provide
a higher event rate and also full fiducial coverage.

The xenon is activated by placing a small amount of natural xenon in the volume that
normally holds the $\mathrm{SD}_2$ source, freezing it, and then exposing it to the moderated
neutron flux for several minutes. This produces a plethora of radioactive isotopes
with various half-lives. The xenon is then warmed up to a gas and stored. The gas
is then released into spectrometer during position mapping periods, and the decay products
are detected \cite{mpmThesis}. The various radioactive isotopes provide several features to fit across the entire
detector surface.

\begin{figure}[h] 
\centering
\includegraphics[scale=.5]{3-UCNAAnalysis/xenonSpectrum.pdf}
\caption{Example energy spectrum of the neutron-activated xenon used for position map determination. }
\label{fig:xenonSpectrum}
\end{figure}

The spectral shape of the activated xenon changes with time due to the different half-lives
of the isotopes, but this is not a concern as the PMT will see the same shape at all positions
across the detector, just with different ADC scales due to the position dependence of the light
collection. Therefore one only needs to choose a feature of the spectrum to fit in different
positions to map the relative response.

\subsection{Position Maps}

The scintillator is divided into a grid of $5\times5\mathrm{~mm}^2$ squares (in the 1~T
decay trap coordinates) with one square directly in the center,
and the xenon events are collected for each of these ``pixels''. A key feature
from the spectrum is then chosen and fit in every pixel, with the position dependent response
factor in pixel $i$ for a single PMT defined as
%
\begin{equation}
  \eta_i = \frac{Q_i}{Q_0},
\end{equation}
%
where $Q_i$ is the fitted ADC value of the feature in pixel $i$ and $Q_0$ is the fitted ADC value of the
feature in the center pixel. This normalizes the position response for a PMT to the center pixel. The
position response at some position $(x,y)$, referred to as the continuous variable $\eta(x,y)$,
is then calculated via a two-dimensional Catmull-Rom
cubic interpolating spline \cite{catmull1974}. The same interpolation is used in the plots of the
position dependence in figure \ref{fig:posmaps}.


\begin{figure}[hp] 
\centering
\subfloat[East Detector]{\includegraphics[page=1,scale=.55]{3-UCNAAnalysis/position_map_4_5mm_endpoint.pdf}} \\
\subfloat[East Detector]{\includegraphics[page=2,scale=.55]{3-UCNAAnalysis/position_map_4_5mm_endpoint.pdf}}
\caption{Typical set of position maps from a xenon position mapping period. The position maps
  remain fairly constant throughout the two run periods.}
\label{fig:posmaps}
\end{figure}

\begin{figure}[h] 
\centering
\includegraphics[scale=.55]{3-UCNAAnalysis/posmapComp_4_5mm_endpoint_vs_peak.pdf}
\caption{Ratio of $\eta_{\mathrm{peak}} / \eta_{\mathrm{endpoint}}$ for the East side PMTs. The differences
  are most pronounced in areas of high and low light collection. The West side shows more
  consistency when comparing maps calculated using the different features.}
\label{fig:posmapCompare}
\end{figure}

A typical xenon energy spectrum can be seen in figure \ref{fig:xenonSpectrum}. The two obvious features one could
fit are the peak between 100~keV and 200~keV or the 915~keV $\beta$-decay endpoint. The peak is a
superposition of several isotopes, while the endpoint comes from the
$^{135}\mathrm{Xe~}\frac{3}{2}+$ isotope. The position maps are fit in terms of pedestal and gain corrected ADC,
so unfortunately the luxury of knowing an initial guess for the feature position is not afforded as it would
be if the fit was done in the energy domain. This makes
fitting the peak more reliable upon first inspection, as the endpoint fit (done via a Kurie plot as described
in section \ref{sssec:endpoint}) is more sensitive to the range of the fit especially when the spectrum is not
purely a $\beta$-spectrum. There is a problem with using the peak though, as areas with low light collection
for a given PMT lose a portion of the peak below the trigger threshold. This changes the feature shape
compared to regions of higher light output, thus biasing the position map. The better choice is then the
$^{135}\mathrm{Xe~}\frac{3}{2}+$ endpoint. 

The problem with fitting the endpoint in pixels with different light collection efficiencies is illustrated by
imagining that every pixel sees the same xenon energy spectrum (as in figure \ref{xenonSpectrum}), but that
the spectrum is compressed or stretched when compared to the spectrum in the center pixel depending on where
the pixel is located. Since each pixel has the same ADC range, choosing the proper fit range becomes difficult
as it is different in every pixel. To avoid this issue, a secondary feature was derived to be used as a
seed to the endpoint fit range. This secondary feature is calculated by first fitting the peak with a Gaussian
and extracting the mean ($\mu$) and sigma ($\sigma$), and then calculating the average ADC value, $\xi$, of the spectrum
from $\mu+1.5\sigma$ and beyond. Then the endpoint fit is done over the range $(\xi,2\xi)$. The range used to
calculate $\xi$ and the range over which the endpoint were fit were determined via trial and error, and produce
consistent results across the entire detector.

It should also be noted that the position maps were calculated using both the peak and the endpoint as the key
feature, and the differences are small. This is illustrated in figure \ref{fig:posmapCompare}, where the
ratio of the two methods is plotted.



%----------------------------------------------------------


\section{Calibration Overview}

In the next chapter, we will take an in depth look at the calibrations of both
the wirechamber and the scintillator. Here we simply highlight their use in the analysis.

\subsection{PMT Calibration}

The PMT calibration uses well known conversion electron sources from table \ref{tab:conversionSources}
to understand the detector response to these well-defined energies. By comparing the detector
response (coupling both ADC values and position map values) to simulated responses,
we can map ADC signals from each PMT to visible energies deposited
in the scintillator. These calibrations parameterize the detector response so that the
functional form of the calibration can be applied to the $\beta$-decay data. 

\subsection{Wirechamber Calibration}

The wirechamber calibration maps the anode signal in the MWPC to an energy deposited in the MWPC. This is
primarily only useful for separating the Type 2/3 backscattering events as the
energy deposition in the MWPC is not used within the reconstruction of an electron's initial
energy. This separation is important though as it drastically reduces the systematic
correction for these backscattering events.
 

\section{Polarimetry} \label{sec:polarimetry}

The polarimatry analysis was carried out Eric Dees of North Carolina State University and
deserves the attention of an entire dissertation itself. The previous polarimetry measurement
is described in detail in a dissertation by Adam Holley \cite{holley2012ultracold}. A major
difference between the previous depolarization measurement and the current depolarization
hinges upon the installation of the shutter between the spin flipper and decay trap. An
update of the new polarimetry measurement method can be found in the publication of the result
presented within this dissertation \cite{brown2017}.

\begin{figure}[h] 
\centering
\includegraphics[scale=.55]{3-UCNAAnalysis/Switcher_signals.pdf}
\caption{Figure courtesy of E. Dees and A. R. Young as published in \cite{brown2017}.
  ``Switcher signal as a function of time, during ``D''-type runs: (1) the shutter
  closes and the switcher state changes, permitting UCN in the
  guide outside the decay volume to drain to the switcher UCN detector, (2) the AFP
  spin-flipper changes state, allowing depolarized neutrons in the guides outside the
  decay volume to drain to the switcher, (3) the shutter opens, permitting depolarized
  neutrons within the decay volume to drain to the switcher detector, (4) the AFP
  spin-flipper returns to its initial state, allowing the initially loaded spin state
  to drain from the decay volume, (5) backgrounds are measured after the UCN
  population in the decay volume has drained away.  The presented data were taken in 2011
  and UCN loaded into the decay volume with the spin-flipper off.''}
\label{fig:switcherSignal}
\end{figure}

Here is a brief description of the polarimetry measurement.
The polarimetry measurement determines the average neutron polarization in each spin state using
the depolarization runs that follow every $\beta$-decay run. As detailed in section
\ref{sec:polarization}, the neutrons are initially polarized by traversing the 7~T primary
polarizing magnet, and then the desired spin is chosen using the AFP spin flipper.

After a $\beta$-decay run, the depolarization run begins by closing the shutter and changing the switcher
state to allow
the UCN to flow into the switcher detector for measurement of UCN populations. The switcher signal
as a function of time can be found in figure \ref{fig:switcherSignal}. The guides are cleaned of UCN while
the UCN population in the trap remains contained. Then the spin flipper state is changed allowing the
neutrons that were trapped between the 7~T magnet and the decay trap to pass back through the high
field region (these trapped neutrons result from UCN who underwent an unwanted spin flip after passing
the high field region, subsequently keeping them passing the 7~T field). Now that the guides are cleaned,
the shutter is opened and all UCN with spins not equal to the desired loaded spin are emptied (remember that
the spin flipper is switched from its original state). Once this population has been measured in the
switcher detector, the spin-flipper state is again changed back to its original state, and the
properly polarized UCN population is measured in the switcher detector. This process is followed by a short
background measurement in the switcher detector.

The ratio of these populations is then extrapolated back to the equilibrium population as established
during $\beta$-decay running by utilizing further \textit{ex situ} measurements and comparisons with
simulation. The results of these measurements as well as the correction to the uncertainty resulting
from imperfect polarization ($P<1$) can be found in section \ref{ssec:polCorr}.






\copyrightnotice

% Chapter 4
%\chapter{UCNA Calibrations}
\chapter{UCNA Calibrations}
\label{ch:UCNA_Calibrations}
%%%%%%%%%%%%%%%%%%%%%%%%%%%%%%%%%%%%%%%%%%%%%%%%%%%%%%%%%%%%%%%%%%%%%%%%%%%%%%%
%%%%%%%%%%%%%%%%%%%%%%%%%%%%%%%%%%%%%%%%%%%%%%%%%%%%%%%%%%%%%%%%%%%%%%%%%%%%%%%
%%%%%%%%%%%%%%%%%%%%%%%%%%%%%%%%%%%%%%%%%%%%%%%%%%%%%%%%%%%%%%%%%%%%%%%%%%%%%%%

Detector calibrations are a beautiful combination
of simulation and data manipulation which allows one to extract the energy of
an event based solely on some electronic signal. Imagine a baseball pitcher
throwing a fastball into a sheet, and the observer behind the sheet must
determine the velocity of the ball from only seeing the impression the pitch
made on the sheet. This is the task every nuclear physics experiment is faced
with, only the baseball is a particle and the sheet is our detector system.
Below we focus on the energy calibration of our apparatus.

This chapter will begin by discussing the position reconstruction algorithm,
which relies on signals from the wirechamber (MWPC). Then we will describe
the energy calibration of the scintillators which utilizes the
position of the events, as each individual PMT response is highly position
dependent. Last of all, an energy calibration of the wirechamber will be
discussed, which, while not imperative to the analysis, provides the ability to
distinguish between a Type 2 and Type 3 backscattering event more
effectively.

%--------------------------------------------------------------


%----------------------------------------------------------

\section{Wirechamber Position Reconstruction}

\subsection{Wirechamber signals}

The wirechamber signals for each detector include a summed anode signal, and two cathode signal
collections consisting of 16 individual ``wire'' readings. Again we use the term
wire loosely, as there are technically 64 wires in a plane, but they are read out in groups
of four. The position of each wire group is taken to be the midpoint of the grouping (between
the two center wires). The two sets of cathode signal collections come from perpendicular
planes, so the position can be reconstructed along both the $x-$ and $y-$axis.

The signal is read out by a peak sensing ADC, and so the maximum value of the signal is recorded
for each event. The pedestal is determined using the bismuth pulser events (different from the
PMT pedestals) and is subtracted from the ADC signal for every event. For the cathodes, a software
threshold for each wire is set at 100~channels above pedestal, so only wires above this threshold
will be used in the position reconstruction.

\subsubsection{Wirechamber trigger} \label{sssec:mwpctrigg}

As mentioned in sections \ref{sec:ExpMWPC} and \ref{sec:backscattering}, a wirechamber
software trigger is set and used to eliminate the gamma background and to identify different
backscattering events. One could use either the summed anode signal or a signal formed from
the cathode signals. The best trigger is one that separates the pedestal furthest from the
triggering data. For this analysis, this was found to be the sum of the maximum cathode
signals from the two wire planes. (Add figure of the two signals?)

\subsubsection{Cathode wire clipping}

Low energy event deposit a substantial amount of energy in the MWPC, and thus create large
signals in one or more wires in cathode planes. This can create a signal which is beyond the
range of the ADC, thus creating an overflow event which is read out as the max value of the ADC.
These events are called ``clipped'' events. A plot of the typical ratio of clipped events and the
number of clipped wires can be seen in Figure \ref{fig:nClipped}.

\begin{figure}[h]
  \centering
  \includegraphics[scale=0.6,page=1]{4-UCNACalibrations/mwpc_position.pdf} 
  \caption{Number of clipped wires in the x-plane of the East detector. Most of the events
    (\~92~\%) exhibit no clipping.}
  \label{fig:nClipped}
\end{figure}


A clipped event poses a problem in the position reconstruction because the true signal
is no longer known, only that it was above the maximum ADC value. Using this wire in
the position reconstruction will not properly account for the strength of the
signal at the position of the clipped wire.

\subsection{Wirechamber position reconstruction}

Determining the position of the spiraling electrons is vital for every component of the analysis
that follows. Before now, the gain, pedestals, and trigger thresholds are determined with no
knowledge of the position of an event, but rather basic trigger logic. As will be seen in
the Section \ref{sec:posmaps}, the actual response of the PMT is position dependent, and so we introduce the
position reconstruction algorithm now.

\subsubsection{Events with no cathode clipping}

The algorithm developed for reconstructing the positions of the events for this analysis
is meant to be as simple of possible, with more complex measures taken for special events.
If we have an event that passes through the MWPC, creates a software trigger, and has no
clipped wires, then the position is calculated as the average position of the signals:
%
\begin{equation}
  \bar{x} = \frac{\sum q_i x_i}{q_i},
\end{equation}
%
where the sum runs over wires above the individual pedestal subtracted threshold, $q_i$ are the
pedestal subtracted wire signals, and $x_i$ are the positions of the wires.  This can
be interpreted either as a weighted average of the wire positions with the weights equal to
the wire signals, or it can be seen as a simple average where each unit of the signal is seen
as an event entering into the average with a value equal to the wire position. The final position
of the event is then given by $(\bar{x}, \bar{y})$, where $\bar{y}$ is calculated in the same manner.

\begin{figure}[h]
  \centering
  \begin{tabular}{cc}
    \subfloat[East Detector]{\includegraphics[scale=0.37,page=2]{4-UCNACalibrations/mwpc_position.pdf}} &
    \subfloat[West Detector]{\includegraphics[scale=0.37,page=3]{4-UCNACalibrations/mwpc_position.pdf}}
    \end{tabular}
    \caption{Wirechamber reconstructed positions for all electron events in a single octet of
      $\beta$-decay data. The black dashed line indicates the 50~mm fiducial cut applied to the
      data during asymmetry extraction.}
  \label{fig:wirechamberPos}
\end{figure}

For well-behaved wires (no clipping, no missing or ``dead'' wire segments), this method produces
a continuous reconstruction of the events based on where they pass through the wirechamber.
An example of the position dependence can be seen in Figure \ref{fig:wirechamberPos}.

\subsubsection{Events with cathode clipping or missing wire signals}

In the event of a clipped wire or the rare case of a missing wire signal, the basic average method
shown above will not work. Using the clipped wire in the average can systematically shift the
reconstructed position if the signal is not symmetric about the clipped wire, and the same can
be said of a missing signal. While this may not appear a visible distortion to the
position distribution, the method
loses its integrity.

To handle these types of events, a method similar to that used in the previous analysis
\cite{mpmThesis} was adopted. By assuming that the wirechamber charge cloud (or
ionization cloud) takes a Gaussian shape in the MWPC, then we can expect that the signals
in the wires should also sit on a Gaussian given by
%
\begin{equation}
  q = Ne^{(x-\bar{x})^2/\sigma^2},
\end{equation}
%
where $q$ is the signal and $x$ is the position of the signal. If we take the log of this
expression we have
%
\begin{align} \label{eq:gaussPos}
   &\ln(q) = \ln(N) + \frac{(x-\bar{x})^2}{\sigma^2} \\
   &\ln(q) = \frac{1}{\sigma^2}x^2 - \frac{2\bar{x}}{\sigma^2}x + \Big(\ln(N)+\frac{\bar{x}^2}{\sigma^2}\Big) \\
   &\ln(q) = Ax^2 + Bx + C 
\end{align}
%
with
\begin{equation}
  A = \frac{1}{\sigma^2}, \qquad B = - \frac{2\bar{x}}{\sigma^2}, \qquad C = \ln(N)+\frac{\bar{x}^2}{\sigma^2}.
\end{equation}
This equation has three unknowns, thus with three ``good'' wires giving three points on the Gaussian, the unknowns
are fully determined. The three chosen wires are the maximum non-clipped wire and the next two largest signals
surrounding this maximum signal. These two wires can be on one side of the maximum wire, as would be the case if the
maximum wire were the edge wire. The extraction of the mean is the approximation of the center of the event.

\begin{figure}[h]
  \centering
  \includegraphics[scale=0.5,page=7]{4-UCNACalibrations/mwpc_position.pdf} 
  \caption{Collection of extraced $\sigma$ values for the Gaussian determination of event position. The
    average of this distribution is an estimate of the characteristic width of the wirechamber charge cloud,
    $\sigma_c$, used when only two wires are usable in the position reconstruction.}
  \label{fig:meanSigma}
\end{figure}

Now there is also the special case where there are only two usable wires above the individual cathode
wire threshold. If they are consecutive wires, the average method is applied, but if they are not consecutive wires
(so they are separated by a clipped wire(s)), a new approximation must be applied. The expression in
equation \ref{eq:gaussPos} requires three wire signals to estimate a position, so we must eliminate one of the unknowns
to use this method. Again, if we assume that all signals follow a typical Gaussian shape, then the width of the
charge cloud should be roughly constant. Thus we can determine a characteristic width, $\sigma_c$, from the
mean extracted $\sigma$ from the rest of the events (including the non-clipped events). An histogram of the
extracted $\sigma$ can be seen in Figure \ref{fig:meanSigma}. A value of $\sigma_c = 6\text{~mm}$ was used
for these type events, and then the mean was calculated by solving \ref{eq:gaussPos} with two unknowns.

Last of all, there is the situation where only one wire was above the individual threshold. For this type of event,
the only reasonable choice is to set the position of the event to the position of this wire.

\subsection{Simulation of wirechamber positions}

In order to reduce any unforeseen systematic effects, an attempt at including
every aspect of the experiment within the simulated data is made. Thus for the wirechamber
position reconstruction, we would like to employ the above methods for every event. 

Embedded within the simulation by M. Mendenhall during the previous analysis
is a model for the charge collection within the wirechamber based on the work
from \cite{mathieson1991induced}. In summary, based on the wirechamber geometry, an estimate
of the charge cloud as seen by the cathode and anode can be calculated for an event
that deposits energy $E_{\mathrm{MWPC}}$ in the wirechamber. The agreement between data and simulation was
shown to be good (\cite{mpmThesis} section 6.3.6), and thus the charge collection
model was used in this analysis.

A new contribution to the model is the application of the observed wirechamber thresholds
and clipping conditions to each of the cathode wires. The model already included is simply an estimation of the charge
collection on the range from $0$ to $\infty$. This could be used in the simple average method for ``good'' events
with no clipping and no missing wires, but then any systematic effects from wire clipping would go unnoticed.

\subsubsection{Wire model}

For each wire in each plane, the response can be characterized by two values: a clipping threshold and a trigger threshold.
Recall that we apply an individual software trigger threshold of 100 ADC channels for each wire, so a model parameter must
be determined for this. Also, in the simulation model, the signal on each cathode is not bounded, so an artificial
clipping parameter must be introduced.

To determine the trigger threshold $E_T$ for a single cathode wire in a given plane, the ratio of
events with signal above the software trigger threshold for that wire
to the total number of electron events identified by that detector is calculated as
\begin{equation}
  R_{T} = \frac{N_T}{N_{ALL}},
\end{equation}
where $N_T$ indicates ``trigger'' events and $N_{ALL}$ refers to the total number of electron events.

Then for the corresponding simulation of this wirechamber cathode wire, the trigger threshold $E_T$
is applied starting at $E_T=0.01~keV$ and the same ratio as above is calculated for the simulation,
$R^{\mathrm{sim}}_{T}$. The
threshold is incrementally increased by $0.01$~keV and the ratio is recalculated until
$R^{\mathrm{sim}}_{T} = R_T$. The value for $E_T$ is saved for application within the new simulation
model.

Now with the knowledge of the low energy threshold, a similar method for the high energy clipping
threshold $E_C$ can be applied for this wire. The ratio of events from data becomes
\begin{equation}
  R_{C} = \frac{N_C}{N_{T}},
\end{equation}
so that this is now the ratio of clipped events to triggered events for the wire of interest.
Then within the simulation, the clipping threshold can be scanned down from an arbitrarily high
threshold until $R^{\mathrm{sim}}_{C} = R_C$. It was found that starting the 
the clipping threshold at $E_C=9$~keV and incrementing by $-0.1$~keV provided nice enough
agreement while not using exceptionally high computation times, as this process is carried out
for every wire grouping (64/run) in every $\beta$-decay run (\~1000).

\begin{figure}[h]
  \centering
  \begin{tabular}{cc}
  \subfloat[Multiplicity]{\includegraphics[scale=0.3,page=5]{4-UCNACalibrations/mwpc_position.pdf}}  &
  \subfloat[Number of clipped wires]{\includegraphics[scale=0.3,page=4]{4-UCNACalibrations/mwpc_position.pdf}}
  \end{tabular}
  \caption{Comparisons between data (blue line) and simulation (red dashed line) after
    the application of the single wire trigger and clipping model. There is a discrepancy in the multiplicity,
    but the general features are captured. The number of clipped wires in data and simulation is in better agreement.}
  \label{fig:multDataSim}
\end{figure}


\subsubsection{Results of individual wire model}

The agreement between data and simulation from application
of the trigger threshold can be seen in Figure \ref{fig:multDataSim} a.), where
the multiplicity of triggered wires for both data and simulation is shown. Without the
the application of a nonzero individual trigger threshold, the multiplicity would
generally be much higher for the simulation.


\begin{figure}[h]
  \centering
  \includegraphics[scale=0.6,page=6]{4-UCNACalibrations/mwpc_position.pdf} 
  \caption{Position dependence of clipped events for data (blue line) and simulation (red dashed line) after
    the application of the single wire trigger and clipping model to the simulation. The model captures the
    position dependence of the clipping nicely.}
  \label{fig:clippedPos}
\end{figure}

The application of the clipping threshold to the simulation also provides nice agreement
as seen in Figure \ref{fig:multDataSim} b.). More importantly, the position dependence
of the clipped events is properly accounted for in the simulation as shown in figure
\ref{fig:clippedPos}.

With these effects accounted for within the simulation, effects regarding position
reconstruction should be accounted for within the systematic corrections. Any subsequent
MWPC systematic effects from the efficiency of the MWPC trigger described in section
\ref{sssec:mwpctrigg} is accounted for separately as will be shown in Section \ref{sssec:mwpcEff}.

%The disagreement
%in the zero bin is attributed to the possibility of the MWPC generating a trigger given the
%summed max cathode trigger cut from Section \ref{sssec:mwpctrigg} while having no signal above
%threshold
%----------------------------------------------------






\section{Scintillator Energy Calibration}

\subsection{Method}
Talk about the parameters needed (resolution, trigger thresholds, pedestals, position maps, etc.)
Talk about the cyclical nature.

\subsection{Position Dependent Light Transport Maps} \label{ssec:posmaps}

As mentioned in the experimental description, each PMT is coupled to a quadrant of
the scintillator and collects the most light from this quadrant. The light collection
is therefore position dependent, and an individual PMT will receive a different
amount of light for an event of energy $E_i$ depending on where that event strikes
the scintillator. To properly map the PMT signal to energy, this position dependence
must be accounted for on a PMT-by-PMT basis. The collection of values which correct for this
dependence will appropriately be called position maps from here on.


\subsubsection{Activated Xenon}

To map the position response of the scintillator, signals must be present across the
entire face of the detector. The $\beta$-decay spectrum
is an obvious option, and was used for these position maps prior to 2010,
but the event rate is low when divided into small position bins across the scintillator.
Prior to running in 2010, a method using activated xenon was developed to provide
a higher event rate and also full fiducial coverage.

The xenon is activated by placing a small amount of natural xenon in the volume that
normally holds the $\mathrm{SD}_2$ source, freezing it, and then exposing it to the moderated
neutron flux for several minutes. This produces a plethora of radioactive isotopes
with various half-lives. The xenon is then warmed up to a gas and stored. The gas
is then released into spectrometer during position mapping periods, and the decay products
are detected \cite{mpmThesis}. The various radioactive isotopes provide several features to fit across the entire
detector surface.

\begin{figure}[h] 
\centering
\includegraphics[scale=.5]{4-UCNACalibrations/xenonSpectrum.pdf}
\caption{Example energy spectrum of the neutron-activated xenon used for position map determination. }
\label{fig:xenonSpectrum}
\end{figure}

The spectral shape of the activated xenon changes with time due to the different half-lives
of the isotopes, but this is not a concern as the PMT will see the same shape at all positions
across the detector, just with different ADC scales due to the position dependence of the light
collection. Therefore one only needs to choose a feature of the spectrum to fit in different
positions to map the relative response.

\subsubsection{Position Maps} \label{sssec:posmaps}

The scintillator is divided into a grid of $5\times5\mathrm{~mm}^2$ squares (in the 1~T
decay trap coordinates) with one square directly in the center,
and the xenon events are collected for each of these ``pixels''. A key feature
from the spectrum is then chosen and fit in every pixel, with the position dependent response
factor in pixel $i$ for a single PMT defined as
%
\begin{equation}
  \eta_i = \frac{Q_i}{Q_0},
\end{equation}
%
where $Q_i$ is the fitted ADC value of the feature in pixel $i$ and $Q_0$ is the fitted ADC value of the
feature in the center pixel. This normalizes the position response for a PMT to the center pixel. The
position response at some position $(x,y)$, referred to as the continuous variable $\eta(x,y)$,
is then calculated via a two-dimensional Catmull-Rom
cubic interpolating spline \cite{catmull1974}. The same interpolation is used in the plots of the
position dependence in Figure \ref{fig:posmaps}.


\begin{figure}[hp] 
\centering
\subfloat[East Detector]{\includegraphics[page=1,scale=.55]{4-UCNACalibrations/position_map_4_5mm_endpoint.pdf}} \\
\subfloat[East Detector]{\includegraphics[page=2,scale=.55]{4-UCNACalibrations/position_map_4_5mm_endpoint.pdf}}
\caption{Typical set of position maps from a xenon position mapping period. The position maps
  remain fairly constant throughout the two run periods.}
\label{fig:posmaps}
\end{figure}

\begin{figure}[h] 
\centering
\includegraphics[scale=.55]{4-UCNACalibrations/posmapComp_4_5mm_endpoint_vs_peak.pdf}
\caption{Ratio of $\eta_{\mathrm{peak}} / \eta_{\mathrm{endpoint}}$ for the East side PMTs. The differences
  are most pronounced in areas of high and low light collection. The West side shows more
  consistency when comparing maps calculated using the different features.}
\label{fig:posmapCompare}
\end{figure}

A typical xenon energy spectrum can be seen in Figure \ref{fig:xenonSpectrum}. The two obvious features one could
fit are the peak between 100~keV and 200~keV or the 915~keV $\beta$-decay endpoint. The peak is a
superposition of several isotopes, while the endpoint comes from the
$^{135}\mathrm{Xe~}\frac{3}{2}+$ isotope. The position maps are fit in terms of pedestal and gain corrected ADC,
so unfortunately the luxury of knowing an initial guess for the feature position is not afforded as it would
be if the fit was done in the energy domain. This makes
fitting the peak more reliable upon first inspection, as the endpoint fit (done via a Kurie plot as described
in Section \ref{sssec:endpoint}) is more sensitive to the range of the fit especially when the spectrum is not
purely a $\beta$-spectrum. There is a problem with using the peak though, as areas with low light collection
for a given PMT lose a portion of the peak below the trigger threshold. This changes the feature shape
compared to regions of higher light output, thus biasing the position map. The better choice is then the
$^{135}\mathrm{Xe~}\frac{3}{2}+$ endpoint. 

The problem with fitting the endpoint in pixels with different light collection efficiencies is illustrated by
imagining that every pixel sees the same xenon energy spectrum (as in Figure \ref{fig:xenonSpectrum}), but that
the spectrum is compressed or stretched when compared to the spectrum in the center pixel depending on where
the pixel is located. Since each pixel has the same ADC range, choosing the proper fit range becomes difficult
as it is different in every pixel. To avoid this issue, a secondary feature was derived to be used as a
seed to the endpoint fit range. This secondary feature is calculated by first fitting the peak with a Gaussian
and extracting the mean ($\mu$) and sigma ($\sigma$), and then calculating the average ADC value, $\xi$, of the spectrum
from $\mu+1.5\sigma$ and beyond. Then the endpoint fit is done over the range $(\xi,2\xi)$. The range used to
calculate $\xi$ and the range over which the endpoint were fit were determined via trial and error, and produce
consistent results across the entire detector.

It should also be noted that the position maps were calculated using both the peak and the endpoint as the key
feature, and the differences are small. This is illustrated in Figure \ref{fig:posmapCompare}, where the
ratio of the two methods is plotted.


\subsection{Linearity Curves}
\subsection{PMT Resolution Factors} \label{ssec:PMTresolution}
\subsection{MC to Data Agreement}


%------------------------------------------------------------------------

\section{Wirechamber Energy Calibration}

\subsection{Method}

The goal of the wirechamber energy calibration is to provide an estimate of
the energy deposited within the MWPC, $E_{\mathrm{MWPC}}$, depending on the
response of the wirechamber. The available information from the wirechamber includes the
summed anode signal, which is proportional to the total amount of ionization
within the MWPC gas and thus the total energy deposited, and the reconstructed position from the cathode signals.
Also available is the reconstructed initial energy of all events, $E_{\mathrm{recon}}$, as a result of the
previously described scintillator energy calibration. This is important as higher energy events deposit
less energy in the wirechamber than lower energy events, and thus subsets of events with
different $E_{\mathrm{recon}}$ will provide different distributions of $E_{\mathrm{MWPC}}$.
With this in mind, the general idea for calibrating the wirechamber is as follows:

\begin{itemize}
\item Map out the position dependence of the anode signals in a similar manner to the
  PMT light transport maps.
\item Separate data into subsets based on $E_{\mathrm{recon}}$
  to provide several different distributions of wirechamber signals (anode for data and $E_{\mathrm{MWPC}}$
  for simulation).
\item For each $E_{\mathrm{recon}}$ subset, fit the wirechamber ADC signals for the data.
\item For each $E_{\mathrm{recon}}$ subset, fit the simulated wirechamber deposited energy, $E_{\mathrm{MWPC}}$.
\item Plot the extracted ADC values from the data vs. the extracted $E_{\mathrm{MWPC}}$ from the simulation to
  determine the parameterization from anode signal to wirechamber deposited energy.
\end{itemize}

The energy calibration of the wirechamber is carried out for every individual $\beta$-decay run,
rather than using the periodic
source calibration runs as was done for the scintillator energy calibration. This is advantageous as
it automatically corrects for any changes in the anode gain on the run-by-run time scale. This will
be illustrated in the coming sections.


\subsection{Position dependence of anode signal}

While the wirechamber is constructed to be as homogeneous as possible, there is still a residual
position dependence to the anode signal. The cause of such position dependence is undetermined,
but it only affects the energy calibration of the wirechamber, which is
a non-essential component of the analysis. Also, by mapping the position dependence in a manner
similar to the PMT position maps (Section \ref{sssec:posmaps}, we can remove any effects altogether.

\begin{figure}[h]
  \centering
  \includegraphics[scale=0.3,page=1]{4-UCNACalibrations/ImageHolder.pdf}
  \caption{Typical wirechamber energy deposition for a xenon calibration run in a
    single pixel. The fit function is a TMath::Landau function from the ROOT Data Analysis
    Framework, and the extracted value is the MPV (most probable value) from the
    distribution.}
  \label{fig:xenonMWPCsignal}
\end{figure}

\begin{figure}[h]
  \centering
  \begin{tabular}{cc}
    \subfloat[East MWPC]{\includegraphics[scale=0.3,page=1]{4-UCNACalibrations/MWPC_posmap.pdf}}  &
  \subfloat[West MWPC]{\includegraphics[scale=0.3,page=2]{4-UCNACalibrations/MWPC_posmap.pdf}}
  \end{tabular}
  \caption{Example position dependent anode response for each MWPC. These MWPC position maps were
    created using electrons with $300\mathrm{~keV}<E_{\mathrm{recon}}<350\mathrm{~keV}$.}
  \label{fig:mwpcPosMap}
\end{figure}

The dedicated xenon calibration runs were used to determine the MWPC position maps.
The wirechamber anode signal was histogrammed in 5~mm~$\times$~5~mm position bins (pixels) across the detector,
giving distributions like the one seen in Figure \ref{fig:xenonMWPCsignal}. The distribution
in each pixel is fit with a TMath::Landau function from the ROOT Data Analysis Framework.
The most probable value of
the distribution is returned from the fit for every pixel, and then the position dependent response factor is formed
via
%
\begin{equation}
  \eta^{\mathrm{MWPC}}_i = \frac{\mathrm{MPV}_i}{\mathrm{MPV}_0},
\end{equation}
where $\mathrm{MPV}_0$ is the most probably value of the center pixel.
The continuous values plotted in Figure \ref{fig:mwpcPosMap} are determined as they were in
Section \ref{sssec:posmaps}. The position dependent response factor is divided out of the anode signal
prior to determining the wirechamber calibration, and thus also prior to applying the wirechamber
calibration to the data.

One should note that the MWPC position maps can be determined for different initial energies of the
electrons, since the scintillator calibration has already been applied. The position response plots
in Figure \ref{fig:mwpcPosMap} were made using electrons with $300\mathrm{~keV}<E_{\mathrm{recon}}<350\mathrm{~keV}$,
although the position maps are stable for any 50~keV window chosen. Thus for ease of application, the MWPC position
maps from the above energy range were used for electrons of all initial energies.


\subsection{Relating Anode Signal to $E_{\mathrm{MWPC}}$}

As mentioned above, the wirechamber energy calibration is carried out on a run-by-run basis, so
the discussion that follows applies to every individual $\beta$-decay run.
After dividing out the MWPC position dependent response from the anode signals of all electron events,
we can attempt to relate the anode response to the expected wirechamber response from simulation.
To do this, we first separate the electron events by their $E_{\mathrm{recon}}$ into 50~keV energy bins beginning at 100~keV and ending at
800~keV, giving twelve separate collections of electron events. The lower bound is chosen to avoid any possible
effects from differences between the simulated and actual scintillator thresholds. The upper bound is just beyond
the endpoint energy of the $\beta$-decay electrons.

\begin{figure}[h]
  \centering
  \begin{tabular}{cc}
    \subfloat[Simulated MWPC response]{\includegraphics[scale=0.3,page=1]{4-UCNACalibrations/ImageHolder.pdf}}  &
  \subfloat[Actual MWPC response]{\includegraphics[scale=0.3,page=1]{4-UCNACalibrations/ImageHolder.pdf}}
  \end{tabular}
  \caption{Landau fit of both simulation and data for electron events which deposited between xxx-xxx keV
    in the scintillator.}
  \label{fig:landauFits}
\end{figure}

\begin{figure}[h]
  \centering
  \begin{tabular}{cc}
    \subfloat[East MWPC]{\includegraphics[scale=0.3,page=1]{4-UCNACalibrations/MWPC_cal_17126.pdf}}  &
  \subfloat[West MWPC]{\includegraphics[scale=0.3,page=2]{4-UCNACalibrations/MWPC_cal_17126.pdf}}
  \end{tabular}
  \caption{Example MWPC calibration curve for a single run in 2011-2012. A calibration like this one
  is carried out for every $\beta$-decay run.}
  \label{fig:mwpcCal}
\end{figure}

The twelve electron energy groupings provide twelve different distributions
to be fit with a Landau function, and the MPV of each should be different. This is a result
of the electrons of interest
lying in an energy range such that interactions with the wirechamber fill gas produce more energy loss
per unit length for less energetic electrons, or in other words the lower energy electrons fall below
the minimum ionizing particle energy for the fill gas. A nice summary of particle interactions with matter
can be found in \cite{pdg}, but the important takeaway from this is that the twelve different initial energy
groupings yield twelve different energy deposition distributions in the wirechamber, which provides
twelve data points to use when comparing simulation with data.

For the simulation, the MPV is extracted in the energy domain, where the energy is the
amount of energy deposited within the wirechamber gas. For the data, the MPV is an ADC value.
Examples of each fit for a single $E_{\mathrm{recon}}$ grouping can be seen in Figure \ref{fig:landauFits}.
If we
have modeled the wirechamber correctly and as long as the MWPC anode signal is proportional to
the total ionization created in the wirechamber, then we expect a linear relationship between the
simulated energy depostion and the anode signal for each of the $E_{\mathrm{recon}}$ groupings. By plotting
the MPVs from data and simulation and fitting with a straight line, we define the conversion
from (position corrected) MWPC anode signal to energy deposited within the wirechamber. An example
calibration is demonstrated in Figure \ref{fig:mwpcCal}. The calibration fit is of the form
%
\begin{equation} \label{eq:pureLin}
  E_{\mathrm{MWPC}}(\mathrm{ADC}) = C_1 + C_2\cdot \mathrm{ADC},
\end{equation}
and the fit is extrapolated to the origin in a continuous manner using
\begin{equation}
  E_{\mathrm{MWPC}} = C_3 \cdot \mathrm{ADC}^{C_4}
\end{equation}
with the parameters $C_3$ and $C_4$ determined by ensuring the two equations are equal at the
transition point along with their first derivatives. The transition is defined as 50~ADC channels
below the lowest data point.

\begin{figure}[h]
  \centering
  \begin{tabular}{cc}
    \subfloat[East MWPC]{\includegraphics[scale=0.3,page=1]{4-UCNACalibrations/ImageHolder.pdf}}  &
  \subfloat[West MWPC]{\includegraphics[scale=0.3,page=1]{4-UCNACalibrations/ImageHolder.pdf}}
  \end{tabular}
  \caption{Simulation vs. Data after application of the wirechamber calibration
  for a single run.}
  \label{fig:simVsDataMWPC}
\end{figure}

Including the $C_1$ parameter ($y-$intercept) allows for
imperfections in the MWPC anode pedestal subtraction. The extrapolation to the origin produces better agreement
with simulation, as data events with very low energy deposition in the wirechamber could otherwise be
assigned negative energies if only Equation \ref{eq:pureLin} is used and the $y-$intercept is negative.
Agreement between data and simulation for a single run after application of the wirechamber
calibration is shown in Figure \ref{fig:simVsDataMWPC}. 







\subsection{Backscattering Separation for Type 2/3 Events} \label{sssec:backscSep}
Should we put the 2/3 separation in the results chapter?

%----------------------------------------------------------







\copyrightnotice

% Chapter 5
%\chapter{UCNA Calibrations}
\chapter{UCNA Results}
\label{ch:UCNA_Results}
%%%%%%%%%%%%%%%%%%%%%%%%%%%%%%%%%%%%%%%%%%%%%%%%%%%%%%%%%%%%%%%%%%%%%%%%%%%%%%%
%%%%%%%%%%%%%%%%%%%%%%%%%%%%%%%%%%%%%%%%%%%%%%%%%%%%%%%%%%%%%%%%%%%%%%%%%%%%%%%
%%%%%%%%%%%%%%%%%%%%%%%%%%%%%%%%%%%%%%%%%%%%%%%%%%%%%%%%%%%%%%%%%%%%%%%%%%%%%%%


\section{Constructing an Asymmetry}

\subsection{Decay Rate Model}
Nominally the decay rate for polarized neutrons is expressed in terms of the spin
vector and all possible correlations to the products. For simplicity,
we begin by writing the decay rate only as a function of the
electron asymmetry: 

\begin{equation} \label{eq:simpleRate}
\Gamma\left(E,\Omega\right)=C \cdot S(E) \cdot \left[ 1+A(E)\beta\cos\theta \right],
\end{equation}

\noindent where $S(E)$ is the Standard Model decay rate for unpolarized beta decay, $C$
is a constant which encompasses anything not included in $S(E)$ and serves the 
purpose of absorbing other constants, $A(E)$ is the 
energy dependent electron asymmetry parameter, $\beta$ is the normal $v/c$, and 
$\theta$ is the angle between the spin of the neutron and the electron momentum.

We can also introduce a polarization, giving

\begin{equation}
\Gamma\left(E,\Omega\right)=C \cdot S(E) \cdot \epsilon \cdot \left[ 1+ P\cdot A(E)\beta\cos\theta \right],
\end{equation}

\noindent where $\epsilon$ is the loading efficiency of neutrons in this spin state which I haven't absorbed
into the constant for reasons to be seen, and 
$P$ is the polarization, i.e. a number between 0 and 1 (very close to 1
in our case).

At this point, assumptions can be made regarding the experimental setup utilized
to further express the decay rate in terms of detector rates. Due to the polarization
being aligned with the 1 T magnetic field in the decay trap, electrons emitted with
a momentum component along the spin will spiral towards one detector, while electrons
with a component of momentum opposite the spin will be detected in the opposite 
detector. If we fix the z-axis in our decay rate using the static position of our
apparatus, we can call the detector at $\theta=0$ detector
1 and the opposite detector 2. 

Now, all electrons with $0 < \theta < \pi/2$ will head towards detector 1 and those 
with $\pi/2 < \theta < \pi$ will be directed towards detector 2. The solid angle can
then be integrated out for each detector by integrating $\phi$ from $(0,2\pi)$ and $\cos\theta$ over the intervals given, where the integral of $\cos\theta$ over detectors
1 and 2 yields $\pm1/2$ respectively: 

\begin{equation} 
\Gamma_{1,2}\left(E\right)= 2\pi \cdot C \cdot S(E) \cdot \epsilon \cdot \eta_{1,2}(E) \left[ 1+ P\cdot A(E)\beta \left(\pm_{1,2}\frac{1}{2}\right) \right],
\end{equation} 

\begin{equation} 
\Gamma_{1,2}\left(E\right)=C' \cdot S(E) \cdot \epsilon \cdot \eta_{1,2}(E) \left[ 1 \pm_{1,2} P\cdot A(E) \frac{\beta}{2} \right],
\end{equation} 

\noindent with $\eta_{1,2}(E)$ signifying energy dependent electron detection efficiencies for detectors 1 and 2. 

Thus far we have assumed a fixed polarization, but in UCNA we have the ability to 
flip the spin of the neutrons prior to loading. We call these states flipper on ($+$) 
and flipper off ($-$), denoted by $P_{\pm}$. The flipper on state introduces a negative in front of the polarization efficiency:

\begin{equation}
\Gamma_{1,2}^{\pm}\left(E\right)=C' \cdot S(E) \cdot \epsilon_{\pm} \cdot \eta_{1,2} \left[ 1 \pm_{1,2} 
\left(\mp P_{\pm}\right) \cdot A(E) \frac{\beta}{2} \right],
\end{equation} 

\noindent where $\epsilon_{\pm}$ accounts for differing loading efficiencies for 
different spin states.

If we now assume that we are equally as good at polarizing in either spin state, we
can say $P_{+}=P_{-}=P$:

\begin{equation}
\Gamma_{1,2}^{\pm}\left(E\right)=C' \cdot S(E) \cdot \epsilon_{\pm} \cdot \eta_{1,2} \left[ 1 \pm_{1,2} 
\left(\mp P \right) \cdot A(E) \frac{\beta}{2} \right].
\end{equation}


\subsection{Super-Ratio}

A simple asymmetry over some energy bin for a single 
polarization state can be defined as

\begin{equation} 
A_{\mathrm{simple}} = \frac{\Gamma_1 - \Gamma_2}{\Gamma_1 + \Gamma_2}, 
\end{equation}

\noindent and assuming that the efficiencies $\eta_1$ and $\eta_2$ are the same one
finds

\begin{equation} \label{eq:Asimple}
A_{\mathrm{simple}} = P \cdot A(E) \cdot \frac{\beta}{2},
\end{equation}

\noindent where $\beta$ is calculated as either the velocity associated with the energy 
at the center of the bin or the average energy of the bin.   

In reality we can't assume with certainty that the detectors are identical, but we can 
utilize the fact that we can flip the spins to cancel differing efficiencies. If we 
take two runs with opposite polarizations, we can define the super-ratio asymmetry in 
some energy range as:

\begin{equation}
A_{\mathrm{SR}} = \frac{1-\sqrt{R}}{1+\sqrt{R}} ,
\end{equation}
\noindent where 
\begin{equation} \label{eq:finalA_SR}
 R = \frac{\Gamma_{1}^+ \cdot \Gamma_{2}^-}{\Gamma_{1}^- \cdot \Gamma_{2}^+}.
\end{equation}

Some algebra yields an identical expression to equation \ref{eq:Asimple},

\begin{equation*}
R = \frac{ \left[ C'  S(E)  \epsilon_{+}  \eta_{1} \left[ 1 + 
      \left(- P \right)  A(E) \frac{\beta}{2} \right] \right] 
\left[ C'  S(E)  \epsilon_{-}  \eta_{2} \left[ 1 - 
    \left(+ P \right)  A(E) \frac{\beta}{2} \right] \right] }
{ \left[ C'  S(E)  \epsilon_{-}  \eta_{1} \left[ 1 + 
      \left(+ P \right)  A(E) \frac{\beta}{2} \right] \right] 
\left[ C'  S(E)  \epsilon_{+}  \eta_{2} \left[ 1 - 
    \left(- P \right)  A(E) \frac{\beta}{2} \right] \right] },
\end{equation*}

\begin{equation*}
R = \frac{ \left[ 1 + \left(- P \right)  A(E) \frac{\beta}{2} \right] 
\left[ 1 - \left(+ P \right)  A(E) \frac{\beta}{2} \right]  }
{  \left[ 1 +  \left(+ P \right)  A(E) \frac{\beta}{2}  \right] 
 \left[ 1 -  \left(- P \right)  A(E) \frac{\beta}{2} \right]},
\end{equation*}

\begin{equation*}
R = \frac{ \left[ 1 -  P  A(E) \frac{\beta}{2} \right]^2 }
{  \left[ 1 + P  A(E) \frac{\beta}{2}  \right]^2 },
\end{equation*}

\noindent and plugging into $A_{\mathrm{SR}}$

\begin{equation*}
  A_{\mathrm{SR}} = \frac{1-\frac{ \left[ 1 -  P  A(E) \frac{\beta}{2} \right] }
{  \left[ 1 + P  A(E) \frac{\beta}{2}  \right] } }
  {1+\frac{ \left[ 1 -  P  A(E) \frac{\beta}{2} \right] }
{  \left[ 1 + P  A(E) \frac{\beta}{2}  \right] }},
\end{equation*}

\begin{equation*}
A_{\mathrm{SR}} = \frac{2 P A(E) \frac{\beta}{2}}{2},
\end{equation*}


\begin{equation}
A_{\mathrm{SR}} = P \cdot A(E) \cdot \frac{\beta}{2}.
\end{equation}

\subsection{Extracting $A_0$}

The quantity of interest isn't the raw measured asymmetry, or even $A(E)$, but rather $A_0$, which is also directly proportional to $\lambda=g_A/g_V$, yielding a direct 
measurement of the axial vector coupling constant if the CVC hypothesis is assumed. 

$A_0$ manifests itself in our measured asymmetry as:

\begin{equation}
A(E) = A_0 \cdot \left( 1 + \Delta_{\mathrm{Th}}(E) \right),
\end{equation}

\noindent where $\Delta_{\mathrm{Th}}(E)$ is the energy dependent systematic shift in the
asymmetry due to theory corrections like finite mass effects and radiative 
corrections to name a few. The energy dependence will be dropped for neatness from here on.

If one were to assume we could perfectly measure our super-ratio asymmetry, free of
systematic effects with only statistical uncertainty, we could write down from 
equation \ref{eq:finalA_SR}:

\begin{equation}
A_{\mathrm{SR}}^{\mathrm{pure}} = P \cdot A(E) \cdot \frac{\beta}{2}, 
\end{equation}

\begin{equation}
A_{\mathrm{SR}}^{\mathrm{pure}} = P \cdot A_0 \cdot \left(1 + \Delta_{\mathrm{Th}} \right) \cdot \frac{\beta}{2}, 
\end{equation}

\noindent and finally

\begin{equation}
A_0 = \frac{A_{\mathrm{SR}}^{\mathrm{pure}}}{P \cdot \left( 1+\Delta_{\mathrm{Th}} \right) \cdot \beta/2}. 
\end{equation}

Obviously our measurement of the super-ratio is not free of systematic errors, so 
this equation alone provides us with little, but we can assume that, providing our
simulation and detector response model has been properly benchmarked against data (both conversion electron and 
$\beta$-decay), we can characterize the systematic effects on a bin-by-bin basis
by comparing pure simulation super-ratios to those from processed simulation data. The 
processed simulation data refers to data which has been run through the detector 
response model and thus treated like ``real" data. If this is the case, then we can say:

\begin{equation}
A_{\mathrm{SR}}^{\mathrm{pure}} = A_{\mathrm{SR}}^{\mathrm{Exp}} \cdot \left(1+\Delta_{\mathrm{Exp}}(E) \right).
\end{equation}

At this point it's straighforward to write down $A_0$ as a function of quantities 
which are either measured directly, well defined in the theory, or determined via simulation studies as:

\begin{equation}
A_0 = \frac{A_{\mathrm{SR}}^{\mathrm{Exp}} \cdot \left(1 + \Delta_{\mathrm{Exp}} \right) }{P \cdot \left( 1 + \Delta_{\mathrm{Th}} \right) \cdot \beta/2},
\end{equation}

\noindent where the energy dependence on $\Delta_{\mathrm{Exp}}$ was also dropped for 
aesthetics. The super-ratios are calculated and systematic effects applied 
over discrete energy bins, and ultimately one can fit over a well chosen energy range to determine $A_0$.
The optimization of this energy range is discussed elsewhere.


%--------------------------------------------------------------------------------------------------
\section{Systematic Correctons}

\subsection{Extracting $\Delta_{\mathrm{Exp}}$}

\subsubsection{Determining Relative Contributions to $\Delta_{\mathrm{Exp}}$}

\subsection{$\Delta_{\mathrm{Th}}$ Contributions}

\subsection{Polarimetry}

\subsection{Energy Reconstruction}

%--------------------------------------------------------------

\section{Analysis Choices}













\copyrightnotice

% Appendix 1
%\appendix
\addappheadtotoc
\chapter{Magnetic Field Non-Uniformity and the Super-Ratio}
\label{app:MagField}

%\copyrightnotice


%-----------------------------------------------
\backmatter

\printbibliography[title={References}]
%\bibliographystyle{alpha}
%\bibliography{references}

%\addtocontents{toc}{\protect\setcounter{tocdepth}{0}}

%\setcounter{tocdepth}{0}
\chapter{Vita}

\section*{Personal Information}
\begin{tabular} {ll}
  Name: & Michael Anthony-Paul Brown \\
  Place of Birth: & Cincinnati, OH 
\end{tabular}

\section*{Educational Institutions}
\begin{tabular} {l}
  M.S. Physics    \\ University of Kentucky, Lexington, KY   \\May 2014 \\ \\
  B.S. Physics    \\ Morehead State University, Morehead, KY \\May 2011 \\ \\
  B.S. Mathematics\\ Morehead State University, Morehead, KY \\May 2011
\end{tabular}

\section*{Professional Positions}
\begin{tabular} {l}
  Graduate Research Assistant \\University of Kentucky, Lexington, KY \\
  Department of Physics and Astronomy\\
  January 2012 $-$ January 2018 \\
  \\
  Graduate Teaching Assistant \\University of Kentucky, Lexington, KY \\
  Department of Physics and Astronomy \\
  August 2011 $-$ December 2011 \\
  \\
  Undergraduate Research Assistant\\
  Research Experience for Undergraduates (REU)\\
  Duke University, Durham, NC \\
  Department of Physics \\
  May 2010 $-$ August 2010
 
\end{tabular}


\section*{Publications}
\begin{tabular} {l}
  Brown, M. A.-P., et al. "New result for the neutron $\beta$-asymmetry \\
  parameter $ A_0 $ from UCNA." Physical Review C 97 (2018): 035505. \\
  \\
  Hickerson, K. P., et al. "First direct constraints on Fierz interference \\
  in free-neutron β decay." Physical Review C 96.4 (2017): 042501.\\
  \\
  Broussard, L. J., et al. "Detection system for neutron β decay correlations\\
  in the UCNB and Nab experiments." Nuclear Instruments and Methods in Physics\\
  Research Section A: Accelerators, Spectrometers, Detectors and Associated\\
  Equipment 849 (2017): 83-93.\\
  \\
  Nouri, N., et al. "Frequency Shifts Induced by Field Gradients in Muon $g-2$\\
  Experiments." arXiv preprint arXiv:1607.04691 (2016).\\
  \\
  Nouri, N., et al. "A prototype vector magnetic field monitoring system for \\
  a neutron electric dipole moment experiment." Journal of Instrumentation\\
  10.12 (2015): P12003.\\
  \\
  MacMullin, S., et al. "Measurement of the elastic scattering cross section \\
  of neutrons from argon and neon." Physical Review C 87.5 (2013): 054613.
  
\end{tabular}

\end{document}
